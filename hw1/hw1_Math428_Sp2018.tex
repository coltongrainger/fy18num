\documentclass [12pt]{article}
\setlength{\topmargin}{-0.5cm} \setlength{\oddsidemargin}{0.0cm}
\setlength{\evensidemargin}{0.0cm} \setlength{\textwidth}{6in}
\setlength{\textheight}{9in}

\usepackage{latexsym,fleqn}
\usepackage{graphicx}

\begin{document}

\def\e{\mathop{\rm e}\nolimits}
\noindent
\begin{center}
{ \bf  {Math/Phys/Engr 428, Math 529/Phys 528 \\
Numerical Methods - Spring 2018 }}\\[7pt]
\underline{\bf Homework 1}\\

%Assigned: Friday, January 20, 2012\\
Due: {\bf Friday, January 26, 2018}

\end{center}

\begin{itemize}
  \item Include a cover page and a problem sheet.

  \item Always clearly label all plots (title, $x$-label, $y$-label,
    and legend).

  \item Use the {\tt subplot} command from MATLAB when comparing 2 or
    more plots to make comparisons easier and to save paper.

  \item Include all of your script.

  \item Place a comment at the top of each function
    or script that you submit which includes the name of the
    function or script, your name, the date and MATH 428, PHYS 428, ENGR 428 or MATH 529, PHYS 528.

\end{itemize}

%\noindent {\bf BOOK PROBLEMS:}  \#6 pg.40, \#12 pg.52 \\

%\noindent {\bf ADDITIONAL PROBLEMS:} \\

\noindent {\bf PROBLEMS:}

\begin{enumerate}

\item Do the following calculations by hand.

\smallskip
(a) Convert to base 10: $(1011011.001)_2$

\smallskip
(b) Convert to base 2: $(2018)_{10}$

\item The floating point representation of a real number is $x=\pm
(0.d_{1}d_{2}\ldots d_{n})_\beta\cdot \beta^e$, where
$d_{1}\not=0$, $-M\leq e\leq M$. Suppose that $\beta=2$, $n=6$,
$M=5$.

\begin{enumerate}

\item Find the smallest (positive) and largest floating point
numbers that can be represented. Give the answers in decimal form.

\item Find the floating point number in this system which is
closest to $\pi$.

\end{enumerate}

\item Near certain values of $x$ each of the following functions
cannot be accurately  computed using the formula as given due to
cancellation error. Identify the values of $x$ which are involved
(e.g. near $x=0$ or large positive $x$) and propose a
reformulation of the function (e.g., using Taylor series,
rationalization, trigonometric identities, etc.) to remedy the
problem. This is problem \# 12 from the textbook. Please also see pages 48-49 for examples and more details.

\[
\begin{array}{ll}
\mbox{\bf (a)} \ f(x)=1+\cos x & \hspace{50pt} \mbox{\bf (b)} \
f(x)=\e^{-x}+\sin x-1 \\
\mbox{\bf (c)} \ f(x)=\ln x-\ln(1/x) & \hspace{50pt} \mbox{\bf
(d)} \
f(x)=\sqrt{x^2+1}-\sqrt{x^2+4} \\
\mbox{\bf (e)} \ f(x)=1-2\sin^2x & \hspace{50pt} \mbox{\bf (f)} \
f(x)=\ln(x+\sqrt{x^2+1}) %\\
%\mbox{\bf (g)} \ f(x)=x-\sin x & \hspace{50pt} \mbox{\bf (h)} \
%f(x)=\ln x-1 \\
\end{array}
\]

%\item 
% Consider the function $f(x)=1+\cos(x+2)$, $x_0=0$. Approximate the
% derivative $f'(x_0)$ by computing $Df=\frac{f(x_0+h)-f(x_0-h)}{2h}$
% for a sequence of increasing small $h$ values, for example,
% $h=10^{-n}$, $n=1,\ldots,25$. Plot the error against $h$ using the logarithmic scale (use command {\tt loglog} instead of {\tt plot}) and
% explain your observations. 
% What is the theoretical order of accuracy (rate of convergence) of this approximation? Make a table with the following information.
% 
%{\small
%\begin{tabular}{|c|c|c|c|c|c|c|}
%\hline
% n & $h$  & $Df$ & $f'(0)-Df$ & $(f'(0)-Df)/h$ & $(f'(0)-Df)/h^2$ & $(f'(0)-Df)/h^3$\\
% \hline
%\end{tabular}
%}\\\\
%\noindent Discuss the results and determine numerically the
%order of accuracy. Does it agree with the theoretically predicted one? Do you observe the effect of the loss of precision? If yes, what is the reason of this?

%
%\item Some problems using Taylor polynomials.

%\begin{enumerate}

%\item Consider a function $f(x)=\ln{x}$. Find the fourth order
%Taylor polynomial about the point $x_0=2$. Then determine the
%minimum number of terms (i.e. $n$) that would guarantee an
%absolute error of no more than $10^{-3}$ in the Taylor
%approximation for $f(x)$ for all $x$ in the interval $[1, \, 2]$.

%
%\item Show that

%\[
%f''(x)=\frac{f(x-h)-2f(x)+f(x+h)}{h^2}+O(h^2)
%\]
%for sufficiently small $h$ by expanding $f(x\pm h)$ out to at
%least fourth order using Taylor polynomials.

%\end{enumerate}

%\item Compute the following limits and determine the corresponding rates of convergence.
%\begin{enumerate}
%\item $\lim_{n\to\infty}$ $\frac{n^2+1}{n^5-2n}$
%\item $\lim_{n\to\infty}$ $\left(\sqrt{n+1}-\sqrt{n}\right)$
%\item $\lim_{n\to\infty}$ $\frac{\sin n}{n}$ 
%\end{enumerate}

\item The following algorithm

step 1: $x_0 := x$; $j:=0$

step 2: {\it while $x_j \not= 0$, do}

\hskip 60pt $a_j :=$ remainder of integer divide $x_j/2$

\hskip 60pt $x_{j+1} :=$ quotient of integer divide $x_j/2$

\hskip 60pt $j:=j+1$

\hskip 40pt {\it end while}

\noindent can be used to convert a positive decimal integer $x$ to
its binary equivalent,
$$x=(a_na_{n-1}\cdots a_1a_0)_2.$$

%$$x=a_n\cdot 2^n +a_{n-1}\cdot 2^{n-1} +\cdots + a_1\cdot 2^1 +a_0\cdot 2^0$$

\smallskip
\noindent Implement the algorithm (write a computer program) and
apply it to convert the following integers to their binary
equivalents.

(a) 56 \hskip 15pt (b) 1543

\bigskip

\smallskip
\noindent (The Matlab library functions {\it rem}, {\it mod} and
{\it floor} might be helpful when you use Matlab. Try {\bf help
rem}, {\bf help mod} and {\bf help floor} to see how to use them.)

\bigskip

\hspace*{-15pt}
{\bf  Finite Precision Arithmetic}
%\begin{enumerate}

\item Use three-digit rounding arithmetic to compute the following
sums (sum in the given order):

\[
\mbox{\bf(a)} \quad \sum_{k=1}^6\frac{1}{3^k} \qquad \qquad
\mbox{\bf(b)} \quad \sum_{k=1}^6\frac{1}{3^{7-k}}
\]

\underline{Hint}: use floating point representation, e.g. $\frac{1}{3^6}=0.001371742112483 \approx 0.137\times 10^{-2}$ but not $0.001$; answers in (a) and (b) should be slightly different. 

\bigskip

\hspace*{-15pt}
{\bf Finite-Difference Approximation}

\item Let $f(x)$ be a given function and recall the forward difference approximation of $f'(x)$:
%
\[
D_+f(x)=\frac{f(x+h)-f(x)}{h},
\]
%
where $h>0$ is the step size.

{\bf }

\begin{enumerate}

\item Take $f(x)=\sin x$, $x=\pi/4$, $h=2^{-n}$ for $n=1,2,\ldots,6$. Following example in class, plot the error versus $h$ (use command {\tt loglog} instead of {\tt plot}) and make a table with the following information: 

\medskip

%column 1: $h$, column 2: $D_+f(x)$, column 3: $f'(x)-D_+f(x)$, column 4: $\biggl(f'(x)-D_+f(x)\biggr)/h$, column 5: $\biggl(f'(x)-D_+f(x)\biggr)/h^2$:  

{\small
\hspace*{-20pt}
\begin{tabular}{|c|c|c|c|c|c|c|}
\hline
$h$  & $D_+f$ & $f'(\frac\pi 4)-D_+f$ & $(f'(\frac\pi 4)-D_+f)/h$ & $(f'(\frac\pi 4)-D_+f)/h^2$ & $(f'(\frac\pi 4)-D_+f)/h^3$\\
 \hline
\end{tabular}
}\

You may modify the Matlab code shown in class. Present at least eight decimal digits (type ``{\tt format long}" in Matlab to get the full 15 digits). 

\item Repeat for central difference approximation,
%
\[
D_0f(x)=\frac{f(x+h)-f(x-h)}{2h},
\]
%
which also approximates $f'(x)$. Which approximation is more accurate? Explain why.
\end{enumerate}

\item The forward and backward finite-difference operators are defined by
%
\[
D_+f(x)=\frac{f(x+h)-f(x)}{h}, \qquad D_-f(x)=\frac{f(x)-f(x-h)}{h}.
\]
%

\begin{enumerate}

\item Show that $\displaystyle D_+D_-f(x)=\frac{f(x+h)-2f(x)+f(x-h)}{h^2}$.

\item Use Taylor expansions and the result in part (a) to show that $D_+D_-f(x)=f''(x)+O(h^2)$. Find the asymptotic error constant.

\end{enumerate}

\hspace*{-15pt}
{\bf Rootfinding}

\item Consider $f(x) = x^2 - 5$. Since $f(2) < 0$, $f(3) > 0$, it follows that $f(x)$ has a root $p$ in the
interval $(2, 3)$. Compute an approximation to $p$ by the following methods. Take $10$ steps in
each case. Use Matlab and print the answers to $15$ digits.

\begin{enumerate}

\item bisection method, starting interval $[a, b] = [2, 3]$;

\item fixed-point iteration with $g_1(x) = 5/x$ and $g_2(x) = x - f(x)/3$, starting value $x_0 = 2.5$;

\item Newton's method, starting value $x_0 = 2.5$.

\end{enumerate}

Present the results in a table with columns as below for each method. Do the results agree
with the theory discussed in class?

\hspace*{20pt} column 1 : $n$ (step)
\\[5pt]
\hspace*{20pt}  column 2 : $x_n$ (approximation)
\\[5pt]
\hspace*{20pt}  column 3 : $f(x_n)$  (residual)
\\[5pt]
\hspace*{20pt}  column 4 : $|p - x_n|$  (error)

\item Consider the function $g(x)=2x(1-x)$.

\begin{enumerate}

\item Verify $x=0$ and $x=1/2$ are fixed points of $g(x)$.

\item Why should we expect that fixed point iteration, starting even with a value very close to zero, will fail to converge toward $x=0$?

\item Why should we expect that fixed point iteration, starting with $p_0\in(0,1)$ will converge toward $x=1/2$? What order of convergence should we expect?

\item Perform seven iterations starting from an arbitrary $p_0\in(0,1)$ and numerical confirm the order of convergence.

\end{enumerate}

%\end{enumerate}

%\end{enumerate}

%%%%%%%%%%%%%%%%%%

\smallskip
\noindent {\it (Please include your program for completeness.)}

%\end{document}

%\newpage
\bigskip

\item{{\bf Suggested / Additional problems for Math 529/Phys 528 students}}: 

\begin{enumerate}

\item Plot the function $f(x)=1-\cos x$ over the interval $-5\times 10^{-8}\leq x\leq 5\times 10^{-8}$. Generate points at 1001 uniformly spaced abscissas and perform all calculations in IEEE standard double precision (in Matlab, for example).

\item Reformulate $f$ to avoid cancellations error and then repeat part (a).
\end{enumerate}

\item %{{\bf Suggested / Additional problem for Phys 528 students}}: 

Verify that $x=\sqrt{a}$ is a fixed point of the function
%
\[
g(x)=\frac 12 \left(x+\frac ax\right).
\]
%
Use the techniques of Section 2.3 to determine the order of convergence and the asymptotic error constant of the sequence $p_n=g(p_{n-1})$ toward $x=\sqrt a$.

\end{enumerate}

\end{document}


\item  Let $f(x)=\cos(x+2)$. Compute $f'(0)$ using the difference
quotients given below and step-size $h=2^{-n}$, $n=1,\ldots,5$.

\begin{tabular}{ll}
\mbox{a)} & $D_+f=\biggl(f(x+h)-f(x)\biggr)/h$ \\[10pt]
\mbox{b)} & $D_0f=\biggl(f(x+h)-f(x-h)\biggr)/2h$
\end{tabular}

For each difference formula, make a table which contains the
following information.

\begin{tabular}{ll}
\mbox{column 1:} & $h$ \\[5pt]
\mbox{column 2:} & $Df$ \\[5pt]
\mbox{column 3:} & $f'(0)-Df$ \\[5pt]
\mbox{column 4:} & $(f'(0)-Df)/h$ \\[5pt]
\mbox{column 5:} & $(f'(0)-Df)/h^2$ \\[5pt]
\mbox{column 6:} & $(f'(0)-Df)/h^3$
\end{tabular}

\noindent Discuss your results (\underline{Hint}: order of
convergence).

\item  Show that $D_+D_-f=f''(x)+O(h^2)$, where

\[
D_+=\frac{f(x+h)-f(x)}{h}, \qquad  D_-=\frac{f(x)-f(x-h)}{h}
\] \[
D_+D_-f=\displaystyle \frac{f(x-h)-2f(x)+f(x+h)}{h^2}
\]

\underline{Hint:} Expand $f(x \pm h)$ for sufficiently small $h$
to at least fourth order using Taylor polynomials.
