\documentclass [12pt]{article}
\setlength{\topmargin}{-0.5cm} \setlength{\oddsidemargin}{0.0cm}
\setlength{\evensidemargin}{0.0cm} \setlength{\textwidth}{6in}
\setlength{\textheight}{9in}

\usepackage{latexsym,fleqn,mathrsfs}
\usepackage{graphicx}
\usepackage{amsmath,amsthm,amsfonts,amscd}

\begin{document}

\def\e{\mathop{\rm e}\nolimits}
\def\abs{\mathop{\rm abs}\nolimits}
\def\sign{\mathop{\rm sign}\nolimits}
\font\bb=msbm10 scaled \magstep1 % Blackboard bold for real numbers field
                                 % and matrices
\def\R{\hbox{\bb R}}

\noindent
\begin{center}
{ \bf  {Math/Phys/Engr 428, Math 529/Phys 528 \\
Numerical Methods - Spring 2018 }}\\[7pt]
\underline{\bf Homework 5}\\

%Assigned: Friday, January 20, 2012\\
Due: {\bf Monday, April 9, 2018}

\end{center}

\begin{enumerate}

%%%%%%%%%%%%%%%%%%%%%%%%%%%%%%%%%%%
%	Problem 1
%%%%%%%%%%%%%%%%%%%%%%%%%%%%%%%%%%%
\item \textbf{(Natural Splines)}  

Find the natural cubic spline $S(x)$ satisfying \[
S(0)=0,\qquad S(1/2)=1,\qquad S(1)=0\,.\]
 Your answer will be 2 cubic polynomials, $S_{0}(x)$, $S_{1}(x)$.
\textbf{Verify} that your answer satisfies all the necessary
conditions (interpolation, continuity of 1st and 2nd derivatives,
boundary conditions).

\item The following data describe the shape of a car called
``Buggy":

\medskip

\begin{tabular}{cccccccccccccccc}
\hspace*{-50pt}x=[0.0 & 0.5 & 1.0 & 1.5 & 1.7 & 1.85 & 2.0 & 2.5 &
3.0 & 3.5 &
4.0 & 4.5 & 5.0 & 5.5 & 5.75 & 6.0]; \\[5pt]
\hspace*{-50pt}y=[0.0 & 0.9 & 1.2 & 1.35 & 1.4 & 1.7 & 1.95 & 2.3
& 2.35 & 2.4 & 2.35 & 2.25 & 1.8 & 1.0 & 0.7 & 0.0 ];
\end{tabular}

\medskip

\begin{tabular}{cccccccccccccccc}
\hspace*{-50pt} v=[0.0 & 0.5 & 1.0 & 1.25 & 1.5 & 1.75 & 2.0 &
2.25 & 2.5 & 2.75 & 3.0 & 3.25 & 3.5 & 3.75 & 4.0 \\[5pt]
\hspace*{-40pt}  4.25 & 4.5 & 4.75 & 5.0 & 5.25 & 5.5 &
5.75 & 6.0]; &&&&&& \\[10pt]
\hspace*{-50pt} w=[0.0 & 0.0 & 0.0 & 0.0 & 0.0 & 0.45 & 0.6 & 0.45
& 0.0 & 0.0 & 0.0 & 0.0 &  0.0 & 0.0 & 0.0 \\[5pt]
\hspace*{-40pt} 0.45 & 0.6 & 0.45 & 0.0 & 0.0 & 0.0 & 0.0 & 0.0];
&&&&&&
\end{tabular}

\medskip

\noindent
Points $(x,y)$ describe the upper part of the car while
points $(v,w)$ give the lower part of the car.

Approximate the shape of the car using

\begin{enumerate}

\item polynomial interpolant;

\item cubic spline interpolant with ``not-a-knot" boundary conditions.

\end{enumerate}
Which approximation is more accurate? Why?

\underline{Note}. The Matlab commands ``polyfit", ``polyval" and
``spline" may be useful.

\item   \textbf{(Newton--Cotes).}  Suppose that $f$ is a function with four continuous derivatives on the interval $[a, b]$.  Recall that the error bound for the composite trapezoidal rule $T(h)$ with panel width $h$ is

$$
T(h) - \int_a^b f(x) dx = \frac{(b - a)\, h^2}{12} f''(\xi)
$$ 

for some $\xi \in [a, b]$.  The error in the composite Simpson's rule $S(h)$ with panel width $h$ is

$$
S(h) - \int_a^b f(x) dx = \frac{(b - a)\, h^4}{180} f^{(4)}(\xi)
$$

for some (different) $\xi \in [a, b]$.



\vspace{1pc}



\begin{enumerate}

\item  A quadrature rule has degree of precision $n$ if it correctly integrates all polynomials of degree $n$.  Use these error bounds to demonstrate that the composite trapezoidal rule has precision of degree one and that the composite Simpson's rule has precision of degree three.



\item   Let $f(x) = \e^{-x} \sin x$.  For the composite trapezoidal rule and the composite Simpson's rule, find the number of panels $n$ required to integrate $f$ on the interval $[0, 2\pi]$ with error at most $10^{-4}$.  Recall that $h = 2\pi/n$.  How many function evaluations are required in each case?


\item   Using the number of panels determined in the last part of the problem, use each rule to approximate the integral numerically.  Compare your results with the actual value of the integral.  For your information, an antiderivative of $f$ is

$$
\int \e^{-x} \sin x dx =
        -0.5\, \e^{-x} (\sin x + \cos x).
$$

%\item  Do Problem 22(a, b, d) on page 481.  Present your results in a table, and provide a detailed discussion of the rate of convergence.
\end{enumerate}

%%%%%


\item  \textbf{(Richardson Extrapolation Applied to Differentiation).} 
\begin{enumerate}
\item Suppose that $N(h)$ is an approximation to $M$ for every
$h>0$ and that
\[
M=N(h)+K_{1}h^{1}+K_{2}h^{2}+K_{3}h^{3}+\ldots
\]
for some constants $K_{1}$, $K_{2}$, $K_{3}$, $\ldots$. Use the
values $N(h)$, $N(\frac{h}{3})$, and $N(\frac{h}{9})$ to produce
an ${\mathcal{O}}(h^{3})$ approximation to $M$.

\item Recall that
\[
\frac{df(x_{0})}{dx}=\frac{f(x_{0}+h)-f(x_{0})}{h}+
\sum_{i=2}^{\infty}\frac{h^{i-1}}{i!}f^{(i)}(x_{0})~.
\]
Use the formula you constructed in part (a) to construct an
$O(h^{3})$ approximation to $\frac{df(x_{0})}{dx}$.
\end{enumerate}


\item Consider the definite integral $\int_a^b \sin(\sqrt{\pi x})\,dx$. Numerically determine the rate of convergence of the composite trapezoid rule for each of the following integration intervals. (See problem \# 22, pg. 481 from Section 6.5, including discussion of Periodic Integrands)

\begin{enumerate}

\item $[a,b]=[0,1]$

\item $[a,b]=[\pi/4, 9\pi/4]$

\item $[a,b]=[\pi,2\pi]$

\item Explain any variation among the rates of convergence obtained in parts (a), (b), and (c).

\end{enumerate}

\item  \textbf{(Gram-Schmidt Orthogonalization Method).}  

\begin{enumerate}

\item Apply the Gram-Schmidt orthogonalization method to find the
4th degree Legendre polynomial $P_4(x)$. The first 3 were derived
in class and are:

\[
P_0=1, \qquad P_1=x
\] \[
P_2=x^2-\frac 13, \qquad P_3=x^3-\frac 35 x
\]

\item Express $x^4$ as a linear combination of the first four
Legendre polynomials $\{P_0,P_1,P_2,P_3,P_4\}$.

\end{enumerate}


\item \textbf{(Gaussian Integration).}  Consider the integral,
$$
\int_0^1 {x \exp ^{-x^2}} dx~.
$$

\begin{enumerate}

\item Use the 4-point Gaussian quadrature rule to approximate the
integral (after changing variables to obtain an integral over
$[-1,1]$).

\vskip 5pt \noindent
The points and weights are:

$$\begin{array}{ll}
x_1 = -0.861136311594053 \;\;\;\;\;\;\;\;\;\; & c_1 = 0.347854845137454 \\

x_2 = -0.339981043584856  & c_2 = 0.652145154862546 \\

x_3 = -x_2                & c_3 = c_2 \\

x_4 = -x_1                & c_4 = c_1
\end{array} $$

\item What value of $n$ would be needed to obtain the same
accuracy if the compound Trapezoid rule were used?

\end{enumerate}




\item  \textbf{(Gauss-Laguerre Quadrature).} 

The first three Laguerre polynomials are $L_0(x)=1$,
$L_1(x)=1-x$, and $L_2(x)=x^2-4x+2$.

\begin{enumerate}

\item Show that these polynomials are orthogonal over the interval
$(0,\infty)$ with respect to the weight function $w(x)=e^{-x}$.

\item It is easily seen that the roots of $L_2(x)$ are $x_{1,2}=2
\pm \sqrt{2}$.  Using the method of undetermined coefficients, and
the fact the 2-point Gauss-Laquerre quadrature rule
$$
\int_0^{\infty} f(x) e^{-x} \, dx \approx c_1f(x_1)+c_2 f(x_2)
$$
is exact for all polynomials of degree $\leq 3$, derive the
weights $c_{1,2}$. (They are $c_{1,2}=x_{2,1}/4$.)


\end{enumerate}

\item  \textbf{(Gauss-Legendre Quadrature).}  We proved in class that Gauss-Legendre Quadrature rule
$$
\int_{-1}^{1} f(x) \, dx \approx \sum_{j=1}^n c_j f(x_j)
\leqno{\rm (*)}
$$
is {\bf exact} for polynomials of degree $\leq 2n\!-\!1$, where
$\{x_j\}_{j=1}^{n}$ are the $n$ distinct roots of the Legendre
polynomial $p_n(x)$ of degree $n$, and $\{c_j\}_{j=1}^{n}$ the
corresponding weights. Show that indeed this is the best we can
expect by {\bf proving} that $(*)$ is {\bf not exact} for
$$
f(x)= \prod_{i=1}^n {(x-x_i)}^2 \,,
$$
a polynomial of degree $2n$. (Hint: Compute the approximation for
any $n$).

\bigskip

{\bf Suggested  / Additional problems for Math 529 / Phys 528 students}:

\item \textbf{(Clamped Splines)} 

 In the case of the clamped spline, the column vector of
unknowns is 
$m=(a_0 , a_1 , \ldots , a_{n-1} , a_n )^T$. Note that
the equations for $a_0$ and $a_n$ are no longer $a_0=0$ and
$a_n=0$, so that the tridiagonal matrix $B$ will change slightly.
One can write down the matrix and right hand side for the linear system
$Bm=f$ which determines $m$. The matrix $B$ is invertible,
and hence the clamped cubic spline exits and is unique. (Matrix $B$ is strictly diagonal dominant, hence, it is non-singular.)

Determine the clamped cubic spline $S(x)$ that interpolates
the data $f(0)=0$, $f(1)=1$, $f(2)=2$ and satisfies
$S'(0)=S'(2)=1$. Again, your answer will consist of 2 cubic
polynomials, $S_{0}(x)$, $S_{1}(x)$. \textbf{Verify} all the
necessary conditions and note that the boundary conditions for the
clamped spline are different from those for the natural spline.
Plot the spline over the interval $[0,2]$.

\item ({\bf Integration Quadrature})

\begin{enumerate}

\item Construct the quadrature formula for $\int_a^{b} f(x) dx$
using a second order polynomial approximation to $f(x)$.
    The polynomial should pass through the points $x_0=a$,
    $x_1=a+h+\varepsilon$, and $x_2=a+2h$, where $h=\frac{b-a}{2}$ and
    $\varepsilon \in (-\frac{a+b}{2},\frac{a+b}{2})$.

\item Show that for any choice of $\varepsilon$ other than
    $\varepsilon=0$, the method is $O(h^4)$ instead of being $O(h^5)$.

\underline{Note}: symbolic software like Maple, Maxima or Matlab symbolic toolbox should be useful.
\end{enumerate}


%\newpage

%{\bf Additional problem for Phys 528 students}:

\item \textbf{(Romberg Integration (see Section 6.7))} 

\begin{enumerate}

\item Starting with only one subinterval, construct the four row  Romberg integration table for $\int_{3}^{3.5} \frac{x}{\sqrt{x^2-4}}dx$ (see problem \# 8, pg. 503).

\item What is the error estimate for the final approximation? How does this compare with the actual error?

\item How many subintervals would have been necessary to achieve the same accuracy using the composite trapezoid rule without extrapolation?

\end{enumerate}

\end{enumerate}

\end{document}

