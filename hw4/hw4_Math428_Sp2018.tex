\documentclass [12pt]{article}
\setlength{\topmargin}{-0.5cm} \setlength{\oddsidemargin}{0.0cm}
\setlength{\evensidemargin}{0.0cm} \setlength{\textwidth}{6in}
\setlength{\textheight}{9in}

\usepackage{latexsym,fleqn,mathrsfs}
\usepackage{graphicx}
\usepackage{amsmath,amsthm,amsfonts,amscd}

\sloppy

\begin{document}

\def\e{\mathop{\rm e}\nolimits}
\def\abs{\mathop{\rm abs}\nolimits}
\def\sign{\mathop{\rm sign}\nolimits}
\font\bb=msbm10 scaled \magstep1 % Blackboard bold for real numbers field
                                 % and matrices
\def\R{\hbox{\bb R}}

\noindent
\begin{center}
{ \bf  {Math/Phys/Engr 428, Math 529/Phys 528 \\
Numerical Methods - Spring 2018 }}\\[7pt]
\underline{\bf Homework 4}\\

%Assigned: Friday, January 20, 2012\\
Due: {\bf Wednesday, March 21, 2018}

\end{center}

\begin{enumerate}

%%%%%%%%%%%%%%%%%%%%%%%%%%%%%%%%%%%
%	Problem 1
%%%%%%%%%%%%%%%%%%%%%%%%%%%%%%%%%%%
\item   \textbf{(Iterative Methods: Practice).}  This problem
extends \#8 and \#11 on page 235 of Bradie.  Consider the linear
system
$$
\begin{pmatrix}
4 & -1 & 0 \cr -1 & 4 & - 1 \cr 0 & -1 & 4 
\end{pmatrix}
\begin{pmatrix}
x_1 \cr x_2 \cr x_3 
\end{pmatrix}
= \begin{pmatrix}
 2 \cr 4 \cr 10 \end{pmatrix}.
$$
The optimal solution is $x = [1 \ 2 \ 3]^T$.

\begin{enumerate}
\item   Solve the system numerically using Jacobi's Method,
Gauss--Seidel, and SOR with (optimal) parameter $\omega = 2 / (1 +
\sqrt{7/8})$.  In each case, let $x^{(0)} = 0$.  Terminate
iteration after step $k$ if $||{ x^{(k)} - x^{(k-1)}}||_\infty <
10^{-7}$. Provide your code.

\item   Make a {\tt semilogy} plot that compares the evolution of
the error ${||x - x^{(k)}||}_\infty$ for all three methods.

\item   Use the formula
$$
C \approx \frac{|| {x^{(k)} - x^{(k-1)}}||_\infty }{ {||x^{(k-1)}
- x^{(k-2)}||} _\infty}
$$
to estimate the asymptotic error constant for each method.
Tabulate these estimates along with the number of iterations
required for each method to converge.
\end{enumerate}
Discuss your results.

%%%%%%%%%%%%%%%%%%%%%%%%%%%%%%%%%%%
%	Problem 2
%%%%%%%%%%%%%%%%%%%%%%%%%%%%%%%%%%%

\item \textbf{(Nonlinear Systems)} 

Write down Newton's method to
solve the system $ x^2+y^2=4$, $ x^2-y^2=1 $. Perform one step of
Newton's method with initial guess $x_0=1$, $y_0=1$.

%%%%%%%%%%%%%%%%%%%%%%%%%%%%%%%%%%%
%	Problem 3
%%%%%%%%%%%%%%%%%%%%%%%%%%%%%%%%%%%

\item \textbf{(Interpolation)}

The function $f(x) = \e^x$ is given at the 4 points: $x_0 =0$, $x_1
=1$, $x_2 =2$, $x_3 =3$.

\begin{enumerate}
\item
Write the interpolating polynomial in Lagrange form.

\item Write the interpolating polynomial in Newton form. (Recall that the interpolating polynomial is unique, so interpolating polynomial in Lagrange or Newton form should be  the same after simplification.)

\item Evaluate $\e^{1.5}$ and $\e^{4}$ using the interpolating
polynomial. Note that the first value is interpolated while the
second is extrapolated. Which approximate value is more accurate?

\item
Use the error formula to find an upper bound for the maximum
error
$$
{||f-p_3||}_{\infty} = \max_{0 \leq x \leq 3} | f(x)-p_3(x)| \,.
$$

\end{enumerate}


\item
The following data are taken from a polynomial $p(x)$ of degree $\leq 5$.
What is the degree of $p(x)$?  Explain. %\underline{Hint}: use a divided difference table.

\begin{center}
\begin{tabular}{c|cccccc}
$x$ & $-2$ & $-1$ & $0$ & $1$ & $2$ & $3$ \\ \hline
$p(x)$ & $-5$ & $1$ & $1$ & $1$ & $7$ & $25$
\end{tabular}
\end{center}


\item  Show that $\displaystyle \sum_{k=0}^{n} \ell_k(x) = 1 $ \hskip 10pt
(Hint: consider the function $f(x)=1$, a polynomial of degree 0! Don't forget to discuss the error term.)

\item
 We want to study the effect of different choices of interpolation points
$\{ x_0,x_1,\ldots,x_n \}$ on the function
$w_n(x)=(x-x_0)(x-x_1)\ldots(x-x_n)$ in the formula for the error
in interpolation polynomials. In particular, we want to study
evenly spaced points and Chebyshev points in the interval
$[-1,\,1]$. Consider the following choices:

\begin{enumerate}

\item  $x_i=-1+{\displaystyle \frac{2i}{n}}$ \hskip 15pt
$i=0,\ldots,n$

\item  $x_i=-\cos \left[{\displaystyle
\frac{\pi}{n+1}\left(\frac{1}{2} + i\right)}\right]$ \hskip 15pt
$i=0,\ldots,n$.

\end{enumerate}

In each case, plot $w_{n}(x)$ in the interval [-1,\,1] for $n=10$ using fine enough resolution (with the number of points more than $n$). Also plot $w_{n}(x)$ with $n=20$ and $30$. Discuss the results. 

\item Write a computer program to perform polynomial interpolation
using equally spaced points and the Chebyshev points on the
interval [-1,\,1] for the function $f(x)$. Investigate the
convergence of $p_n$ to $f$ by running the program for $n = 8, \ 16, \
32$ in the following cases
$$
f_1 (x) = |x| ~~ ,~\hbox{and} \quad
f_2 (x) = 
\begin{cases}
-1 & \ \mbox{if} \quad x < 0,\cr
                  0 \ & \ \mbox{if} \quad x = 0,\cr
                  1 \ & \ \mbox{if}  \quad x> 0.\cr
\end{cases}
$$
%
Use a fine mesh to plot your interpolating polynomials. Discuss the results.
As $n$ gets larger, is there pointwise convergence?
Is convergence uniform in $x$?

\vskip 8pt
\noindent
(in MATLAB $|x|=\abs(x)$ and the second function is $\sign(x)$, you can use the
library function ``polyfit" in MATLAB. Use ``help polyfit" to find how to use it).

\newpage

{(\bf Hermite Interpolation)}


\item \label{Hermite_error} 
The theorem describing
the error in using Hermite interpolation is as follows.

\textbf{Theorem:} If $f\in\mathscr{C}^{2n+2}[a,b]$, then

\[
f(x)=H(x)+\frac{(x-x_{0})^{2}\ldots(x-x_{n})^{2}}{(2n+2)!}f^{(2n+2)}(\xi)
\]
for some $\xi$ with $a<\xi<b$.\\
\\
Now, %for the function from problem \# 3 on page 414:
consider $f(x)=x \ln x$, $n=1$, $x_{0}=1$, and $x_{1}=3$ (from
problem \# 3 on page 414).

\begin{enumerate}

\item Use linear interpolation and Hermite interpolation to
approximate the value of $f(1.5)$. Which estimate is more
accurate?

\item Verify that the error bound for Hermite interpolation holds
for the Hermite polynomial found in (a). % (\ref{Hermite_error}).

\end{enumerate}

\item Find a polynomial of least degree satisfying: 
%
\[
p(1)=-1, \quad p'(1)=2, \quad p''(1)=0, \quad p(2)=1, \quad p'(2)=-2
\]
%
\underline{Note:} Extend the idea from Hermite interpolation to the case when more than two conditions are specified at the same point (see Lecture \# 26).

\bigskip

{\bf Suggested  /  Additional problems for Math 529 / Phys 528 students}:

\item Consider Hermite interpolation for $n=1$, $x_{0}=1$, and
$x_{1}=3$. Compute (by hand) $\tilde{h}_{1}(x)$ using the Lagrange
polynomials and using the Newton form (from the divided difference
table) and then plot $\tilde{h}_{1}(x)$.

%\medskip

%{\bf Additional problems for Phys 528 students}:

\item  {\bf  Iterative Methods}

\begin{enumerate}

\item Use both the Jacobi and the Gauss-Seidel method to solve the linear system of equations
%
\begin{eqnarray}
4x_1+x_2+x_3+x_4 & = & -5 \nonumber \\
x_1+8x_2+2x_3 +3x_4&=& 23 \nonumber \\
x_1+2x_2-5x_3  &=& 9 \nonumber \\
-x_1+2x_3+4x_4 &= & 4 \nonumber 
\end{eqnarray}
%
Take ${\mathbf x}^{(0)}={\mathbf 0}$, and terminate iteration when $||{\mathbf x}^{(k+1)}-{\mathbf x}^{(k)}||_\infty$ falls below $5\times 10^{-6}$. Record the number of iterations required to achieve convergence.

\item For the system in (a), generate a plot of the number of iterations required by the SOR method to achieve convergence as a function of the relaxation parameter $w$. Take ${\mathbf x}^{(0)}={\mathbf 0}$, and terminate iteration when $||{\mathbf x}^{(k+1)}-{\mathbf x}^{(k)}||_\infty$ falls below $5\times 10^{-6}$. Over roughly what range of  $w$ values does the SOR method outperforms the Gauss-Seidel? the Jacobi method?

\end{enumerate}


\item {\bf Chebyshev polynomials}

The Chebyshev polynomials are defined for $x\in [-1,1]$ by
$\hbox{T}_n (x)=\cos (n\theta )$, $x=\cos \theta$.

\begin{enumerate}

\item Derive the 3-term recurrence relation,
$$
\hbox{T}_{n+1}(x) = 2x\hbox{T}_n (x) - \hbox{T}_{n-1} (x)~.
$$

\item Given $\hbox{T}_0 (x)=1$ and $\hbox{T}_1 (x)=x$, use the
recurrence relation to find $\hbox{T}_2 (x)$ and $\hbox{T}_3 (x)$.

\item What are the roots of $T_3(x)$?

\end{enumerate}

\end{enumerate}

\end{document}

