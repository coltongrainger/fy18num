



    
\documentclass{article}
    
    \usepackage{parskip}
    \setcounter{secnumdepth}{0} %Suppress section numbers
    \usepackage[breakable]{tcolorbox}
    \tcbset{nobeforeafter}
    \usepackage{needspace}
    
    \usepackage{amsmath} % Equations
    \usepackage{amssymb} % Equations
    \usepackage[T1]{fontenc}
    \usepackage[sfdefault,scaled=0.85]{FiraSans}
    \usepackage[small,euler-digits]{eulervm}
    % We will generate all images so they have a width \maxwidth. This means
    % that they will get their normal width if they fit onto the page, but
    % are scaled down if they would overflow the margins.
    \usepackage{graphicx}
    \makeatletter
    \def\maxwidth{\ifdim\Gin@nat@width>\linewidth\linewidth
    \else\Gin@nat@width\fi}
    \makeatother
    \let\Oldincludegraphics\includegraphics
    % Set max figure width to be 80% of text width, for now hardcoded.
    \renewcommand{\includegraphics}[1]{\Oldincludegraphics[width=.6\maxwidth]{#1}}
    % Ensure that by default, figures have no caption (until we provide a
    % proper Figure object with a Caption API and a way to capture that
    % in the conversion process - todo).
    \usepackage{caption}
    \DeclareCaptionLabelFormat{nolabel}{}
    \captionsetup{labelformat=nolabel}

    \usepackage{adjustbox} % Used to constrain images to a maximum size 
    \usepackage{xcolor} % Allow colors to be defined
    \usepackage{enumerate} % Needed for markdown enumerations to work
    \usepackage{geometry} % Used to adjust the document margins
    \usepackage{textcomp} % defines textquotesingle
    % Hack from http://tex.stackexchange.com/a/47451/13684:
    \AtBeginDocument{%
        \def\PYZsq{\textquotesingle}% Upright quotes in Pygmentized code
    }
    \usepackage{upquote} % Upright quotes for verbatim code
    \usepackage{eurosym} % defines \euro
    \usepackage[mathletters]{ucs} % Extended unicode (utf-8) support
    \usepackage[utf8x]{inputenc} % Allow utf-8 characters in the tex document
    \usepackage{fancyvrb} % verbatim replacement that allows latex
    \usepackage{grffile} % extends the file name processing of package graphics 
                         % to support a larger range 
    % The hyperref package gives us a pdf with properly built
    % internal navigation ('pdf bookmarks' for the table of contents,
    % internal cross-reference links, web links for URLs, etc.)
    \usepackage{hyperref}
    \usepackage{longtable} % longtable support required by pandoc >1.10
    \usepackage{booktabs}  % table support for pandoc > 1.12.2
    \usepackage[inline]{enumitem} % IRkernel/repr support (it uses the enumerate* environment)
    \usepackage[normalem]{ulem} % ulem is needed to support strikethroughs (\sout)
                                % normalem makes italics be italics, not underlines
    

    \let\Oldtex\TeX     % provide compatibility with nbconvert <= 5.3.1
    \let\Oldlatex\LaTeX % pre-included in nbconvert > 5.3.1 so redundant
    
    % Colors for the hyperref package
    \definecolor{urlcolor}{rgb}{0,.145,.698}
    \definecolor{linkcolor}{rgb}{.71,0.21,0.01}
    \definecolor{citecolor}{rgb}{.12,.54,.11}

    % ANSI colors
    \definecolor{ansi-black}{HTML}{3E424D}
    \definecolor{ansi-black-intense}{HTML}{282C36}
    \definecolor{ansi-red}{HTML}{E75C58}
    \definecolor{ansi-red-intense}{HTML}{B22B31}
    \definecolor{ansi-green}{HTML}{00A250}
    \definecolor{ansi-green-intense}{HTML}{007427}
    \definecolor{ansi-yellow}{HTML}{DDB62B}
    \definecolor{ansi-yellow-intense}{HTML}{B27D12}
    \definecolor{ansi-blue}{HTML}{208FFB}
    \definecolor{ansi-blue-intense}{HTML}{0065CA}
    \definecolor{ansi-magenta}{HTML}{D160C4}
    \definecolor{ansi-magenta-intense}{HTML}{A03196}
    \definecolor{ansi-cyan}{HTML}{60C6C8}
    \definecolor{ansi-cyan-intense}{HTML}{258F8F}
    \definecolor{ansi-white}{HTML}{C5C1B4}
    \definecolor{ansi-white-intense}{HTML}{A1A6B2}

    % commands and environments needed by pandoc snippets
    % extracted from the output of `pandoc -s`
    \providecommand{\tightlist}{%
      \setlength{\itemsep}{0pt}\setlength{\parskip}{0pt}}
    \DefineVerbatimEnvironment{Highlighting}{Verbatim}{commandchars=\\\{\},fontsize=\small}
    % Add ',fontsize=\small' for more characters per line
    \newenvironment{Shaded}{}{}
    \newcommand{\KeywordTok}[1]{\textcolor[rgb]{0.00,0.44,0.13}{\textbf{{#1}}}}
    \newcommand{\DataTypeTok}[1]{\textcolor[rgb]{0.56,0.13,0.00}{{#1}}}
    \newcommand{\DecValTok}[1]{\textcolor[rgb]{0.25,0.63,0.44}{{#1}}}
    \newcommand{\BaseNTok}[1]{\textcolor[rgb]{0.25,0.63,0.44}{{#1}}}
    \newcommand{\FloatTok}[1]{\textcolor[rgb]{0.25,0.63,0.44}{{#1}}}
    \newcommand{\CharTok}[1]{\textcolor[rgb]{0.25,0.44,0.63}{{#1}}}
    \newcommand{\StringTok}[1]{\textcolor[rgb]{0.25,0.44,0.63}{{#1}}}
    \newcommand{\CommentTok}[1]{\textcolor[rgb]{0.38,0.63,0.69}{\textit{{#1}}}}
    \newcommand{\OtherTok}[1]{\textcolor[rgb]{0.00,0.44,0.13}{{#1}}}
    \newcommand{\AlertTok}[1]{\textcolor[rgb]{1.00,0.00,0.00}{\textbf{{#1}}}}
    \newcommand{\FunctionTok}[1]{\textcolor[rgb]{0.02,0.16,0.49}{{#1}}}
    \newcommand{\RegionMarkerTok}[1]{{#1}}
    \newcommand{\ErrorTok}[1]{\textcolor[rgb]{1.00,0.00,0.00}{\textbf{{#1}}}}
    \newcommand{\NormalTok}[1]{{#1}}
    
    % Additional commands for more recent versions of Pandoc
    \newcommand{\ConstantTok}[1]{\textcolor[rgb]{0.53,0.00,0.00}{{#1}}}
    \newcommand{\SpecialCharTok}[1]{\textcolor[rgb]{0.25,0.44,0.63}{{#1}}}
    \newcommand{\VerbatimStringTok}[1]{\textcolor[rgb]{0.25,0.44,0.63}{{#1}}}
    \newcommand{\SpecialStringTok}[1]{\textcolor[rgb]{0.73,0.40,0.53}{{#1}}}
    \newcommand{\ImportTok}[1]{{#1}}
    \newcommand{\DocumentationTok}[1]{\textcolor[rgb]{0.73,0.13,0.13}{\textit{{#1}}}}
    \newcommand{\AnnotationTok}[1]{\textcolor[rgb]{0.38,0.63,0.69}{\textbf{\textit{{#1}}}}}
    \newcommand{\CommentVarTok}[1]{\textcolor[rgb]{0.38,0.63,0.69}{\textbf{\textit{{#1}}}}}
    \newcommand{\VariableTok}[1]{\textcolor[rgb]{0.10,0.09,0.49}{{#1}}}
    \newcommand{\ControlFlowTok}[1]{\textcolor[rgb]{0.00,0.44,0.13}{\textbf{{#1}}}}
    \newcommand{\OperatorTok}[1]{\textcolor[rgb]{0.40,0.40,0.40}{{#1}}}
    \newcommand{\BuiltInTok}[1]{{#1}}
    \newcommand{\ExtensionTok}[1]{{#1}}
    \newcommand{\PreprocessorTok}[1]{\textcolor[rgb]{0.74,0.48,0.00}{{#1}}}
    \newcommand{\AttributeTok}[1]{\textcolor[rgb]{0.49,0.56,0.16}{{#1}}}
    \newcommand{\InformationTok}[1]{\textcolor[rgb]{0.38,0.63,0.69}{\textbf{\textit{{#1}}}}}
    \newcommand{\WarningTok}[1]{\textcolor[rgb]{0.38,0.63,0.69}{\textbf{\textit{{#1}}}}}
    
    
    % Define a nice break command that doesn't care if a line doesn't already
    % exist.
    \def\br{\hspace*{\fill} \\* }
    % Math Jax compatability definitions
    \def\gt{>}
    \def\lt{<}
    % Document parameters
    \title{2018-07-19-interpolation}
    
    
    
% Pygments definitions
\makeatletter
\def\PY@reset{\let\PY@it=\relax \let\PY@bf=\relax%
    \let\PY@ul=\relax \let\PY@tc=\relax%
    \let\PY@bc=\relax \let\PY@ff=\relax}
\def\PY@tok#1{\csname PY@tok@#1\endcsname}
\def\PY@toks#1+{\ifx\relax#1\empty\else%
    \PY@tok{#1}\expandafter\PY@toks\fi}
\def\PY@do#1{\PY@bc{\PY@tc{\PY@ul{%
    \PY@it{\PY@bf{\PY@ff{#1}}}}}}}
\def\PY#1#2{\PY@reset\PY@toks#1+\relax+\PY@do{#2}}

\expandafter\def\csname PY@tok@gd\endcsname{\def\PY@tc##1{\textcolor[rgb]{0.63,0.00,0.00}{##1}}}
\expandafter\def\csname PY@tok@gu\endcsname{\let\PY@bf=\textbf\def\PY@tc##1{\textcolor[rgb]{0.50,0.00,0.50}{##1}}}
\expandafter\def\csname PY@tok@gt\endcsname{\def\PY@tc##1{\textcolor[rgb]{0.00,0.27,0.87}{##1}}}
\expandafter\def\csname PY@tok@gs\endcsname{\let\PY@bf=\textbf}
\expandafter\def\csname PY@tok@gr\endcsname{\def\PY@tc##1{\textcolor[rgb]{1.00,0.00,0.00}{##1}}}
\expandafter\def\csname PY@tok@cm\endcsname{\let\PY@it=\textit\def\PY@tc##1{\textcolor[rgb]{0.25,0.50,0.50}{##1}}}
\expandafter\def\csname PY@tok@vg\endcsname{\def\PY@tc##1{\textcolor[rgb]{0.10,0.09,0.49}{##1}}}
\expandafter\def\csname PY@tok@vi\endcsname{\def\PY@tc##1{\textcolor[rgb]{0.10,0.09,0.49}{##1}}}
\expandafter\def\csname PY@tok@vm\endcsname{\def\PY@tc##1{\textcolor[rgb]{0.10,0.09,0.49}{##1}}}
\expandafter\def\csname PY@tok@mh\endcsname{\def\PY@tc##1{\textcolor[rgb]{0.40,0.40,0.40}{##1}}}
\expandafter\def\csname PY@tok@cs\endcsname{\let\PY@it=\textit\def\PY@tc##1{\textcolor[rgb]{0.25,0.50,0.50}{##1}}}
\expandafter\def\csname PY@tok@ge\endcsname{\let\PY@it=\textit}
\expandafter\def\csname PY@tok@vc\endcsname{\def\PY@tc##1{\textcolor[rgb]{0.10,0.09,0.49}{##1}}}
\expandafter\def\csname PY@tok@il\endcsname{\def\PY@tc##1{\textcolor[rgb]{0.40,0.40,0.40}{##1}}}
\expandafter\def\csname PY@tok@go\endcsname{\def\PY@tc##1{\textcolor[rgb]{0.53,0.53,0.53}{##1}}}
\expandafter\def\csname PY@tok@cp\endcsname{\def\PY@tc##1{\textcolor[rgb]{0.74,0.48,0.00}{##1}}}
\expandafter\def\csname PY@tok@gi\endcsname{\def\PY@tc##1{\textcolor[rgb]{0.00,0.63,0.00}{##1}}}
\expandafter\def\csname PY@tok@gh\endcsname{\let\PY@bf=\textbf\def\PY@tc##1{\textcolor[rgb]{0.00,0.00,0.50}{##1}}}
\expandafter\def\csname PY@tok@ni\endcsname{\let\PY@bf=\textbf\def\PY@tc##1{\textcolor[rgb]{0.60,0.60,0.60}{##1}}}
\expandafter\def\csname PY@tok@nl\endcsname{\def\PY@tc##1{\textcolor[rgb]{0.63,0.63,0.00}{##1}}}
\expandafter\def\csname PY@tok@nn\endcsname{\let\PY@bf=\textbf\def\PY@tc##1{\textcolor[rgb]{0.00,0.00,1.00}{##1}}}
\expandafter\def\csname PY@tok@no\endcsname{\def\PY@tc##1{\textcolor[rgb]{0.53,0.00,0.00}{##1}}}
\expandafter\def\csname PY@tok@na\endcsname{\def\PY@tc##1{\textcolor[rgb]{0.49,0.56,0.16}{##1}}}
\expandafter\def\csname PY@tok@nb\endcsname{\def\PY@tc##1{\textcolor[rgb]{0.00,0.50,0.00}{##1}}}
\expandafter\def\csname PY@tok@nc\endcsname{\let\PY@bf=\textbf\def\PY@tc##1{\textcolor[rgb]{0.00,0.00,1.00}{##1}}}
\expandafter\def\csname PY@tok@nd\endcsname{\def\PY@tc##1{\textcolor[rgb]{0.67,0.13,1.00}{##1}}}
\expandafter\def\csname PY@tok@ne\endcsname{\let\PY@bf=\textbf\def\PY@tc##1{\textcolor[rgb]{0.82,0.25,0.23}{##1}}}
\expandafter\def\csname PY@tok@nf\endcsname{\def\PY@tc##1{\textcolor[rgb]{0.00,0.00,1.00}{##1}}}
\expandafter\def\csname PY@tok@si\endcsname{\let\PY@bf=\textbf\def\PY@tc##1{\textcolor[rgb]{0.73,0.40,0.53}{##1}}}
\expandafter\def\csname PY@tok@s2\endcsname{\def\PY@tc##1{\textcolor[rgb]{0.73,0.13,0.13}{##1}}}
\expandafter\def\csname PY@tok@nt\endcsname{\let\PY@bf=\textbf\def\PY@tc##1{\textcolor[rgb]{0.00,0.50,0.00}{##1}}}
\expandafter\def\csname PY@tok@nv\endcsname{\def\PY@tc##1{\textcolor[rgb]{0.10,0.09,0.49}{##1}}}
\expandafter\def\csname PY@tok@s1\endcsname{\def\PY@tc##1{\textcolor[rgb]{0.73,0.13,0.13}{##1}}}
\expandafter\def\csname PY@tok@dl\endcsname{\def\PY@tc##1{\textcolor[rgb]{0.73,0.13,0.13}{##1}}}
\expandafter\def\csname PY@tok@ch\endcsname{\let\PY@it=\textit\def\PY@tc##1{\textcolor[rgb]{0.25,0.50,0.50}{##1}}}
\expandafter\def\csname PY@tok@m\endcsname{\def\PY@tc##1{\textcolor[rgb]{0.40,0.40,0.40}{##1}}}
\expandafter\def\csname PY@tok@gp\endcsname{\let\PY@bf=\textbf\def\PY@tc##1{\textcolor[rgb]{0.00,0.00,0.50}{##1}}}
\expandafter\def\csname PY@tok@sh\endcsname{\def\PY@tc##1{\textcolor[rgb]{0.73,0.13,0.13}{##1}}}
\expandafter\def\csname PY@tok@ow\endcsname{\let\PY@bf=\textbf\def\PY@tc##1{\textcolor[rgb]{0.67,0.13,1.00}{##1}}}
\expandafter\def\csname PY@tok@sx\endcsname{\def\PY@tc##1{\textcolor[rgb]{0.00,0.50,0.00}{##1}}}
\expandafter\def\csname PY@tok@bp\endcsname{\def\PY@tc##1{\textcolor[rgb]{0.00,0.50,0.00}{##1}}}
\expandafter\def\csname PY@tok@c1\endcsname{\let\PY@it=\textit\def\PY@tc##1{\textcolor[rgb]{0.25,0.50,0.50}{##1}}}
\expandafter\def\csname PY@tok@fm\endcsname{\def\PY@tc##1{\textcolor[rgb]{0.00,0.00,1.00}{##1}}}
\expandafter\def\csname PY@tok@o\endcsname{\def\PY@tc##1{\textcolor[rgb]{0.40,0.40,0.40}{##1}}}
\expandafter\def\csname PY@tok@kc\endcsname{\let\PY@bf=\textbf\def\PY@tc##1{\textcolor[rgb]{0.00,0.50,0.00}{##1}}}
\expandafter\def\csname PY@tok@c\endcsname{\let\PY@it=\textit\def\PY@tc##1{\textcolor[rgb]{0.25,0.50,0.50}{##1}}}
\expandafter\def\csname PY@tok@mf\endcsname{\def\PY@tc##1{\textcolor[rgb]{0.40,0.40,0.40}{##1}}}
\expandafter\def\csname PY@tok@err\endcsname{\def\PY@bc##1{\setlength{\fboxsep}{0pt}\fcolorbox[rgb]{1.00,0.00,0.00}{1,1,1}{\strut ##1}}}
\expandafter\def\csname PY@tok@mb\endcsname{\def\PY@tc##1{\textcolor[rgb]{0.40,0.40,0.40}{##1}}}
\expandafter\def\csname PY@tok@ss\endcsname{\def\PY@tc##1{\textcolor[rgb]{0.10,0.09,0.49}{##1}}}
\expandafter\def\csname PY@tok@sr\endcsname{\def\PY@tc##1{\textcolor[rgb]{0.73,0.40,0.53}{##1}}}
\expandafter\def\csname PY@tok@mo\endcsname{\def\PY@tc##1{\textcolor[rgb]{0.40,0.40,0.40}{##1}}}
\expandafter\def\csname PY@tok@kd\endcsname{\let\PY@bf=\textbf\def\PY@tc##1{\textcolor[rgb]{0.00,0.50,0.00}{##1}}}
\expandafter\def\csname PY@tok@mi\endcsname{\def\PY@tc##1{\textcolor[rgb]{0.40,0.40,0.40}{##1}}}
\expandafter\def\csname PY@tok@kn\endcsname{\let\PY@bf=\textbf\def\PY@tc##1{\textcolor[rgb]{0.00,0.50,0.00}{##1}}}
\expandafter\def\csname PY@tok@cpf\endcsname{\let\PY@it=\textit\def\PY@tc##1{\textcolor[rgb]{0.25,0.50,0.50}{##1}}}
\expandafter\def\csname PY@tok@kr\endcsname{\let\PY@bf=\textbf\def\PY@tc##1{\textcolor[rgb]{0.00,0.50,0.00}{##1}}}
\expandafter\def\csname PY@tok@s\endcsname{\def\PY@tc##1{\textcolor[rgb]{0.73,0.13,0.13}{##1}}}
\expandafter\def\csname PY@tok@kp\endcsname{\def\PY@tc##1{\textcolor[rgb]{0.00,0.50,0.00}{##1}}}
\expandafter\def\csname PY@tok@w\endcsname{\def\PY@tc##1{\textcolor[rgb]{0.73,0.73,0.73}{##1}}}
\expandafter\def\csname PY@tok@kt\endcsname{\def\PY@tc##1{\textcolor[rgb]{0.69,0.00,0.25}{##1}}}
\expandafter\def\csname PY@tok@sc\endcsname{\def\PY@tc##1{\textcolor[rgb]{0.73,0.13,0.13}{##1}}}
\expandafter\def\csname PY@tok@sb\endcsname{\def\PY@tc##1{\textcolor[rgb]{0.73,0.13,0.13}{##1}}}
\expandafter\def\csname PY@tok@sa\endcsname{\def\PY@tc##1{\textcolor[rgb]{0.73,0.13,0.13}{##1}}}
\expandafter\def\csname PY@tok@k\endcsname{\let\PY@bf=\textbf\def\PY@tc##1{\textcolor[rgb]{0.00,0.50,0.00}{##1}}}
\expandafter\def\csname PY@tok@se\endcsname{\let\PY@bf=\textbf\def\PY@tc##1{\textcolor[rgb]{0.73,0.40,0.13}{##1}}}
\expandafter\def\csname PY@tok@sd\endcsname{\let\PY@it=\textit\def\PY@tc##1{\textcolor[rgb]{0.73,0.13,0.13}{##1}}}

\def\PYZbs{\char`\\}
\def\PYZus{\char`\_}
\def\PYZob{\char`\{}
\def\PYZcb{\char`\}}
\def\PYZca{\char`\^}
\def\PYZam{\char`\&}
\def\PYZlt{\char`\<}
\def\PYZgt{\char`\>}
\def\PYZsh{\char`\#}
\def\PYZpc{\char`\%}
\def\PYZdl{\char`\$}
\def\PYZhy{\char`\-}
\def\PYZsq{\char`\'}
\def\PYZdq{\char`\"}
\def\PYZti{\char`\~}
% for compatibility with earlier versions
\def\PYZat{@}
\def\PYZlb{[}
\def\PYZrb{]}
\makeatother

	%Reconfigured pygments
	\makeatletter
	\expandafter\def\csname PY@tok@mi\endcsname{\def\PY@tc##1{\textcolor[HTML]{008800}{##1}}} %numbers
	\expandafter\def\csname PY@tok@mf\endcsname{\def\PY@tc##1{\textcolor[HTML]{008800}{##1}}} %numbers
	\expandafter\def\csname PY@tok@nn\endcsname{\def\PY@tc##1{\textcolor[HTML]{000000}{##1}}} %imports
	\expandafter\def\csname PY@tok@ow\endcsname{\let\PY@bf=\textbf\def\PY@tc##1{\textcolor[HTML]{008000}{##1}}} %in
	\expandafter\def\csname PY@tok@o\endcsname{\def\PY@tc##1{\textcolor[rgb]{0.40,0.40,0.40}{\codetrue##1\codefalse}}} %operators
	\makeatother

	\newif\ifcode
	\codefalse
	\definecolor{purp}{HTML}{AA22FF}
	%If using XeLaTeX, use magic to highlight operators in purple.
	\ifdefined\XeTeXcharclass
	\XeTeXinterchartokenstate = 1
	\newXeTeXintercharclass \mycharclassPurple
	\XeTeXcharclass `\* \mycharclassPurple
	\XeTeXcharclass `\/ \mycharclassPurple
	\XeTeXcharclass `\+ \mycharclassPurple
	\XeTeXcharclass `\- \mycharclassPurple
	\XeTeXcharclass `\@ \mycharclassPurple
	\XeTeXcharclass `\% \mycharclassPurple
	\XeTeXcharclass `\& \mycharclassPurple
	\XeTeXcharclass `\| \mycharclassPurple
	\XeTeXcharclass `\! \mycharclassPurple
	\XeTeXcharclass `\< \mycharclassPurple
	\XeTeXcharclass `\> \mycharclassPurple
	\XeTeXcharclass `\~ \mycharclassPurple
	\XeTeXcharclass `\^ \mycharclassPurple
	\XeTeXcharclass `\? \mycharclassPurple

	\XeTeXinterchartoks 0 \mycharclassPurple   = {\bgroup\ifcode\color{purp}\else\fi}
	\XeTeXinterchartoks 4095 \mycharclassPurple = {\bgroup\ifcode\color{purp}\else\fi}

	\XeTeXinterchartoks \mycharclassPurple 0   = {\egroup}
	\XeTeXinterchartoks \mycharclassPurple 4095 = {\egroup}
	\fi %end magical operator highlighting
	%End Reconfigured Pygments
	
   
    % Exact colors from NB
    \definecolor{incolor}{HTML}{303F9F}
    \definecolor{outcolor}{HTML}{D84315}
    \definecolor{cellborder}{HTML}{CFCFCF}
    \definecolor{cellbackground}{HTML}{F7F7F7}

    % needed definitions
    \newif\ifleftmargins
    \newlength{\promptlength}

    % cell style settings
        \leftmarginstrue

    
	% prompt
    \newcommand{\prompt}[3]{
        \needspace{1.1cm}
        \settowidth{\promptlength}{ #1 [#3] }
        \ifleftmargins\hspace{-\promptlength}\hspace{-5pt}\fi
        {\color{#2}#1 [#3]:}
        \ifleftmargins\vspace{-2.7ex}\fi
    }
	
	
    % environments
    \newenvironment{InVerbatim}{\VerbatimEnvironment%
		\begin{tcolorbox}[breakable, size=fbox, boxrule=1pt, pad at break*=1mm,
			colback=cellbackground, colframe=cellborder]
			\begin{Verbatim}
			}{
			\end{Verbatim}
		\end{tcolorbox}
	}
	\newenvironment{OutVerbatim}{\VerbatimEnvironment%
		\begin{tcolorbox}[breakable, boxrule=.5pt, size=fbox, pad at break*=1mm, opacityfill=0]
			\begin{Verbatim}
			}{
			\end{Verbatim}
		\end{tcolorbox}
	}
	
    %Updated MathJax Compatibility (if future behaviour of the notebook changes this may be removed)
    \renewcommand{\TeX}{\ifmmode \textrm{\Oldtex} \else \textbackslash TeX \fi}
    \renewcommand{\LaTeX}{\ifmmode \Oldlatex \else \textbackslash LaTeX \fi}
    
    % Header Adjustments
    \renewcommand{\paragraph}{\textbf}
    \renewcommand{\subparagraph}[1]{\textit{\textbf{#1}}}

    
    % Prevent overflowing lines due to hard-to-break entities
    \sloppy 
    % Setup hyperref package
    \hypersetup{
      breaklinks=true,  % so long urls are correctly broken across lines
      colorlinks=true,
      urlcolor=urlcolor,
      linkcolor=linkcolor,
      citecolor=citecolor,
      }
    % Slightly bigger margins than the latex defaults
    
    \geometry{margin=1.25in}
    
    
\title{Numerical Analysis HW 4: Interpolation\footnote{for MATH-428 Spring 2018 (EO version)}}
\author{Colton Grainger}
\date{2018-07-29}
    \begin{document}
    \maketitle
    \setcounter{tocdepth}{3}
    \tableofcontents
    
    
    
    

    
    \hypertarget{iterative-methods}{%
\section{Iterative methods}\label{iterative-methods}}

\hypertarget{linear-systems}{%
\subsection{Linear Systems}\label{linear-systems}}

To solve the linear system given \(A\mathbf{x} = \mathbf{b}\), we'll
implement three iterative methods: Jacobi, Gauss-Seidel, and Successive
Over Relaxation (SOR).

    
\prompt{In}{incolor}{1}
\begin{InVerbatim}[fontsize=\small,commandchars=\\\{\}]
\PY{k+kn}{import} \PY{n+nn}{numpy} \PY{k}{as} \PY{n+nn}{np}

\PY{n}{A} \PY{o}{=} \PY{n}{np}\PY{o}{.}\PY{n}{array}\PY{p}{(}\PY{p}{[}\PY{p}{[} \PY{l+m+mi}{4}\PY{p}{,} \PY{o}{\PYZhy{}}\PY{l+m+mi}{1}\PY{p}{,}  \PY{l+m+mi}{0}\PY{p}{]}\PY{p}{,} 
              \PY{p}{[}\PY{o}{\PYZhy{}}\PY{l+m+mi}{1}\PY{p}{,}  \PY{l+m+mi}{4}\PY{p}{,} \PY{o}{\PYZhy{}}\PY{l+m+mi}{1}\PY{p}{]}\PY{p}{,} 
              \PY{p}{[} \PY{l+m+mi}{0}\PY{p}{,} \PY{o}{\PYZhy{}}\PY{l+m+mi}{1}\PY{p}{,}  \PY{l+m+mi}{4}\PY{p}{]}\PY{p}{]}\PY{p}{)}
\PY{n}{b} \PY{o}{=} \PY{n}{np}\PY{o}{.}\PY{n}{array}\PY{p}{(}\PY{p}{[} \PY{l+m+mi}{2}\PY{p}{,}  \PY{l+m+mi}{4}\PY{p}{,} \PY{l+m+mi}{10}\PY{p}{]}\PY{p}{)}
\PY{n}{x0} \PY{o}{=} \PY{n}{np}\PY{o}{.}\PY{n}{zeros}\PY{p}{(}\PY{l+m+mi}{3}\PY{p}{)}
\PY{n}{eps} \PY{o}{=} \PY{l+m+mi}{10}\PY{o}{*}\PY{o}{*}\PY{o}{\PYZhy{}}\PY{l+m+mi}{7}
\end{InVerbatim}

    We will:

\begin{itemize}
\tightlist
\item
  Examine whether the sequence of vectors generated by each method's
  fixed point iteration function does converge to the analytic solution
  \(\mathbf{x} = (1,2,3)^T\).
\item
  Take \(\mathbf{x}_0 = (0,0,0)^T\) as our initial term and proceed
  until the \(l_\infty\) norm of the difference between two successive
  approximations is less than \(\varepsilon = 10^{-7}\).
\item
  Time the convergence of each method with python's \texttt{\%\%timeit}
  magic function.
\end{itemize}

\hypertarget{jacobi-method}{%
\subsubsection{Jacobi method}\label{jacobi-method}}

This method simultaneously computes all entries in the output vector
with matrix multiplication.

The function \texttt{jacobi} takes as input

\begin{itemize}
\tightlist
\item
  \texttt{A}, a 2-D square array
\item
  \texttt{b} and \texttt{x0}, 1-D arrays
\end{itemize}

and outputs (the tuple)

\begin{itemize}
\tightlist
\item
  \texttt{x1}, a 1-D array that succeeds \texttt{x0} in approximating
  \texttt{b}
\item
  \texttt{np.linalg.norm(x1\ -\ x0,\ ord=np.inf)}, the \(l_\infty\) norm
  of \texttt{x1-x0}
\end{itemize}

    
\prompt{In}{incolor}{2}
\begin{InVerbatim}[fontsize=\small,commandchars=\\\{\}]
\PY{k}{def} \PY{n+nf}{jacobi}\PY{p}{(}\PY{n}{A}\PY{p}{,}\PY{n}{b}\PY{p}{,}\PY{n}{x0}\PY{p}{,} \PY{o}{*}\PY{n}{args}\PY{p}{)}\PY{p}{:}
    \PY{n}{U} \PY{o}{=} \PY{n}{np}\PY{o}{.}\PY{n}{triu}\PY{p}{(}\PY{n}{A}\PY{p}{,}\PY{l+m+mi}{1}\PY{p}{)}
    \PY{n}{D} \PY{o}{=} \PY{n}{np}\PY{o}{.}\PY{n}{diag}\PY{p}{(}\PY{n}{A}\PY{p}{)}
    \PY{n}{L} \PY{o}{=} \PY{n}{np}\PY{o}{.}\PY{n}{tril}\PY{p}{(}\PY{n}{A}\PY{p}{,}\PY{o}{\PYZhy{}}\PY{l+m+mi}{1}\PY{p}{)}
    \PY{n}{x1} \PY{o}{=} \PY{p}{(}\PY{l+m+mi}{1}\PY{o}{/}\PY{n}{D}\PY{p}{)} \PY{o}{*} \PY{p}{(}\PY{n}{b} \PY{o}{\PYZhy{}} \PY{n}{np}\PY{o}{.}\PY{n}{matmul}\PY{p}{(}\PY{p}{(}\PY{n}{L}\PY{o}{+}\PY{n}{U}\PY{p}{)}\PY{p}{,}\PY{n}{x0}\PY{p}{)}\PY{p}{)}
    \PY{k}{return} \PY{n}{x1}\PY{p}{,} \PY{n}{np}\PY{o}{.}\PY{n}{linalg}\PY{o}{.}\PY{n}{norm}\PY{p}{(}\PY{n}{x1} \PY{o}{\PYZhy{}} \PY{n}{x0}\PY{p}{,} \PY{n+nb}{ord}\PY{o}{=}\PY{n}{np}\PY{o}{.}\PY{n}{inf}\PY{p}{)}
\end{InVerbatim}

    We now iterate the state of \texttt{x\_approx} by looping calls to the
\texttt{jacobi} function while \texttt{interstep} is greater than
\(\varepsilon\).

    
\prompt{In}{incolor}{3}
\begin{InVerbatim}[fontsize=\small,commandchars=\\\{\}]
\PY{n}{U} \PY{o}{=} \PY{n}{np}\PY{o}{.}\PY{n}{triu}\PY{p}{(}\PY{n}{A}\PY{p}{,}\PY{l+m+mi}{1}\PY{p}{)}
\PY{n}{D} \PY{o}{=} \PY{n}{np}\PY{o}{.}\PY{n}{diag}\PY{p}{(}\PY{n}{A}\PY{p}{)}
\PY{n}{L} \PY{o}{=} \PY{n}{np}\PY{o}{.}\PY{n}{tril}\PY{p}{(}\PY{n}{A}\PY{p}{,}\PY{o}{\PYZhy{}}\PY{l+m+mi}{1}\PY{p}{)}
\PY{n}{x1} \PY{o}{=} \PY{p}{(}\PY{l+m+mi}{1}\PY{o}{/}\PY{n}{D}\PY{p}{)} \PY{o}{*} \PY{p}{(}\PY{n}{b} \PY{o}{\PYZhy{}} \PY{n}{np}\PY{o}{.}\PY{n}{matmul}\PY{p}{(}\PY{p}{(}\PY{n}{L}\PY{o}{+}\PY{n}{U}\PY{p}{)}\PY{p}{,}\PY{n}{x0}\PY{p}{)}\PY{p}{)}
\end{InVerbatim}

    
\prompt{In}{incolor}{4}
\begin{InVerbatim}[fontsize=\small,commandchars=\\\{\}]
\PY{n}{x\PYZus{}approx} \PY{o}{=} \PY{n}{x0}
\PY{n}{interstep} \PY{o}{=} \PY{n}{eps}\PY{o}{+}\PY{l+m+mi}{1} \PY{c+c1}{\PYZsh{}must dominate eps to enter while loop}
\PY{k}{while} \PY{n}{interstep} \PY{o}{\PYZgt{}} \PY{n}{eps}\PY{p}{:}
    \PY{n}{x\PYZus{}approx}\PY{p}{,}\PY{n}{interstep} \PY{o}{=} \PY{n}{jacobi}\PY{p}{(}\PY{n}{A}\PY{p}{,}\PY{n}{b}\PY{p}{,}\PY{n}{x\PYZus{}approx}\PY{p}{)}
\PY{n}{x\PYZus{}approx}
\end{InVerbatim}

            
\prompt{Out}{outcolor}{4}
\begin{OutVerbatim}[fontsize=\small,commandchars=\\\{\}]
array([0.99999997, 1.99999994, 2.99999997])
\end{OutVerbatim}
        
    Lastly, we time the execution of the entire cell, and take the average
of the 3 best results.

    
\prompt{In}{incolor}{5}
\begin{InVerbatim}[fontsize=\small,commandchars=\\\{\}]
\PY{o}{\PYZpc{}\PYZpc{}}\PY{k}{timeit}
x\PYZus{}approx = x0
interstep = eps+1
while interstep \PYZgt{} eps:
    x\PYZus{}approx,interstep = jacobi(A,b,x\PYZus{}approx)
\end{InVerbatim}

    \begin{Verbatim}[fontsize=\small,commandchars=\\\{\}]
1.14 ms ± 6.58 µs per loop (mean ± std. dev. of 7 runs, 1000 loops each)

    \end{Verbatim}

    \hypertarget{gauss-seidel-method}{%
\subsubsection{Gauss-Seidel method}\label{gauss-seidel-method}}

In contrast to the Jacobi method, the Gauss-Seidel method computes each
entry \texttt{x1{[}i{]}} of the output vector in succession.

    
\prompt{In}{incolor}{6}
\begin{InVerbatim}[fontsize=\small,commandchars=\\\{\}]
\PY{k}{def} \PY{n+nf}{gauss\PYZus{}seidel}\PY{p}{(}\PY{n}{A}\PY{p}{,}\PY{n}{b}\PY{p}{,}\PY{n}{x0}\PY{p}{,} \PY{o}{*}\PY{n}{args}\PY{p}{)}\PY{p}{:}
    \PY{n}{n} \PY{o}{=} \PY{n+nb}{len}\PY{p}{(}\PY{n}{x0}\PY{p}{)}
    \PY{n}{x1}\PY{o}{=}\PY{n}{np}\PY{o}{.}\PY{n}{zeros}\PY{p}{(}\PY{n}{n}\PY{p}{)}
    \PY{k}{for} \PY{n}{i} \PY{o+ow}{in} \PY{n+nb}{range}\PY{p}{(}\PY{n}{n}\PY{p}{)}\PY{p}{:}
        \PY{n}{new} \PY{o}{=} \PY{n}{np}\PY{o}{.}\PY{n}{sum}\PY{p}{(}\PY{n}{A}\PY{p}{[}\PY{n}{i}\PY{p}{,}\PY{p}{:}\PY{n}{i}\PY{p}{]} \PY{o}{*} \PY{n}{x1}\PY{p}{[}\PY{p}{:}\PY{n}{i}\PY{p}{]}\PY{p}{)}
        \PY{n}{old} \PY{o}{=} \PY{n}{np}\PY{o}{.}\PY{n}{sum}\PY{p}{(}\PY{n}{A}\PY{p}{[}\PY{n}{i}\PY{p}{,}\PY{n}{i}\PY{o}{+}\PY{l+m+mi}{1}\PY{p}{:}\PY{p}{]} \PY{o}{*} \PY{n}{x0}\PY{p}{[}\PY{n}{i}\PY{o}{+}\PY{l+m+mi}{1}\PY{p}{:}\PY{p}{]}\PY{p}{)}
        \PY{n}{x1}\PY{p}{[}\PY{n}{i}\PY{p}{]} \PY{o}{=} \PY{p}{(}\PY{l+m+mi}{1} \PY{o}{/} \PY{n}{A}\PY{p}{[}\PY{n}{i}\PY{p}{,}\PY{n}{i}\PY{p}{]}\PY{p}{)} \PY{o}{*} \PY{p}{(}\PY{n}{b}\PY{p}{[}\PY{n}{i}\PY{p}{]} \PY{o}{\PYZhy{}} \PY{n}{new} \PY{o}{\PYZhy{}} \PY{n}{old}\PY{p}{)}
    \PY{k}{return} \PY{n}{x1}\PY{p}{,} \PY{n}{np}\PY{o}{.}\PY{n}{linalg}\PY{o}{.}\PY{n}{norm}\PY{p}{(}\PY{n}{x1} \PY{o}{\PYZhy{}} \PY{n}{x0}\PY{p}{,} \PY{n+nb}{ord}\PY{o}{=}\PY{n}{np}\PY{o}{.}\PY{n}{inf}\PY{p}{)}
\end{InVerbatim}

    We update \texttt{x\_approx} in a similar a while loop.

    
\prompt{In}{incolor}{7}
\begin{InVerbatim}[fontsize=\small,commandchars=\\\{\}]
\PY{n}{x\PYZus{}approx} \PY{o}{=} \PY{n}{x0}
\PY{n}{interstep} \PY{o}{=} \PY{n}{eps}\PY{o}{+}\PY{l+m+mi}{1}
\PY{k}{while} \PY{n}{interstep} \PY{o}{\PYZgt{}} \PY{n}{eps}\PY{p}{:}
    \PY{n}{x\PYZus{}approx}\PY{p}{,}\PY{n}{interstep} \PY{o}{=} \PY{n}{gauss\PYZus{}seidel}\PY{p}{(}\PY{n}{A}\PY{p}{,}\PY{n}{b}\PY{p}{,}\PY{n}{x\PYZus{}approx}\PY{p}{)}
\PY{n}{x\PYZus{}approx}
\end{InVerbatim}

            
\prompt{Out}{outcolor}{7}
\begin{OutVerbatim}[fontsize=\small,commandchars=\\\{\}]
array([0.99999999, 1.99999999, 3.        ])
\end{OutVerbatim}
        
    And we'll time it.

    
\prompt{In}{incolor}{8}
\begin{InVerbatim}[fontsize=\small,commandchars=\\\{\}]
\PY{o}{\PYZpc{}\PYZpc{}}\PY{k}{timeit}
x\PYZus{}approx = x0
interstep = eps+1
while interstep \PYZgt{} eps:
    x\PYZus{}approx,interstep = gauss\PYZus{}seidel(A,b,x\PYZus{}approx)
\end{InVerbatim}

    \begin{Verbatim}[fontsize=\small,commandchars=\\\{\}]
975 µs ± 82.2 µs per loop (mean ± std. dev. of 7 runs, 1000 loops each)

    \end{Verbatim}

    \hypertarget{successive-over-relaxation}{%
\subsubsection{Successive over
relaxation}\label{successive-over-relaxation}}

Successive over relaxation generalizes the Gauss-Seidel method, and as
such the \texttt{SOR} function requires an additional input
\texttt{omega}.

We choose the optimal parameter \(\omega = 2/(1+\sqrt{7/8})\).

    
\prompt{In}{incolor}{9}
\begin{InVerbatim}[fontsize=\small,commandchars=\\\{\}]
\PY{k}{def} \PY{n+nf}{SOR}\PY{p}{(}\PY{n}{A}\PY{p}{,}\PY{n}{b}\PY{p}{,}\PY{n}{x0}\PY{p}{,}\PY{n}{omega}\PY{p}{)}\PY{p}{:}
    \PY{n}{n} \PY{o}{=} \PY{n+nb}{len}\PY{p}{(}\PY{n}{x0}\PY{p}{)}
    \PY{n}{x1}\PY{o}{=}\PY{n}{np}\PY{o}{.}\PY{n}{zeros}\PY{p}{(}\PY{n}{n}\PY{p}{)}
    \PY{k}{for} \PY{n}{i} \PY{o+ow}{in} \PY{n+nb}{range}\PY{p}{(}\PY{n}{n}\PY{p}{)}\PY{p}{:}
        \PY{n}{jaterm} \PY{o}{=} \PY{p}{(}\PY{l+m+mi}{1}\PY{o}{\PYZhy{}}\PY{n}{omega}\PY{p}{)} \PY{o}{*} \PY{n}{x0}\PY{p}{[}\PY{n}{i}\PY{p}{]}
        \PY{n}{gsnew} \PY{o}{=} \PY{n}{np}\PY{o}{.}\PY{n}{sum}\PY{p}{(}\PY{n}{A}\PY{p}{[}\PY{n}{i}\PY{p}{,}\PY{p}{:}\PY{n}{i}\PY{p}{]} \PY{o}{*} \PY{n}{x1}\PY{p}{[}\PY{p}{:}\PY{n}{i}\PY{p}{]}\PY{p}{)}
        \PY{n}{gsold} \PY{o}{=} \PY{n}{np}\PY{o}{.}\PY{n}{sum}\PY{p}{(}\PY{n}{A}\PY{p}{[}\PY{n}{i}\PY{p}{,}\PY{n}{i}\PY{o}{+}\PY{l+m+mi}{1}\PY{p}{:}\PY{p}{]} \PY{o}{*} \PY{n}{x0}\PY{p}{[}\PY{n}{i}\PY{o}{+}\PY{l+m+mi}{1}\PY{p}{:}\PY{p}{]}\PY{p}{)}
        \PY{n}{gsterm} \PY{o}{=} \PY{p}{(}\PY{n}{omega} \PY{o}{/} \PY{n}{A}\PY{p}{[}\PY{n}{i}\PY{p}{,}\PY{n}{i}\PY{p}{]}\PY{p}{)} \PY{o}{*} \PY{p}{(}\PY{n}{b}\PY{p}{[}\PY{n}{i}\PY{p}{]} \PY{o}{\PYZhy{}} \PY{n}{gsnew} \PY{o}{\PYZhy{}} \PY{n}{gsold}\PY{p}{)}
        \PY{n}{x1}\PY{p}{[}\PY{n}{i}\PY{p}{]} \PY{o}{=} \PY{n}{jaterm} \PY{o}{+} \PY{n}{gsterm}
    \PY{k}{return} \PY{n}{x1}\PY{p}{,} \PY{n}{np}\PY{o}{.}\PY{n}{linalg}\PY{o}{.}\PY{n}{norm}\PY{p}{(}\PY{n}{x1}\PY{o}{\PYZhy{}}\PY{n}{x0}\PY{p}{,} \PY{n+nb}{ord}\PY{o}{=}\PY{n}{np}\PY{o}{.}\PY{n}{inf}\PY{p}{)}
\end{InVerbatim}

    
\prompt{In}{incolor}{10}
\begin{InVerbatim}[fontsize=\small,commandchars=\\\{\}]
\PY{n}{omega} \PY{o}{=} \PY{l+m+mi}{2}\PY{o}{/}\PY{p}{(}\PY{l+m+mi}{1}\PY{o}{+}\PY{n}{np}\PY{o}{.}\PY{n}{sqrt}\PY{p}{(}\PY{l+m+mi}{7}\PY{o}{/}\PY{l+m+mi}{8}\PY{p}{)}\PY{p}{)}
\PY{n}{x\PYZus{}approx} \PY{o}{=} \PY{n}{x0}
\PY{n}{interstep} \PY{o}{=} \PY{n}{eps}\PY{o}{+}\PY{l+m+mi}{1}
\PY{k}{while} \PY{n}{interstep} \PY{o}{\PYZgt{}} \PY{n}{eps}\PY{p}{:}
    \PY{n}{x\PYZus{}approx}\PY{p}{,}\PY{n}{interstep} \PY{o}{=} \PY{n}{SOR}\PY{p}{(}\PY{n}{A}\PY{p}{,}\PY{n}{b}\PY{p}{,}\PY{n}{x\PYZus{}approx}\PY{p}{,}\PY{n}{omega}\PY{p}{)}
\PY{n}{x\PYZus{}approx}
\end{InVerbatim}

            
\prompt{Out}{outcolor}{10}
\begin{OutVerbatim}[fontsize=\small,commandchars=\\\{\}]
array([1., 2., 3.])
\end{OutVerbatim}
        
    
\prompt{In}{incolor}{11}
\begin{InVerbatim}[fontsize=\small,commandchars=\\\{\}]
\PY{o}{\PYZpc{}\PYZpc{}}\PY{k}{timeit}
x\PYZus{}approx = x0
interstep = eps+1
while interstep \PYZgt{} eps:
    x\PYZus{}approx,interstep = SOR(A,b,x\PYZus{}approx,omega)
\end{InVerbatim}

    \begin{Verbatim}[fontsize=\small,commandchars=\\\{\}]
777 µs ± 46.8 µs per loop (mean ± std. dev. of 7 runs, 1000 loops each)

    \end{Verbatim}

    \hypertarget{error-analysis}{%
\subsubsection{Error analysis}\label{error-analysis}}

The function \texttt{capture\_error} logs the \(l_\infty\) norm of the
error as each \texttt{method} runs until the desired accuracy is
achieved.

    
\prompt{In}{incolor}{12}
\begin{InVerbatim}[fontsize=\small,commandchars=\\\{\}]
\PY{k}{def} \PY{n+nf}{capture\PYZus{}error}\PY{p}{(}\PY{n}{method}\PY{p}{,}\PY{n}{A}\PY{p}{,}\PY{n}{b}\PY{p}{,}\PY{n}{x\PYZus{}approx}\PY{p}{,}\PY{n}{omega}\PY{p}{,}\PY{n}{x\PYZus{}exact}\PY{p}{,}\PY{n}{eps}\PY{p}{)}\PY{p}{:}
    \PY{n}{error} \PY{o}{=} \PY{n}{np}\PY{o}{.}\PY{n}{array}\PY{p}{(}\PY{p}{[}\PY{n}{np}\PY{o}{.}\PY{n}{linalg}\PY{o}{.}\PY{n}{norm}\PY{p}{(}\PY{n}{x\PYZus{}approx} \PY{o}{\PYZhy{}} \PY{n}{x\PYZus{}exact}\PY{p}{,} \PY{n+nb}{ord}\PY{o}{=}\PY{n}{np}\PY{o}{.}\PY{n}{inf}\PY{p}{)}\PY{p}{]}\PY{p}{)}
    \PY{k}{while} \PY{n}{error}\PY{p}{[}\PY{o}{\PYZhy{}}\PY{l+m+mi}{1}\PY{p}{]} \PY{o}{\PYZgt{}} \PY{n}{eps}\PY{p}{:}
        \PY{n}{x\PYZus{}approx} \PY{o}{=} \PY{n}{method}\PY{p}{(}\PY{n}{A}\PY{p}{,}\PY{n}{b}\PY{p}{,}\PY{n}{x\PYZus{}approx}\PY{p}{,}\PY{n}{omega}\PY{p}{)}\PY{p}{[}\PY{l+m+mi}{0}\PY{p}{]}
        \PY{n}{error} \PY{o}{=} \PY{n}{np}\PY{o}{.}\PY{n}{append}\PY{p}{(}\PY{n}{error}\PY{p}{,}\PYZbs{}
                \PY{n}{np}\PY{o}{.}\PY{n}{linalg}\PY{o}{.}\PY{n}{norm}\PY{p}{(}\PY{n}{x\PYZus{}approx} \PY{o}{\PYZhy{}} \PY{n}{x\PYZus{}exact}\PY{p}{,} \PY{n+nb}{ord}\PY{o}{=}\PY{n}{np}\PY{o}{.}\PY{n}{inf}\PY{p}{)}\PY{p}{)}
    \PY{k}{return} \PY{n}{error}
\end{InVerbatim}

    We plot each method's error by iteration, and summarize.

    
\prompt{In}{incolor}{13}
\begin{InVerbatim}[fontsize=\small,commandchars=\\\{\}]
\PY{k+kn}{import} \PY{n+nn}{matplotlib}\PY{n+nn}{.}\PY{n+nn}{pyplot} \PY{k}{as} \PY{n+nn}{plt}
\PY{k+kn}{from} \PY{n+nn}{matplotlib}\PY{n+nn}{.}\PY{n+nn}{ticker} \PY{k}{import} \PY{n}{MaxNLocator}
\PY{n}{ax} \PY{o}{=} \PY{n}{plt}\PY{o}{.}\PY{n}{figure}\PY{p}{(}\PY{p}{)}\PY{o}{.}\PY{n}{gca}\PY{p}{(}\PY{p}{)}

\PY{n}{plt}\PY{o}{.}\PY{n}{title}\PY{p}{(}\PY{l+s+s1}{\PYZsq{}}\PY{l+s+s1}{Prob 1(b): \PYZdl{}l\PYZus{}}\PY{l+s+s1}{\PYZbs{}}\PY{l+s+s1}{infty\PYZdl{} norm of error by iteration}\PY{l+s+s1}{\PYZsq{}}\PY{p}{)}
\PY{n}{x\PYZus{}exact} \PY{o}{=} \PY{n}{np}\PY{o}{.}\PY{n}{array}\PY{p}{(}\PY{p}{[}\PY{l+m+mi}{1}\PY{p}{,}\PY{l+m+mi}{2}\PY{p}{,}\PY{l+m+mi}{3}\PY{p}{]}\PY{p}{)}
\PY{k}{for} \PY{n}{method} \PY{o+ow}{in} \PY{p}{[}\PY{n}{jacobi}\PY{p}{,}\PY{n}{gauss\PYZus{}seidel}\PY{p}{,}\PY{n}{SOR}\PY{p}{]}\PY{p}{:}
    \PY{n}{ax}\PY{o}{.}\PY{n}{semilogy}\PY{p}{(}\PY{n}{capture\PYZus{}error}\PY{p}{(}\PY{n}{method}\PY{p}{,}\PY{n}{A}\PY{p}{,}\PY{n}{b}\PY{p}{,}\PY{n}{x0}\PY{p}{,}\PY{n}{omega}\PY{p}{,}\PY{n}{x\PYZus{}exact}\PY{p}{,}\PY{n}{eps}\PY{p}{)}\PY{p}{,}\PYZbs{}
        \PY{n}{label}\PY{o}{=}\PY{n}{method}\PY{o}{.}\PY{n+nv+vm}{\PYZus{}\PYZus{}name\PYZus{}\PYZus{}}\PY{p}{)}
    
\PY{n}{plt}\PY{o}{.}\PY{n}{legend}\PY{p}{(}\PY{p}{)}
\PY{n}{ax}\PY{o}{.}\PY{n}{xaxis}\PY{o}{.}\PY{n}{set\PYZus{}major\PYZus{}locator}\PY{p}{(}\PY{n}{MaxNLocator}\PY{p}{(}\PY{n}{integer}\PY{o}{=}\PY{k+kc}{True}\PY{p}{)}\PY{p}{)}
\PY{n}{plt}\PY{o}{.}\PY{n}{show}\PY{p}{(}\PY{p}{)}
\end{InVerbatim}

    
    \begin{verbatim}
<Figure size 640x480 with 1 Axes>
    \end{verbatim}

    
    To estimate the asymptotic error constant.

    
\prompt{In}{incolor}{14}
\begin{InVerbatim}[fontsize=\small,commandchars=\\\{\}]
\PY{k}{def} \PY{n+nf}{est\PYZus{}asym\PYZus{}error}\PY{p}{(}\PY{n}{method}\PY{p}{,}\PY{n}{A}\PY{p}{,}\PY{n}{b}\PY{p}{,}\PY{n}{x0}\PY{p}{,}\PY{n}{omega}\PY{p}{,}\PY{n}{eps}\PY{p}{,}\PY{n}{k}\PY{p}{)}\PY{p}{:}
    \PY{n}{interstep} \PY{o}{=} \PY{n}{np}\PY{o}{.}\PY{n}{array}\PY{p}{(}\PY{p}{[}\PY{n}{eps}\PY{o}{+}\PY{l+m+mi}{1}\PY{p}{]}\PY{p}{)}
    \PY{k}{while} \PY{n}{interstep}\PY{p}{[}\PY{o}{\PYZhy{}}\PY{l+m+mi}{1}\PY{p}{]} \PY{o}{\PYZgt{}} \PY{n}{eps}\PY{p}{:}
        \PY{n}{x1} \PY{o}{=} \PY{n}{method}\PY{p}{(}\PY{n}{A}\PY{p}{,}\PY{n}{b}\PY{p}{,}\PY{n}{x0}\PY{p}{,}\PY{n}{omega}\PY{p}{)}\PY{p}{[}\PY{l+m+mi}{0}\PY{p}{]}
        \PY{n}{interstep} \PY{o}{=} \PY{n}{np}\PY{o}{.}\PY{n}{append}\PY{p}{(}\PY{n}{interstep}\PY{p}{,}\PYZbs{}
                \PY{n}{np}\PY{o}{.}\PY{n}{linalg}\PY{o}{.}\PY{n}{norm}\PY{p}{(}\PY{n}{x0} \PY{o}{\PYZhy{}} \PY{n}{x1}\PY{p}{,} \PY{n+nb}{ord}\PY{o}{=}\PY{n}{np}\PY{o}{.}\PY{n}{inf}\PY{p}{)}\PY{p}{)}
        \PY{n}{x0} \PY{o}{=} \PY{n}{x1}
    \PY{n}{C} \PY{o}{=} \PY{n}{interstep}\PY{p}{[}\PY{n}{k}\PY{o}{+}\PY{l+m+mi}{1}\PY{p}{]}\PY{o}{/}\PY{n}{interstep}\PY{p}{[}\PY{n}{k}\PY{p}{]}
    \PY{k}{return} \PY{n}{C}
\end{InVerbatim}

    \begin{longtable}[]{@{}llll@{}}
\toprule
method & iterations & runtime & error const.\tabularnewline
\midrule
\endhead
jacobi & 17 & \textasciitilde{}1100 µs & 0.25\tabularnewline
gauss\_seidel & 10 & \textasciitilde{}900 µs & 0.125\tabularnewline
SOR & 7 & \textasciitilde{}750 µs & 0.0489\tabularnewline
\bottomrule
\end{longtable}

\hypertarget{nonlinear-systems}{%
\subsection{Nonlinear systems}\label{nonlinear-systems}}

Suppose we desire the points in the intersection of the circle
\(x^2 + y^2 = 4\) and the hyperbola \(x^2 - y^2 = 1\). In this case (and
other cases where the polynomials in multiple variables are well
behaved) the solutions are available in a closed form (that is,
attainable from a finite number of arithmetic operations). For other
cases, iterative methods allow us to work towards a solution with the
first order Taylor expansion of the vector function
\(F:\mathbf{R}^n \to\mathbf{R}^n\).

\hypertarget{newtons-method}{%
\subsubsection{Newton's method}\label{newtons-method}}

Here we demonstrate one iteration of Newton's method for the above
system of non-linear equations (whose analytic solutions are the four
points \(\left\{(x,y) : x=1,y=\pm\sqrt{3}\right\}\)), given the initial
vector \((1, 1)^T\).

    
\prompt{In}{incolor}{15}
\begin{InVerbatim}[fontsize=\small,commandchars=\\\{\}]
\PY{k+kn}{import} \PY{n+nn}{sympy} \PY{k}{as} \PY{n+nn}{sym}

\PY{n}{x}\PY{p}{,}\PY{n}{y} \PY{o}{=} \PY{n}{sym}\PY{o}{.}\PY{n}{symbols}\PY{p}{(}\PY{l+s+s1}{\PYZsq{}}\PY{l+s+s1}{x y}\PY{l+s+s1}{\PYZsq{}}\PY{p}{)}
\PY{n}{F} \PY{o}{=} \PY{n}{sym}\PY{o}{.}\PY{n}{Matrix}\PY{p}{(}\PY{p}{[}\PY{n}{x}\PY{o}{*}\PY{o}{*}\PY{l+m+mi}{2} \PY{o}{+} \PY{n}{y}\PY{o}{*}\PY{o}{*}\PY{l+m+mi}{2} \PY{o}{\PYZhy{}} \PY{l+m+mi}{4}\PY{p}{,} \PY{n}{x}\PY{o}{*}\PY{o}{*}\PY{l+m+mi}{2} \PY{o}{\PYZhy{}} \PY{n}{y}\PY{o}{*}\PY{o}{*}\PY{l+m+mi}{2} \PY{o}{\PYZhy{}} \PY{l+m+mi}{1}\PY{p}{]}\PY{p}{)}
\PY{n+nb}{print}\PY{p}{(}\PY{l+s+s2}{\PYZdq{}}\PY{l+s+s2}{F    =}\PY{l+s+s2}{\PYZdq{}}\PY{p}{,} \PY{n}{F}\PY{p}{)}

\PY{n}{J} \PY{o}{=} \PY{n}{F}\PY{o}{.}\PY{n}{jacobian}\PY{p}{(}\PY{n}{sym}\PY{o}{.}\PY{n}{Matrix}\PY{p}{(}\PY{p}{[}\PY{n}{x}\PY{p}{,}\PY{n}{y}\PY{p}{]}\PY{p}{)}\PY{p}{)}
\PY{n+nb}{print}\PY{p}{(}\PY{l+s+s2}{\PYZdq{}}\PY{l+s+s2}{J    =}\PY{l+s+s2}{\PYZdq{}}\PY{p}{,} \PY{n}{J}\PY{p}{)}

\PY{n}{init} \PY{o}{=} \PY{p}{[}\PY{p}{(}\PY{n}{x}\PY{p}{,}\PY{l+m+mi}{1}\PY{p}{)}\PY{p}{,}\PY{p}{(}\PY{n}{y}\PY{p}{,}\PY{l+m+mi}{1}\PY{p}{)}\PY{p}{]}
\PY{n}{vec0} \PY{o}{=} \PY{n}{sym}\PY{o}{.}\PY{n}{Matrix}\PY{p}{(}\PY{p}{[}\PY{n}{coord}\PY{p}{[}\PY{l+m+mi}{1}\PY{p}{]} \PY{k}{for} \PY{n}{coord} \PY{o+ow}{in} \PY{n}{init}\PY{p}{]}\PY{p}{)}
\PY{n+nb}{print}\PY{p}{(}\PY{l+s+s2}{\PYZdq{}}\PY{l+s+s2}{vec0 =}\PY{l+s+s2}{\PYZdq{}}\PY{p}{,} \PY{n}{vec0}\PY{p}{)}

\PY{n}{h} \PY{o}{=} \PY{n}{J}\PY{o}{.}\PY{n}{subs}\PY{p}{(}\PY{n}{init}\PY{p}{)}\PY{o}{.}\PY{n}{LUsolve}\PY{p}{(}\PY{o}{\PYZhy{}}\PY{n}{F}\PY{o}{.}\PY{n}{subs}\PY{p}{(}\PY{n}{init}\PY{p}{)}\PY{p}{)}
\PY{n+nb}{print}\PY{p}{(}\PY{l+s+s2}{\PYZdq{}}\PY{l+s+s2}{h    =}\PY{l+s+s2}{\PYZdq{}}\PY{p}{,} \PY{n}{h}\PY{p}{)}

\PY{n}{vec1} \PY{o}{=} \PY{n}{h} \PY{o}{+} \PY{n}{vec0}
\PY{n+nb}{print}\PY{p}{(}\PY{l+s+s2}{\PYZdq{}}\PY{l+s+s2}{vec1 =}\PY{l+s+s2}{\PYZdq{}}\PY{p}{,} \PY{n}{vec1}\PY{p}{)}
\end{InVerbatim}

    \begin{Verbatim}[fontsize=\small,commandchars=\\\{\}]
F    = Matrix([[x**2 + y**2 - 4], [x**2 - y**2 - 1]])
J    = Matrix([[2*x, 2*y], [2*x, -2*y]])
vec0 = Matrix([[1], [1]])
h    = Matrix([[3/4], [1/4]])
vec1 = Matrix([[7/4], [5/4]])

    \end{Verbatim}

    \hypertarget{interpolation}{%
\section{Interpolation}\label{interpolation}}

\hypertarget{polynomial-approximation}{%
\subsection{Polynomial approximation}\label{polynomial-approximation}}

We'll write two series of functions to generate the interpolating
polynomial as a vector in the space \(\mathcal{P}_n\) of polynomials
with degree at most \(n\). The first will generate a basis of
polynomials in the Lagrange form, the latter will do so in the Newton
form. Both will then interpolate an arbitrary function \(f \in C[a,b]\)
at a finite number of points \(x_i \in [a,b]\).

\hypertarget{lagrange-form}{%
\subsubsection{Lagrange form}\label{lagrange-form}}

Consider \(f \in C[a,b]\) (the set of continuous real-valued functions
on the compact interval \([a,b]\)), and let \(x_0,\ldots,x_n\) be
\(n+1\) distinct points. Does there exist polynomial of least degree
\(p \in \mathbf{R}[x]\) which \emph{interpolates} \(f\) at the given
points? such that \(f(x_i) = p(x_i)\) for all
\(i \in \{0,1,\ldots,n\}\)? Is it unique?

The fundamental theorem of algebra guarantees uniqueness (the difference
between any two least degree interpolating polynomials is the zero
polynomial), and the Lagrange form of the interpolating polynomial
demonstrates existence.

Having \(n+1\) distinct points \(x_i\) specified, the \(n+1\) Lagrange
polynomials \(\ell_k\) are written as the product
\[\ell_k(x) = \prod_{i\neq k} \frac{x-x_i}{x_k-x_i} \text{ for all } k \in \{0,1,\ldots,n\}.\]

That is:

    
\prompt{In}{incolor}{1}
\begin{InVerbatim}[fontsize=\small,commandchars=\\\{\}]
\PY{k+kn}{import} \PY{n+nn}{numpy} \PY{k}{as} \PY{n+nn}{np}
\PY{k+kn}{import} \PY{n+nn}{sympy} \PY{k}{as} \PY{n+nn}{sym}
\PY{k+kn}{import} \PY{n+nn}{matplotlib}\PY{n+nn}{.}\PY{n+nn}{pyplot} \PY{k}{as} \PY{n+nn}{plt}
\PY{k+kn}{import} \PY{n+nn}{math}

\PY{k}{def} \PY{n+nf}{lagrange\PYZus{}basis}\PY{p}{(}\PY{n}{vecx}\PY{p}{)}\PY{p}{:}
    \PY{n}{n} \PY{o}{=} \PY{n+nb}{len}\PY{p}{(}\PY{n}{vecx}\PY{p}{)}
    \PY{c+c1}{\PYZsh{} choose n\PYZhy{}1 of n entries}
    \PY{n}{hinges} \PY{o}{=} \PY{p}{[}\PY{n}{np}\PY{o}{.}\PY{n}{concatenate}\PY{p}{(}\PY{p}{[}\PY{n}{vecx}\PY{p}{[}\PY{p}{:}\PY{n}{i}\PY{p}{]}\PY{p}{,} \PY{n}{vecx}\PY{p}{[}\PY{n}{i}\PY{o}{+}\PY{l+m+mi}{1}\PY{p}{:}\PY{p}{]}\PY{p}{]}\PY{p}{)} \PY{k}{for} \PY{n}{i} \PY{o+ow}{in} \PY{n+nb}{range}\PY{p}{(}\PY{n}{n}\PY{p}{)}\PY{p}{]}
    \PY{c+c1}{\PYZsh{} take product of n\PYZhy{}1 terms, format into lambda functions}
    \PY{n}{polys} \PY{o}{=} \PY{p}{[}\PY{k}{lambda} \PY{n}{t}\PY{p}{,} \PY{n}{i}\PY{o}{=}\PY{n}{i}\PY{p}{:} \PY{n}{np}\PY{o}{.}\PY{n}{prod}\PY{p}{(}\PY{p}{(}\PY{n}{t}\PY{o}{\PYZhy{}}\PY{n}{hinges}\PY{p}{[}\PY{n}{i}\PY{p}{]}\PY{p}{)}\PY{o}{/}\PY{p}{(}\PY{n}{vecx}\PY{p}{[}\PY{n}{i}\PY{p}{]}\PY{o}{\PYZhy{}}\PY{n}{hinges}\PY{p}{[}\PY{n}{i}\PY{p}{]}\PY{p}{)}\PY{p}{)}\PYZbs{}
            \PY{k}{for} \PY{n}{i} \PY{o+ow}{in} \PY{n+nb}{range}\PY{p}{(}\PY{n}{n}\PY{p}{)}\PY{p}{]}
    \PY{c+c1}{\PYZsh{} returns a list of n functions, the Lagrange basis polynomials}
    \PY{k}{return} \PY{n}{polys}
\end{InVerbatim}

    It can be shown that the set of \(n+1\) Lagrange polynomials spans
\(\mathcal{P}_n\). We note the Lagrange polynomials form a basis for
\(\mathcal{P}_n\) since \(\dim(\mathcal{P}_n) = n+1\).

For example:

    
\prompt{In}{incolor}{2}
\begin{InVerbatim}[fontsize=\small,commandchars=\\\{\}]
\PY{n}{vecx} \PY{o}{=} \PY{n}{np}\PY{o}{.}\PY{n}{array}\PY{p}{(}\PY{p}{[}\PY{l+m+mi}{0}\PY{p}{,}\PY{l+m+mi}{1}\PY{p}{,}\PY{l+m+mi}{2}\PY{p}{,}\PY{l+m+mi}{3}\PY{p}{]}\PY{p}{)}

\PY{k}{def} \PY{n+nf}{lagrange\PYZus{}basis\PYZus{}plot}\PY{p}{(}\PY{n}{vecx}\PY{p}{)}\PY{p}{:}
    \PY{n}{polys} \PY{o}{=} \PY{n}{lagrange\PYZus{}basis}\PY{p}{(}\PY{n}{vecx}\PY{p}{)}
    \PY{n}{start}\PY{p}{,} \PY{n}{stop} \PY{o}{=} \PY{n}{np}\PY{o}{.}\PY{n}{min}\PY{p}{(}\PY{n}{vecx}\PY{p}{)}\PY{p}{,} \PY{n}{np}\PY{o}{.}\PY{n}{max}\PY{p}{(}\PY{n}{vecx}\PY{p}{)}
    \PY{n}{stiple} \PY{o}{=} \PY{n}{np}\PY{o}{.}\PY{n}{linspace}\PY{p}{(}\PY{n}{start}\PY{p}{,}\PY{n}{stop}\PY{p}{)}
    \PY{c+c1}{\PYZsh{} returns a list of plots of the basis polynomials, overlaid}
    \PY{k}{return} \PY{p}{[}\PY{n}{plt}\PY{o}{.}\PY{n}{plot}\PY{p}{(}\PY{n}{stiple}\PY{p}{,}\PY{p}{[}\PY{n}{polys}\PY{p}{[}\PY{n}{i}\PY{p}{]}\PY{p}{(}\PY{n}{t}\PY{p}{)} \PY{k}{for} \PY{n}{t} \PY{o+ow}{in} \PY{n}{stiple}\PY{p}{]}\PY{p}{)}\PYZbs{}
            \PY{k}{for} \PY{n}{i} \PY{o+ow}{in} \PY{n+nb}{range}\PY{p}{(}\PY{n+nb}{len}\PY{p}{(}\PY{n}{vecx}\PY{p}{)}\PY{p}{)}\PY{p}{]}

\PY{n}{lagrange\PYZus{}basis\PYZus{}plot}\PY{p}{(}\PY{n}{vecx}\PY{p}{)}
\PY{n}{plt}\PY{o}{.}\PY{n}{show}\PY{p}{(}\PY{p}{)}
\end{InVerbatim}

    \begin{center}
    \adjustimage{max size={0.6\linewidth}{0.6\paperheight}}{2018-07-19-interpolation_files/2018-07-19-interpolation_30_0.png}
    \end{center}
    { \hspace*{\fill} \\}
    
    Now, to show the existence of an interpolating polynomial, we'll specify
a linear combination of basis vectors \(\{\ell_k : k=0,1,\ldots,n\}\).

Let \(f \in C[a,b]\), and specify \(n+1\) distinct points \(x_i\). The
interpolating polynomial in Lagrange form is \(p_n \in \mathcal{P}_n\)
where \[p_n(x) = \sum_{k=0}^n f(x_k) \ell_k(x).\]

We'll scale our basis polynomials:

    
\prompt{In}{incolor}{3}
\begin{InVerbatim}[fontsize=\small,commandchars=\\\{\}]
\PY{k}{def} \PY{n+nf}{lagrange\PYZus{}coords}\PY{p}{(}\PY{n}{f}\PY{p}{,}\PY{n}{vecx}\PY{p}{)}\PY{p}{:}
    \PY{c+c1}{\PYZsh{} returns the n coordinates of the interpolating polynomial}
    \PY{c+c1}{\PYZsh{} as a vector in \PYZdl{}\PYZbs{}mathcal\PYZob{}P\PYZcb{}\PYZus{}n\PYZdl{} with the Lagrange basis}
    \PY{k}{return} \PY{n}{np}\PY{o}{.}\PY{n}{array}\PY{p}{(}\PY{p}{[}\PY{n}{f}\PY{p}{(}\PY{n}{x}\PY{p}{)} \PY{k}{for} \PY{n}{x} \PY{o+ow}{in} \PY{n}{vecx}\PY{p}{]}\PY{p}{)}

\PY{k}{def} \PY{n+nf}{interpol}\PY{p}{(}\PY{n}{coords}\PY{p}{,}\PY{n}{polys}\PY{p}{,}\PY{n}{t}\PY{p}{)}\PY{p}{:}
    \PY{c+c1}{\PYZsh{} scales each vector by its coordinate}
    \PY{n}{scale} \PY{o}{=} \PY{p}{[}\PY{k}{lambda} \PY{n}{p}\PY{p}{,} \PY{n}{a}\PY{o}{=}\PY{n}{a}\PY{p}{:} \PY{n}{p} \PY{o}{*} \PY{n}{a} \PY{k}{for} \PY{n}{a} \PY{o+ow}{in} \PY{n}{coords}\PY{p}{]}
    \PY{c+c1}{\PYZsh{} sums to form the desired linear combination}
    \PY{c+c1}{\PYZsh{} returns a (floating point) scalar}
    \PY{k}{return} \PY{n+nb}{sum}\PY{p}{(}\PY{p}{[}\PY{n}{scale}\PY{p}{[}\PY{n}{i}\PY{p}{]}\PY{p}{(}\PY{n}{polys}\PY{p}{[}\PY{n}{i}\PY{p}{]}\PY{p}{(}\PY{n}{t}\PY{p}{)}\PY{p}{)} \PY{k}{for} \PY{n}{i} \PY{o+ow}{in} \PY{n+nb}{range}\PY{p}{(}\PY{n+nb}{len}\PY{p}{(}\PY{n}{polys}\PY{p}{)}\PY{p}{)}\PY{p}{]}\PY{p}{)}
\end{InVerbatim}

    Since the degree of any linear combination of Lagrange polynomials
(generated by \(n+1\) distinct points) is \(n\), and evaluated at any of
the \(n+1\) generating points \(x_i\), the Lagrange \(k\)th polynomial
is equal to the Kroenecker delta
\[\delta_{ik} = \begin{cases} 1 &\text{ if } i=k\\ 0 &\text{ else,} \end{cases}\]
we have the least degree interpolating polynomial that we sought the
existence of, \(p_n\).

Consider the function \(f(x) = e^x\), and the points \(\{0,1,2,3\}\).

We can compute:

    
\prompt{In}{incolor}{4}
\begin{InVerbatim}[fontsize=\small,commandchars=\\\{\}]
\PY{k+kn}{import} \PY{n+nn}{math}

\PY{n}{f} \PY{o}{=} \PY{k}{lambda} \PY{n}{x}\PY{p}{:} \PY{n}{math}\PY{o}{.}\PY{n}{exp}\PY{p}{(}\PY{n}{x}\PY{p}{)}

\PY{c+c1}{\PYZsh{} to specify the interpolating poly, p}
\PY{n}{polys} \PY{o}{=} \PY{n}{lagrange\PYZus{}basis}\PY{p}{(}\PY{n}{vecx}\PY{p}{)}
\PY{n}{coords} \PY{o}{=} \PY{n}{lagrange\PYZus{}coords}\PY{p}{(}\PY{n}{f}\PY{p}{,}\PY{n}{vecx}\PY{p}{)}
\PY{n}{p} \PY{o}{=} \PY{k}{lambda} \PY{n}{t}\PY{p}{:} \PY{n}{interpol}\PY{p}{(}\PY{n}{coords}\PY{p}{,} \PY{n}{polys}\PY{p}{,} \PY{n}{t}\PY{p}{)}

\PY{c+c1}{\PYZsh{} here\PYZsq{}s a couple evaluations}
\PY{c+c1}{\PYZsh{} notice interpolation has small relative error}
\PY{n+nb}{print}\PY{p}{(}\PY{l+s+s2}{\PYZdq{}}\PY{l+s+s2}{exp(1.5) is approx  }\PY{l+s+s2}{\PYZdq{}}\PY{p}{,} \PY{n}{p}\PY{p}{(}\PY{l+m+mf}{1.5}\PY{p}{)}\PY{p}{)} 
\PY{n+nb}{print}\PY{p}{(}\PY{l+s+s2}{\PYZdq{}}\PY{l+s+s2}{exp(1.5) is exactly }\PY{l+s+s2}{\PYZdq{}}\PY{p}{,} \PY{n}{math}\PY{o}{.}\PY{n}{exp}\PY{p}{(}\PY{l+m+mf}{1.5}\PY{p}{)}\PY{p}{)}

\PY{c+c1}{\PYZsh{} while exterpolation may have great relative error}
\PY{n+nb}{print}\PY{p}{(}\PY{l+s+s2}{\PYZdq{}}\PY{l+s+s2}{exp(4.0) is approx  }\PY{l+s+s2}{\PYZdq{}}\PY{p}{,} \PY{n}{p}\PY{p}{(}\PY{l+m+mf}{4.0}\PY{p}{)}\PY{p}{)}
\PY{n+nb}{print}\PY{p}{(}\PY{l+s+s2}{\PYZdq{}}\PY{l+s+s2}{exp(4.0) is exactly }\PY{l+s+s2}{\PYZdq{}}\PY{p}{,} \PY{n}{math}\PY{o}{.}\PY{n}{exp}\PY{p}{(}\PY{l+m+mf}{4.0}\PY{p}{)}\PY{p}{)}
\end{InVerbatim}

    \begin{Verbatim}[fontsize=\small,commandchars=\\\{\}]
exp(1.5) is approx   4.3675315264574746
exp(1.5) is exactly  4.4816890703380645
exp(4.0) is approx   45.88093841300295
exp(4.0) is exactly  54.598150033144236

    \end{Verbatim}

    
\prompt{In}{incolor}{5}
\begin{InVerbatim}[fontsize=\small,commandchars=\\\{\}]
\PY{c+c1}{\PYZsh{} we sample 50 points in the interval}
\PY{n}{start}\PY{p}{,}\PY{n}{stop} \PY{o}{=} \PY{n}{np}\PY{o}{.}\PY{n}{min}\PY{p}{(}\PY{n}{vecx}\PY{p}{)}\PY{p}{,}\PY{n}{np}\PY{o}{.}\PY{n}{max}\PY{p}{(}\PY{n}{vecx}\PY{p}{)}
\PY{n}{stiple} \PY{o}{=} \PY{n}{np}\PY{o}{.}\PY{n}{linspace}\PY{p}{(}\PY{n}{start}\PY{p}{,}\PY{n}{stop}\PY{p}{)}

\PY{c+c1}{\PYZsh{} evaluate them to make a plot for p}
\PY{n}{fstiple} \PY{o}{=} \PY{p}{[}\PY{n}{interpol}\PY{p}{(}\PY{n}{coords}\PY{p}{,} \PY{n}{polys}\PY{p}{,} \PY{n}{t}\PY{p}{)} \PY{k}{for} \PY{n}{t} \PY{o+ow}{in} \PY{n}{stiple}\PY{p}{]}
\PY{n}{plt}\PY{o}{.}\PY{n}{plot}\PY{p}{(}\PY{n}{stiple}\PY{p}{,}\PY{n}{fstiple}\PY{p}{)}

\PY{c+c1}{\PYZsh{} and overlay the exact plot f}
\PY{n}{plt}\PY{o}{.}\PY{n}{plot}\PY{p}{(}\PY{n}{stiple}\PY{p}{,}\PY{n}{np}\PY{o}{.}\PY{n}{array}\PY{p}{(}\PY{p}{[}\PY{n}{f}\PY{p}{(}\PY{n}{t}\PY{p}{)} \PY{k}{for} \PY{n}{t} \PY{o+ow}{in} \PY{n}{stiple}\PY{p}{]}\PY{p}{)}\PY{p}{)}
\PY{n}{plt}\PY{o}{.}\PY{n}{show}\PY{p}{(}\PY{p}{)}
\end{InVerbatim}

    \begin{center}
    \adjustimage{max size={0.6\linewidth}{0.6\paperheight}}{2018-07-19-interpolation_files/2018-07-19-interpolation_35_0.png}
    \end{center}
    { \hspace*{\fill} \\}
    
    \hypertarget{newton-form}{%
\subsubsection{Newton form}\label{newton-form}}

The Newton form allows us to compute \(p_n\) from \(p_{n-1}\)
recursively. That is, we define \(p_n\) with the same \(n-1\) basis
vectors and \(n-1\) coordinates as \(p_{n-1}\), but add a basis vector
that's \(0\) at all points \(x_i\) where \(i < n\).

Namely, \[p_n(x) = p_{n-1}(x)+a_n\prod_{i < n} (x-x_i)\] with
\[a_n = \left(f(x_n) - p_{n-1}(x_n)\right)\left(\prod_{i < n} (x_n-x_i)\right)^{-1}.\]

Here's a constructive definition, with accompanying plots for the same
generating points \(\{0,1,2,3\}\).

    
\prompt{In}{incolor}{6}
\begin{InVerbatim}[fontsize=\small,commandchars=\\\{\}]
\PY{k}{def} \PY{n+nf}{newton\PYZus{}basis}\PY{p}{(}\PY{n}{vecx}\PY{p}{)}\PY{p}{:}
    \PY{n}{n} \PY{o}{=} \PY{n+nb}{len}\PY{p}{(}\PY{n}{vecx}\PY{p}{)}
    \PY{c+c1}{\PYZsh{} returns a list of n functions, the newton form basis polynomials}
    \PY{k}{return} \PY{p}{[}\PY{k}{lambda} \PY{n}{t}\PY{p}{,} \PY{n}{j}\PY{o}{=}\PY{n}{j}\PY{p}{:} \PY{n}{np}\PY{o}{.}\PY{n}{prod}\PY{p}{(}\PY{p}{[}\PY{n}{t}\PY{o}{\PYZhy{}}\PY{n}{vecx}\PY{p}{[}\PY{n}{i}\PY{p}{]} \PY{k}{for} \PY{n}{i} \PY{o+ow}{in} \PY{n+nb}{range}\PY{p}{(}\PY{n}{j}\PY{p}{)}\PY{p}{]}\PY{p}{)}\PYZbs{}
            \PY{k}{for} \PY{n}{j} \PY{o+ow}{in} \PY{n+nb}{range}\PY{p}{(}\PY{n}{n}\PY{p}{)}\PY{p}{]}

\PY{k}{def} \PY{n+nf}{newton\PYZus{}basis\PYZus{}plot}\PY{p}{(}\PY{n}{vecx}\PY{p}{)}\PY{p}{:}
    \PY{n}{polys} \PY{o}{=} \PY{n}{newton\PYZus{}basis}\PY{p}{(}\PY{n}{vecx}\PY{p}{)}
    \PY{n}{start}\PY{p}{,} \PY{n}{stop} \PY{o}{=} \PY{n}{np}\PY{o}{.}\PY{n}{min}\PY{p}{(}\PY{n}{vecx}\PY{p}{)}\PY{p}{,} \PY{n}{np}\PY{o}{.}\PY{n}{max}\PY{p}{(}\PY{n}{vecx}\PY{p}{)}
    \PY{n}{stiple} \PY{o}{=} \PY{n}{np}\PY{o}{.}\PY{n}{linspace}\PY{p}{(}\PY{n}{start}\PY{p}{,}\PY{n}{stop}\PY{p}{)}
    \PY{c+c1}{\PYZsh{} returns a list of plots of the basis polynomials, overlaid}
    \PY{k}{return} \PY{p}{[}\PY{n}{plt}\PY{o}{.}\PY{n}{plot}\PY{p}{(}\PY{n}{stiple}\PY{p}{,}\PY{p}{[}\PY{n}{polys}\PY{p}{[}\PY{n}{i}\PY{p}{]}\PY{p}{(}\PY{n}{t}\PY{p}{)} \PY{k}{for} \PY{n}{t} \PY{o+ow}{in} \PY{n}{stiple}\PY{p}{]}\PY{p}{)}\PYZbs{}
            \PY{k}{for} \PY{n}{i} \PY{o+ow}{in} \PY{n+nb}{range}\PY{p}{(}\PY{n+nb}{len}\PY{p}{(}\PY{n}{vecx}\PY{p}{)}\PY{p}{)}\PY{p}{]}

\PY{n}{newton\PYZus{}basis\PYZus{}plot}\PY{p}{(}\PY{n}{vecx}\PY{p}{)}
\PY{n}{plt}\PY{o}{.}\PY{n}{show}\PY{p}{(}\PY{p}{)}
\end{InVerbatim}

    \begin{center}
    \adjustimage{max size={0.6\linewidth}{0.6\paperheight}}{2018-07-19-interpolation_files/2018-07-19-interpolation_37_0.png}
    \end{center}
    { \hspace*{\fill} \\}
    
    In practice we express the coordinate \(a_n\) in recursive \emph{divided
differences}, i.e.,
\[a_n = f[x_0,\ldots,x_n] = \frac{f[x_1, \ldots, x_n] - f[x_0, \ldots, x_{n-1}]}{x_n-x_0}\]
where the identification \(f[x] = f(x)\) for all \(x\) is the base for
recursion.

    
\prompt{In}{incolor}{7}
\begin{InVerbatim}[fontsize=\small,commandchars=\\\{\}]
\PY{k}{def} \PY{n+nf}{dd}\PY{p}{(}\PY{n}{f}\PY{p}{,} \PY{n}{vecx}\PY{p}{)}\PY{p}{:}
    \PY{k}{if} \PY{n+nb}{len}\PY{p}{(}\PY{n}{vecx}\PY{p}{)}\PY{o}{\PYZgt{}}\PY{l+m+mi}{1}\PY{p}{:}
    \PY{c+c1}{\PYZsh{} recursive call, returns the coordinate of the nth newton poly}
        \PY{k}{return} \PY{p}{(}\PY{n}{dd}\PY{p}{(}\PY{n}{f}\PY{p}{,}\PY{n}{vecx}\PY{p}{[}\PY{l+m+mi}{1}\PY{p}{:}\PY{p}{]}\PY{p}{)} \PY{o}{\PYZhy{}} \PY{n}{dd}\PY{p}{(}\PY{n}{f}\PY{p}{,}\PY{n}{vecx}\PY{p}{[}\PY{p}{:}\PY{o}{\PYZhy{}}\PY{l+m+mi}{1}\PY{p}{]}\PY{p}{)}\PY{p}{)}\PY{o}{/}\PY{p}{(}\PY{n}{vecx}\PY{p}{[}\PY{o}{\PYZhy{}}\PY{l+m+mi}{1}\PY{p}{]}\PY{o}{\PYZhy{}}\PY{n}{vecx}\PY{p}{[}\PY{l+m+mi}{0}\PY{p}{]}\PY{p}{)}
    \PY{c+c1}{\PYZsh{} floor of recursion}
    \PY{k}{else}\PY{p}{:}
        \PY{k}{return} \PY{n}{f}\PY{p}{(}\PY{n}{vecx}\PY{p}{[}\PY{l+m+mi}{0}\PY{p}{]}\PY{p}{)}

\PY{k}{def} \PY{n+nf}{newton\PYZus{}coords}\PY{p}{(}\PY{n}{f}\PY{p}{,} \PY{n}{vecx}\PY{p}{)}\PY{p}{:}
    \PY{c+c1}{\PYZsh{} TODO: save the recursive divided difference calls in memory}
    \PY{c+c1}{\PYZsh{} so we don\PYZsq{}t have to iterate to load them into this array}
    \PY{k}{return} \PY{n}{np}\PY{o}{.}\PY{n}{array}\PY{p}{(}\PY{p}{[}\PY{n}{dd}\PY{p}{(}\PY{n}{f}\PY{p}{,}\PY{n}{vecx}\PY{p}{[}\PY{p}{:}\PY{n}{i}\PY{o}{+}\PY{l+m+mi}{1}\PY{p}{]}\PY{p}{)} \PY{k}{for} \PY{n}{i} \PY{o+ow}{in} \PY{n+nb}{range}\PY{p}{(}\PY{n+nb}{len}\PY{p}{(}\PY{n}{vecx}\PY{p}{)}\PY{p}{)}\PY{p}{]}\PY{p}{)}
\end{InVerbatim}

    When working with more than \sout{a thousand} 30 generating points, we'd
have to rewrite the above programs to cache intermediate recursive
function calls. The most straight forward means to cache the divided
differences is in a triangular matrix, writing an imperative program to
append new diagonals for the calculation of the \(n\)th coefficient
\(a_n\).

The sentiment is also expressed in the setting of finite differences.

\begin{quote}
An important advantage of finite-difference interpolation
formulas\ldots{}would seem to be the property\ldots{}that enables a term
to be added to them merely by adding one tabular point and computing an
additional row of differences. \ldots{}This enables us to generate a
sequence of interpolants each involving one more tabular point than the
previous one. (1965 A. Ralston First Course Numerical Analysis iii. 57)
\end{quote}

(Aside: Python is not written to cache recursive calls, and as a rule
taps out at 1000 levels deep. Pure functional, strictly typed languages
like Haskell have compilers optimized to cache all recursions, and as
well won't complain at any number of recursive calls.)

Anyways, the method does run smoothly for
\texttt{len(vecx)\ \textless{}\ 25}:

    
\prompt{In}{incolor}{8}
\begin{InVerbatim}[fontsize=\small,commandchars=\\\{\}]
\PY{n}{polys} \PY{o}{=} \PY{n}{newton\PYZus{}basis}\PY{p}{(}\PY{n}{vecx}\PY{p}{)}
\PY{n}{coords} \PY{o}{=} \PY{n}{newton\PYZus{}coords}\PY{p}{(}\PY{n}{f}\PY{p}{,}\PY{n}{vecx}\PY{p}{)}
\PY{n}{start}\PY{p}{,}\PY{n}{stop} \PY{o}{=} \PY{n}{np}\PY{o}{.}\PY{n}{min}\PY{p}{(}\PY{n}{vecx}\PY{p}{)}\PY{p}{,}\PY{n}{np}\PY{o}{.}\PY{n}{max}\PY{p}{(}\PY{n}{vecx}\PY{p}{)}
\PY{n}{stiple} \PY{o}{=} \PY{n}{np}\PY{o}{.}\PY{n}{linspace}\PY{p}{(}\PY{n}{start}\PY{p}{,}\PY{n}{stop}\PY{p}{)}
\PY{n}{fstiple} \PY{o}{=} \PY{p}{[}\PY{n}{interpol}\PY{p}{(}\PY{n}{coords}\PY{p}{,} \PY{n}{polys}\PY{p}{,} \PY{n}{t}\PY{p}{)} \PY{k}{for} \PY{n}{t} \PY{o+ow}{in} \PY{n}{stiple}\PY{p}{]}
\PY{n}{plt}\PY{o}{.}\PY{n}{plot}\PY{p}{(}\PY{n}{stiple}\PY{p}{,}\PY{n}{fstiple}\PY{p}{)}
\PY{n}{plt}\PY{o}{.}\PY{n}{plot}\PY{p}{(}\PY{n}{stiple}\PY{p}{,}\PY{n}{np}\PY{o}{.}\PY{n}{array}\PY{p}{(}\PY{p}{[}\PY{n}{f}\PY{p}{(}\PY{n}{t}\PY{p}{)} \PY{k}{for} \PY{n}{t} \PY{o+ow}{in} \PY{n}{stiple}\PY{p}{]}\PY{p}{)}\PY{p}{)}
\PY{n}{plt}\PY{o}{.}\PY{n}{show}\PY{p}{(}\PY{p}{)}
\end{InVerbatim}

    \begin{center}
    \adjustimage{max size={0.6\linewidth}{0.6\paperheight}}{2018-07-19-interpolation_files/2018-07-19-interpolation_41_0.png}
    \end{center}
    { \hspace*{\fill} \\}
    
    We note the Newton form,
\[p_n(x) = a_0 + a_1(x-x_0) + \cdots + a_n(x-x_0)\cdots(x-x_{n-1})\] can
be optimized also to be evaluated in \(O(n)\) operations, namely, \(n\)
multiplications and \(n\) additions, by factoring
\[p_n(x) =  a_0 + (x-x_0)\bigg(a_1 + (x-x_1)\Big(a_2+\cdots + a_n(x-x_{n-1})\cdots\Big)\bigg).\]

Our code presently \emph{does not} redistribute parentheses to save
computation, instead, it's written to compute each basis polynomial
explicitly (in however many multiplications), scale them all, then sum.
Such operations correspond closing with the matrix interpretation of the
constraint \(p_n = f\) at \(n+1\) points \(x_i\).

\hypertarget{matrix-interpretation}{%
\subsubsection{Matrix interpretation}\label{matrix-interpretation}}

For \(n+1\) generating points \((x_i,f(x_i))\), there is a unique
interpolating polynomial \(p_n\) of degree \(n\).

But \(p_n\) is just point in the vector space \(\mathcal{P}_n\), so we
should try to express the constraint that \(p_n(x_i) = f(x_i)\) for all
\(i \in \{0,1,\ldots,n\}\) as \(n+1\) linear equations to determine the
coordinates of \(p_n\).

The constraint as a matrix equation (in any basis) is
\[A \begin{pmatrix}a_0\\ \vdots\\ a_n\end{pmatrix} = \begin{pmatrix}f(x_0)\\ \vdots\\ f(x_n)\end{pmatrix}.\]

We are required to choose a basis for \(\mathcal{P}_n\), and we've thus
far seen two:

\begin{itemize}
\tightlist
\item
  the Lagrange polynomials \(\{\ell_0, \ell_1, \ldots, \ell_n\}\)
\item
  the basis of the Newton form \(\{1, (x-x_0), \ldots, (x-x_n)\}\)
\end{itemize}

For the Lagrange polynomial basis, we have simply \(A = I\). In the
Newton basis, \(A\) is a lower triangular matrix.

A third basis, the standard basis \(\{1,x,x^2,\ldots, x^n\}\), yields a
power series approximation to \(f\), but corresponds to the highly
ill-conditioned \emph{Vandermode} matrix.

\hypertarget{error-analysis}{%
\subsubsection{Error analysis}\label{error-analysis}}

Let \(f \in C^{n+1}[a,b]\), the \(x_i\) be \(n+1\) distinct points in
\([a,b]\), and suppose \(p_n\) is a polynomial of degree at most \(n\)
that interpolates \(f\) at the \(x_i\). Given \(x \in [a,b]\), there
exists a point \(\xi \in (a,b)\) such that
\[f(x) = p_n(x) + \frac{f^{(n+1)}(\xi)}{(n+1)!}\prod_{\forall i}(x-x_i).\]

For example, interpolating \(f = x \mapsto e^x\) on \([0,3]\) with the
generating points \(\{0,1,2,3\}\), the error is constrained:

    
\prompt{In}{incolor}{9}
\begin{InVerbatim}[fontsize=\small,commandchars=\\\{\}]
\PY{k+kn}{import} \PY{n+nn}{sympy} \PY{k}{as} \PY{n+nn}{sym}
\PY{n}{z} \PY{o}{=} \PY{n}{sym}\PY{o}{.}\PY{n}{symbols}\PY{p}{(}\PY{l+s+s1}{\PYZsq{}}\PY{l+s+s1}{z}\PY{l+s+s1}{\PYZsq{}}\PY{p}{)}
\PY{n}{prod} \PY{o}{=} \PY{n}{np}\PY{o}{.}\PY{n}{prod}\PY{p}{(}\PY{p}{[}\PY{n}{z}\PY{o}{\PYZhy{}}\PY{n}{i} \PY{k}{for} \PY{n}{i} \PY{o+ow}{in} \PY{n+nb}{range}\PY{p}{(}\PY{l+m+mi}{4}\PY{p}{)}\PY{p}{]}\PY{p}{)}
\PY{n}{max\PYZus{}prod} \PY{o}{=} \PY{n+nb}{max}\PY{p}{(}\PY{p}{[}\PY{n}{prod}\PY{o}{.}\PY{n}{subs}\PY{p}{(}\PY{n}{z}\PY{p}{,}\PY{n}{t}\PY{p}{)} \PY{k}{for} \PY{n}{t} \PY{o+ow}{in} \PY{n}{sym}\PY{o}{.}\PY{n}{solve}\PY{p}{(}\PY{n}{sym}\PY{o}{.}\PY{n}{diff}\PY{p}{(}\PY{n}{prod}\PY{p}{,}\PY{n}{z}\PY{p}{)}\PY{p}{,} \PY{n}{z}\PY{p}{)}\PY{p}{]}\PY{p}{)}
\PY{c+c1}{\PYZsh{} take the worst xi possible, namely, xi=3}
\PY{n}{max\PYZus{}error} \PY{o}{=} \PY{p}{(}\PY{n}{math}\PY{o}{.}\PY{n}{exp}\PY{p}{(}\PY{l+m+mi}{3}\PY{p}{)}\PY{o}{/}\PY{l+m+mi}{24}\PY{p}{)}\PY{o}{*}\PY{n}{max\PYZus{}prod}
\PY{n}{max\PYZus{}error}
\end{InVerbatim}

            
\prompt{Out}{outcolor}{9}
\begin{OutVerbatim}[fontsize=\small,commandchars=\\\{\}]
0.470754771637211
\end{OutVerbatim}
        
    \hypertarget{determining-the-minimal-degree}{%
\subsection{Determining the minimal
degree}\label{determining-the-minimal-degree}}

Here's a linear algebraic method to determine the degree of the minimal
interpolating polynomial for a continuous function sampled at \(n+1\)
distinct points. It's equivalent to the computing the Newton form of the
interpolating polynomial through divided differences, but emphasises
that the constraints \(p(x_i) = f(x_i)\) can be satisfied with a exactly
one linear combination of vectors in the Newton form polynomial basis.

\begin{enumerate}
\def\labelenumi{\arabic{enumi}.}
\tightlist
\item
  Write out the Newton form of the matrix \(A\) in the constraint
  equation,
  \(A \begin{pmatrix}a_0\\ \vdots\\ a_n\end{pmatrix} = \begin{pmatrix}f(x_0)\\ \vdots\\ f(x_n)\end{pmatrix}\).
\end{enumerate}

\[A = \begin{pmatrix} 1 & & & 0\\
                       1 &(x_1 - x_0) & & \\
                       \vdots & \vdots & \ddots & \\
                       1 &(x_n - x_0) &\cdots &\prod_{i<n}(x_n - x_i)
       \end{pmatrix}\]

Elementwise, that's
\[(A_{ij}) = \left(\prod_{k=0}^{j-1} (x_i - x_k)\right).\]

\begin{enumerate}
\def\labelenumi{\arabic{enumi}.}
\setcounter{enumi}{1}
\item
  Compute its inverse \(A^{-1}\) (which exists as \(\det A\) is the
  product of \(A\)'s nonzero diagonal entries).
\item
  Multiply
  \(A^{-1} \begin{pmatrix}f(x_0)\\ \vdots\\ f(x_n)\end{pmatrix}\) to
  find \(\vec{a} = \begin{pmatrix}a_0\\ \vdots\\ a_n\end{pmatrix}\).
\item
  The degree of the interpolating polynomial is the index of the last
  nonzero element in \(\vec{a}\).
\end{enumerate}

As a basic example, consider the points \((0,1)\), \((1,1)\), and
\((2,1)\); these are interpolated by the constant polynomial \(p = 1\).
Our method should corroborate this conclusion.

\begin{enumerate}
\def\labelenumi{\arabic{enumi}.}
\item
  With the Newton form of the basis, the matrix \(A\) is given
  \[A = \begin{pmatrix} 1 & 0 & 0\\ 1 & 1 & 0 \\ 1 & 2 & 2\end{pmatrix}.\]
\item
  The inverse exists and is given by
  \[A^{-1}= \begin{pmatrix} 1 & 0 & 0 \\ -1 & 1 & 0\\ \frac12 & -1 & \frac12 \end{pmatrix}.\]
\item
  The coefficient vector \(\vec{a}\) is thus
  \[A^{-1} \begin{pmatrix}1 \\ 1 \\ 1\end{pmatrix} = \begin{pmatrix} 1\\0\\0\end{pmatrix}.\]
\item
  The degree of the interpolating polynomial is therefore \(0\), as
  \(a_0 = 1\) is the last non-zero entry in the coefficient vector. It
  follows that the interpolating polynomial has the (Newton) form
  \[p = \begin{pmatrix}1\\0\\0\end{pmatrix}\cdot\begin{pmatrix}x^0 \\x^1\\x^2\end{pmatrix} = 1\cdot x^0 + 0\cdot x^1 + 0 \cdot x^2.\]
\end{enumerate}

Proceeding in the same fashion, we can find the degree of the minimal
polynomial that interpolates the following \(6\) points:

    
\prompt{In}{incolor}{10}
\begin{InVerbatim}[fontsize=\small,commandchars=\\\{\}]
\PY{n}{points} \PY{o}{=} \PY{n}{np}\PY{o}{.}\PY{n}{array}\PY{p}{(}\PY{p}{[}\PY{p}{[}\PY{o}{\PYZhy{}}\PY{l+m+mi}{2}\PY{p}{,} \PY{o}{\PYZhy{}}\PY{l+m+mi}{1}\PY{p}{,}  \PY{l+m+mi}{0}\PY{p}{,}  \PY{l+m+mi}{1}\PY{p}{,}  \PY{l+m+mi}{2}\PY{p}{,}  \PY{l+m+mi}{3}\PY{p}{]}\PY{p}{,} \PY{p}{[}\PY{o}{\PYZhy{}}\PY{l+m+mi}{5}\PY{p}{,}  \PY{l+m+mi}{1}\PY{p}{,}  \PY{l+m+mi}{1}\PY{p}{,}  \PY{l+m+mi}{1}\PY{p}{,}  \PY{l+m+mi}{7}\PY{p}{,} \PY{l+m+mi}{25}\PY{p}{]}\PY{p}{]}\PY{p}{)}

\PY{k}{def} \PY{n+nf}{nfA}\PY{p}{(}\PY{n}{vecx}\PY{p}{)}\PY{p}{:}
    \PY{n}{n} \PY{o}{=} \PY{n+nb}{len}\PY{p}{(}\PY{n}{vecx}\PY{p}{)}
    \PY{n}{a} \PY{o}{=} \PY{k}{lambda} \PY{n}{i}\PY{p}{,}\PY{n}{j} \PY{p}{:} \PY{n}{np}\PY{o}{.}\PY{n}{prod}\PY{p}{(}\PY{p}{[}\PY{n}{vecx}\PY{p}{[}\PY{n}{i}\PY{p}{]}\PY{o}{\PYZhy{}}\PY{n}{vecx}\PY{p}{[}\PY{n}{k}\PY{p}{]} \PY{k}{for} \PY{n}{k} \PY{o+ow}{in} \PY{n+nb}{range}\PY{p}{(}\PY{n}{j}\PY{p}{)}\PY{p}{]}\PY{p}{)}
    \PY{k}{return} \PY{n}{np}\PY{o}{.}\PY{n}{array}\PY{p}{(}\PY{p}{[}\PY{p}{[}\PY{n}{a}\PY{p}{(}\PY{n}{i}\PY{p}{,}\PY{n}{j}\PY{p}{)} \PY{k}{for} \PY{n}{j} \PY{o+ow}{in} \PY{n+nb}{range}\PY{p}{(}\PY{n}{n}\PY{p}{)}\PY{p}{]} \PY{k}{for} \PY{n}{i} \PY{o+ow}{in} \PY{n+nb}{range}\PY{p}{(}\PY{n}{n}\PY{p}{)}\PY{p}{]}\PY{p}{)}

\PY{n}{nfA}\PY{p}{(}\PY{n}{points}\PY{p}{[}\PY{l+m+mi}{0}\PY{p}{]}\PY{p}{)}
\end{InVerbatim}

            
\prompt{Out}{outcolor}{10}
\begin{OutVerbatim}[fontsize=\small,commandchars=\\\{\}]
array([[  1.,   0.,   0.,   0.,   0.,   0.],
       [  1.,   1.,   0.,   0.,   0.,   0.],
       [  1.,   2.,   2.,   0.,   0.,   0.],
       [  1.,   3.,   6.,   6.,   0.,   0.],
       [  1.,   4.,  12.,  24.,  24.,   0.],
       [  1.,   5.,  20.,  60., 120., 120.]])
\end{OutVerbatim}
        
    
\prompt{In}{incolor}{11}
\begin{InVerbatim}[fontsize=\small,commandchars=\\\{\}]
\PY{c+c1}{\PYZsh{} this function equivalent to computing divided differences}
\PY{n}{np}\PY{o}{.}\PY{n}{linalg}\PY{o}{.}\PY{n}{solve}\PY{p}{(}\PY{n}{nfA}\PY{p}{(}\PY{n}{points}\PY{p}{[}\PY{l+m+mi}{0}\PY{p}{]}\PY{p}{)}\PY{p}{,}\PY{n}{points}\PY{p}{[}\PY{l+m+mi}{1}\PY{p}{]}\PY{p}{)}
\end{InVerbatim}

            
\prompt{Out}{outcolor}{11}
\begin{OutVerbatim}[fontsize=\small,commandchars=\\\{\}]
array([-5.,  6., -3.,  1.,  0.,  0.])
\end{OutVerbatim}
        
    We see the last nonzero entry in the coefficient vector has index \(3\),
and therefore the minimal interpolating polynomial has degree \(3\).

\hypertarget{summing-a-set-of-lagrange-polynomials}{%
\subsection{Summing a set of Lagrange
polynomials}\label{summing-a-set-of-lagrange-polynomials}}

Suppose we have \(n+1\) distinct points \(x_i \in \mathbf{R}\). Let the
\(n+1\) Lagrange polynomials be given by \(\ell_k\) for
\(k \in \{0,1,\ldots,n\}\). Write \(p_n = \sum_{k=0}^n 1\cdot\ell_k\),
which interpolates the constant polynomial \(f = 1\). For all real \(x\)
the error \(\left\lvert f(x) - p_n(x)\right\rvert\) is bounded by
\[\max_{\xi\in S}\left\lvert\frac{f^{(n+1)}(x)}{(n+1)!}\prod_{\forall i} (\xi - x_i)\right\rvert  = 0\]
as all derivatives of the constant function \(f\) are identically zero.
We conclude that \(\sum{k=0}^n \ell_k \equiv 1\in \mathcal{P}_n\).

\hypertarget{optimal-choice-of-sample-points}{%
\subsection{Optimal choice of sample
points}\label{optimal-choice-of-sample-points}}

We plot two polynomials \(w_n(x) = \prod_{\forall i} (x-x_i)\) generated
by a set of \(n+1\) points \(x_i\), the first generated by linearly
spaced points, the second generated by Chebyshev points.

We'll exploit the identity
\[\frac{\pi}{n+1}\left(\frac12 + i\right) = \frac{\pi}{2} + \frac{\pi}{2}\cdot\frac{n}{n+1}\left(-1 + \frac{2i}{n}\right)\]
to express the roots of the \(n\)th Chebyshev polynomial as an
vectorized transformation of the linearly spaced array \texttt{vecx}.

    
\prompt{In}{incolor}{12}
\begin{InVerbatim}[fontsize=\small,commandchars=\\\{\}]
\PY{k}{def} \PY{n+nf}{w}\PY{p}{(}\PY{n}{vecx}\PY{p}{,} \PY{n}{x}\PY{p}{)}\PY{p}{:} 
    \PY{k}{return} \PY{n}{np}\PY{o}{.}\PY{n}{product}\PY{p}{(}\PY{n}{vecx} \PY{o}{\PYZhy{}} \PY{n}{x}\PY{p}{)}

\PY{k}{def} \PY{n+nf}{chebyshev\PYZus{}transform}\PY{p}{(}\PY{n}{vecx}\PY{p}{)}\PY{p}{:}
    \PY{n}{n} \PY{o}{=} \PY{n+nb}{len}\PY{p}{(}\PY{n}{vecx}\PY{p}{)} \PY{o}{\PYZhy{}} \PY{l+m+mi}{1}
    \PY{k}{return} \PY{o}{\PYZhy{}}\PY{l+m+mi}{1}\PY{o}{*}\PY{n}{np}\PY{o}{.}\PY{n}{cos}\PY{p}{(}\PY{n}{vecx}\PY{o}{*}\PY{p}{(}\PY{n}{np}\PY{o}{.}\PY{n}{pi}\PY{o}{/}\PY{l+m+mi}{2}\PY{p}{)}\PY{o}{*}\PY{n}{n}\PY{o}{/}\PY{p}{(}\PY{n}{n}\PY{o}{+}\PY{l+m+mi}{1}\PY{p}{)} \PY{o}{+} \PY{n}{np}\PY{o}{.}\PY{n}{pi}\PY{o}{/}\PY{l+m+mi}{2}\PY{p}{)}

\PY{k}{def} \PY{n+nf}{w\PYZus{}plot}\PY{p}{(}\PY{n}{vecx}\PY{p}{)}\PY{p}{:}
    \PY{n}{stiple} \PY{o}{=} \PY{n}{np}\PY{o}{.}\PY{n}{linspace}\PY{p}{(}\PY{o}{\PYZhy{}}\PY{l+m+mi}{1}\PY{p}{,}\PY{l+m+mi}{1}\PY{p}{,}\PY{l+m+mi}{100}\PY{p}{)}
    \PY{n}{w\PYZus{}values} \PY{o}{=} \PY{n}{np}\PY{o}{.}\PY{n}{array}\PY{p}{(}\PY{p}{[}\PY{n}{w}\PY{p}{(}\PY{n}{vecx}\PY{p}{,} \PY{n}{i}\PY{p}{)} \PY{k}{for} \PY{n}{i} \PY{o+ow}{in} \PY{n}{stiple}\PY{p}{]}\PY{p}{)}
    \PY{k}{return} \PY{n}{plt}\PY{o}{.}\PY{n}{plot}\PY{p}{(}\PY{n}{stiple}\PY{p}{,}\PY{n}{w\PYZus{}values}\PY{p}{)}

\PY{k}{def} \PY{n+nf}{w\PYZus{}contrast}\PY{p}{(}\PY{n}{vecx}\PY{p}{)}\PY{p}{:}
    \PY{k}{return} \PY{n}{w\PYZus{}plot}\PY{p}{(}\PY{n}{vecx}\PY{p}{)}\PY{p}{,} \PY{n}{w\PYZus{}plot}\PY{p}{(}\PY{n}{chebyshev\PYZus{}transform}\PY{p}{(}\PY{n}{vecx}\PY{p}{)}\PY{p}{)}

\PY{k}{for} \PY{n}{i} \PY{o+ow}{in} \PY{p}{[}\PY{l+m+mi}{10}\PY{p}{,}\PY{l+m+mi}{20}\PY{p}{,}\PY{l+m+mi}{30}\PY{p}{]}\PY{p}{:}
    \PY{n}{w\PYZus{}contrast}\PY{p}{(}\PY{n}{np}\PY{o}{.}\PY{n}{linspace}\PY{p}{(}\PY{o}{\PYZhy{}}\PY{l+m+mi}{1}\PY{p}{,}\PY{l+m+mi}{1}\PY{p}{,}\PY{n}{i}\PY{p}{)}\PY{p}{)}
    \PY{n}{plt}\PY{o}{.}\PY{n}{title}\PY{p}{(}\PY{l+s+s1}{\PYZsq{}}\PY{l+s+s1}{\PYZdl{}w\PYZus{}}\PY{l+s+s1}{\PYZob{}}\PY{l+s+si}{\PYZpc{}d}\PY{l+s+s1}{\PYZcb{}\PYZdl{}}\PY{l+s+s1}{\PYZsq{}} \PY{o}{\PYZpc{}} \PY{n}{i}\PY{p}{)}
    \PY{n}{plt}\PY{o}{.}\PY{n}{legend}\PY{p}{(}\PY{p}{(}\PY{l+s+s1}{\PYZsq{}}\PY{l+s+s1}{Linear\PYZhy{}spaced}\PY{l+s+s1}{\PYZsq{}}\PY{p}{,}\PY{l+s+s1}{\PYZsq{}}\PY{l+s+s1}{Chebyshev}\PY{l+s+s1}{\PYZsq{}}\PY{p}{)}\PY{p}{)}
    \PY{n}{plt}\PY{o}{.}\PY{n}{show}\PY{p}{(}\PY{p}{)}
\end{InVerbatim}

    \begin{center}
    \adjustimage{max size={0.6\linewidth}{0.6\paperheight}}{2018-07-19-interpolation_files/2018-07-19-interpolation_48_0.png}
    \end{center}
    { \hspace*{\fill} \\}
    
    \begin{center}
    \adjustimage{max size={0.6\linewidth}{0.6\paperheight}}{2018-07-19-interpolation_files/2018-07-19-interpolation_48_1.png}
    \end{center}
    { \hspace*{\fill} \\}
    
    \begin{center}
    \adjustimage{max size={0.6\linewidth}{0.6\paperheight}}{2018-07-19-interpolation_files/2018-07-19-interpolation_48_2.png}
    \end{center}
    { \hspace*{\fill} \\}
    
    Notice the \(l_\infty\) norm of the Chebyshev \(w_n\) is bounded by
\(2^{1-n}\).

\hypertarget{comparison-chebyshev-and-linearly-spaced-points}{%
\subsubsection{Comparison: Chebyshev and linearly spaced
points}\label{comparison-chebyshev-and-linearly-spaced-points}}

Here we interpolate two functions (the absolute value function and its
first derivative) with the polynomial \(p_n\) (for \(n = 8, 16, 32\))
generated by \(n+1\) linearly spaced and chebyshev points on the
interval \([-1,1]\), respectively.

    
\prompt{In}{incolor}{13}
\begin{InVerbatim}[fontsize=\small,commandchars=\\\{\}]
\PY{k}{def} \PY{n+nf}{f1}\PY{p}{(}\PY{n}{x}\PY{p}{)}\PY{p}{:} 
    \PY{k}{return} \PY{n+nb}{abs}\PY{p}{(}\PY{n}{x}\PY{p}{)}

\PY{k}{def} \PY{n+nf}{f2}\PY{p}{(}\PY{n}{x}\PY{p}{)}\PY{p}{:} 
    \PY{k}{return} \PY{o}{\PYZhy{}}\PY{l+m+mi}{1} \PY{k}{if} \PY{n}{x}\PY{o}{\PYZlt{}}\PY{l+m+mi}{0} \PY{k}{else} \PY{l+m+mi}{1} \PY{k}{if} \PY{n}{x}\PY{o}{\PYZgt{}}\PY{l+m+mi}{0} \PY{k}{else} \PY{l+m+mi}{0}

\PY{k}{def} \PY{n+nf}{ptuples}\PY{p}{(}\PY{n}{f}\PY{p}{,} \PY{n}{vecx}\PY{p}{)}\PY{p}{:}
    \PY{n}{polys} \PY{o}{=} \PY{n}{lagrange\PYZus{}basis}\PY{p}{(}\PY{n}{vecx}\PY{p}{)}
    \PY{n}{coords} \PY{o}{=} \PY{n}{lagrange\PYZus{}coords}\PY{p}{(}\PY{n}{f}\PY{p}{,} \PY{n}{vecx}\PY{p}{)}
    \PY{n}{stiple} \PY{o}{=} \PY{n}{np}\PY{o}{.}\PY{n}{linspace}\PY{p}{(}\PY{o}{\PYZhy{}}\PY{l+m+mi}{1}\PY{p}{,}\PY{l+m+mi}{1}\PY{p}{,}\PY{l+m+mi}{100}\PY{p}{)}
    \PY{n}{interstiple} \PY{o}{=} \PY{n}{np}\PY{o}{.}\PY{n}{array}\PY{p}{(}\PY{p}{[}\PY{n}{interpol}\PY{p}{(}\PY{n}{coords}\PY{p}{,}\PY{n}{polys}\PY{p}{,}\PY{n}{t}\PY{p}{)} \PY{k}{for} \PY{n}{t} \PY{o+ow}{in} \PY{n}{stiple}\PY{p}{]}\PY{p}{)}
    \PY{k}{return} \PY{n}{np}\PY{o}{.}\PY{n}{vstack}\PY{p}{(}\PY{p}{[}\PY{n}{stiple}\PY{p}{,}\PY{n}{interstiple}\PY{p}{]}\PY{p}{)}

\PY{k}{for} \PY{n}{f} \PY{o+ow}{in} \PY{p}{[}\PY{n}{f1}\PY{p}{,}\PY{n}{f2}\PY{p}{]}\PY{p}{:}
    \PY{k}{for} \PY{n}{n} \PY{o+ow}{in} \PY{p}{[}\PY{l+m+mi}{8}\PY{p}{,} \PY{l+m+mi}{16}\PY{p}{,} \PY{l+m+mi}{32}\PY{p}{]}\PY{p}{:}
        \PY{n}{linear} \PY{o}{=} \PY{n}{np}\PY{o}{.}\PY{n}{linspace}\PY{p}{(}\PY{o}{\PYZhy{}}\PY{l+m+mi}{1}\PY{p}{,}\PY{l+m+mi}{1}\PY{p}{,}\PY{n}{n}\PY{o}{+}\PY{l+m+mi}{1}\PY{p}{)}
        \PY{n}{chebyshev} \PY{o}{=} \PY{n}{chebyshev\PYZus{}transform}\PY{p}{(}\PY{n}{linear}\PY{p}{)}
        \PY{n}{p\PYZus{}linear} \PY{o}{=} \PY{n}{ptuples}\PY{p}{(}\PY{n}{f}\PY{p}{,} \PY{n}{linear}\PY{p}{)}
        \PY{n}{p\PYZus{}chebyshev} \PY{o}{=} \PY{n}{ptuples}\PY{p}{(}\PY{n}{f}\PY{p}{,} \PY{n}{chebyshev}\PY{p}{)}
        \PY{n}{plt}\PY{o}{.}\PY{n}{plot}\PY{p}{(}\PY{n}{np}\PY{o}{.}\PY{n}{linspace}\PY{p}{(}\PY{o}{\PYZhy{}}\PY{l+m+mi}{1}\PY{p}{,}\PY{l+m+mi}{1}\PY{p}{)}\PY{p}{,}\PY{p}{[}\PY{n}{f}\PY{p}{(}\PY{n}{t}\PY{p}{)} \PY{k}{for} \PY{n}{t} \PY{o+ow}{in} \PY{n}{np}\PY{o}{.}\PY{n}{linspace}\PY{p}{(}\PY{o}{\PYZhy{}}\PY{l+m+mi}{1}\PY{p}{,}\PY{l+m+mi}{1}\PY{p}{)}\PY{p}{]}\PY{p}{)}
        \PY{n}{plt}\PY{o}{.}\PY{n}{plot}\PY{p}{(}\PY{n}{p\PYZus{}linear}\PY{p}{[}\PY{l+m+mi}{0}\PY{p}{]}\PY{p}{,}\PY{n}{p\PYZus{}linear}\PY{p}{[}\PY{l+m+mi}{1}\PY{p}{]}\PY{p}{)}
        \PY{n}{plt}\PY{o}{.}\PY{n}{plot}\PY{p}{(}\PY{n}{p\PYZus{}chebyshev}\PY{p}{[}\PY{l+m+mi}{0}\PY{p}{]}\PY{p}{,}\PY{n}{p\PYZus{}chebyshev}\PY{p}{[}\PY{l+m+mi}{1}\PY{p}{]}\PY{p}{)}
        \PY{n}{plt}\PY{o}{.}\PY{n}{ylim}\PY{p}{(}\PY{o}{\PYZhy{}}\PY{l+m+mi}{2}\PY{p}{,}\PY{l+m+mi}{2}\PY{p}{)}
        \PY{n}{plt}\PY{o}{.}\PY{n}{title}\PY{p}{(}\PY{n}{f}\PY{o}{.}\PY{n+nv+vm}{\PYZus{}\PYZus{}name\PYZus{}\PYZus{}} \PY{o}{+} \PY{l+s+s1}{\PYZsq{}}\PY{l+s+s1}{ approximated by different \PYZdl{}p\PYZus{}}\PY{l+s+s1}{\PYZob{}}\PY{l+s+si}{\PYZpc{}d}\PY{l+s+s1}{\PYZcb{}\PYZdl{}}\PY{l+s+s1}{\PYZsq{}} \PY{o}{\PYZpc{}} \PY{n}{n}\PY{p}{)}
        \PY{n}{plt}\PY{o}{.}\PY{n}{legend}\PY{p}{(}\PY{p}{(}\PY{n}{f}\PY{o}{.}\PY{n+nv+vm}{\PYZus{}\PYZus{}name\PYZus{}\PYZus{}}\PY{p}{,}\PY{l+s+s1}{\PYZsq{}}\PY{l+s+s1}{Linear\PYZhy{}spaced \PYZdl{}p\PYZus{}}\PY{l+s+s1}{\PYZob{}}\PY{l+s+si}{\PYZpc{}d}\PY{l+s+s1}{\PYZcb{}\PYZdl{}}\PY{l+s+s1}{\PYZsq{}} \PY{o}{\PYZpc{}} \PY{n}{n}\PY{p}{,}\PY{l+s+s1}{\PYZsq{}}\PY{l+s+s1}{Chebyshev \PYZdl{}p\PYZus{}}\PY{l+s+s1}{\PYZob{}}\PY{l+s+si}{\PYZpc{}d}\PY{l+s+s1}{\PYZcb{}\PYZdl{}}\PY{l+s+s1}{\PYZsq{}} \PY{o}{\PYZpc{}} \PY{n}{n}\PY{p}{)}\PY{p}{)}
        \PY{n}{plt}\PY{o}{.}\PY{n}{show}\PY{p}{(}\PY{p}{)}
\end{InVerbatim}

    \begin{center}
    \adjustimage{max size={0.6\linewidth}{0.6\paperheight}}{2018-07-19-interpolation_files/2018-07-19-interpolation_50_0.png}
    \end{center}
    { \hspace*{\fill} \\}
    
    \begin{center}
    \adjustimage{max size={0.6\linewidth}{0.6\paperheight}}{2018-07-19-interpolation_files/2018-07-19-interpolation_50_1.png}
    \end{center}
    { \hspace*{\fill} \\}
    
    \begin{center}
    \adjustimage{max size={0.6\linewidth}{0.6\paperheight}}{2018-07-19-interpolation_files/2018-07-19-interpolation_50_2.png}
    \end{center}
    { \hspace*{\fill} \\}
    
    \begin{center}
    \adjustimage{max size={0.6\linewidth}{0.6\paperheight}}{2018-07-19-interpolation_files/2018-07-19-interpolation_50_3.png}
    \end{center}
    { \hspace*{\fill} \\}
    
    \begin{center}
    \adjustimage{max size={0.6\linewidth}{0.6\paperheight}}{2018-07-19-interpolation_files/2018-07-19-interpolation_50_4.png}
    \end{center}
    { \hspace*{\fill} \\}
    
    \begin{center}
    \adjustimage{max size={0.6\linewidth}{0.6\paperheight}}{2018-07-19-interpolation_files/2018-07-19-interpolation_50_5.png}
    \end{center}
    { \hspace*{\fill} \\}
    
    Each method for choosing sample points naturally corresponds with a
power series representation of the function to be interpolated.

That is, there's a bijection between the sequence of functions
\(c_n:\mathbf{N} \to \mathbf{R}^n\) (used to choose sample points
\(\{x_i : i = 0,1,\ldots, n\}\)) and the partial sums of the
power-series \(\sum_{k=0}^{n}a_k \prod_{i< k}(x-x_i)\) with the
coefficients \(a_n\) generated by divided differences. Call this
power-series \(p_\infty\).

Looking at the above plots, and applying a bit of mind's eye
real-analytic intuition, I suggest.

\begin{itemize}
\item
  When \(p_n\) is generated by Chebyshev points

  \begin{itemize}
  \tightlist
  \item
    \(p_\infty(x) = f_1(x)\) for all \(x\) in \([-1,1]\)
  \item
    \(p_\infty(x) = f_2(x)\) for all \(x\) in \([-1,1]\setminus \{0\}\)
  \end{itemize}
\item
  When \(p_n\) is generated by linearly spaced points

  \begin{itemize}
  \tightlist
  \item
    \(p_\infty(x) = f_1(x)\) at countably many sample points, else
    nowhere.
  \item
    \(p_\infty(x) = f_2(x)\) at countably many sample points, else
    nowhere.
  \end{itemize}
\end{itemize}

In other words, Chebyshev sample points seem to produce sequences of
polynomials that converge uniformly to the function they interpolate
(provided that function is continuous), whereas linearly spaced sample
points (at least on compact sets \([a,b]\) where
\(\lvert b-a \rvert > 1\)) produce sequences of polynomials that don't
even converge pointwise to their interpolating functions.

\hypertarget{hermite-interpolation}{%
\subsubsection{Hermite interpolation}\label{hermite-interpolation}}

Hermite interpolation can be setup in an analogous fashion to (plain)
interpolation: we just obtain \(2n+2\) equations from \(n+1\) function
and \(n+1\) derivative evaluations at the sample points.

Here's the Lagrange form implementation:

    
\prompt{In}{incolor}{14}
\begin{InVerbatim}[fontsize=\small,commandchars=\\\{\}]
\PY{k}{def} \PY{n+nf}{hermite\PYZus{}basis}\PY{p}{(}\PY{n}{vecx}\PY{p}{)}\PY{p}{:}
    \PY{n}{lp} \PY{o}{=} \PY{n}{lagrange\PYZus{}basis}\PY{p}{(}\PY{n}{vecx}\PY{p}{)}
    \PY{c+c1}{\PYZsh{} appealing to sympy for differentiation}
    \PY{c+c1}{\PYZsh{} TODO implement numerically}
    \PY{n}{dp} \PY{o}{=} \PY{p}{[}\PY{n}{sym}\PY{o}{.}\PY{n}{diff}\PY{p}{(}\PY{n}{l}\PY{p}{(}\PY{n}{sym}\PY{o}{.}\PY{n}{Symbol}\PY{p}{(}\PY{l+s+s1}{\PYZsq{}}\PY{l+s+s1}{x}\PY{l+s+s1}{\PYZsq{}}\PY{p}{)}\PY{p}{)}\PY{p}{)} \PY{k}{for} \PY{n}{l} \PY{o+ow}{in} \PY{n}{lp}\PY{p}{]}
    \PY{n}{hf} \PY{o}{=} \PY{p}{[}\PY{k}{lambda} \PY{n}{t}\PY{p}{,} \PY{n}{i}\PY{o}{=}\PY{n}{i}\PY{p}{:}\PYZbs{}
            \PY{p}{(}\PY{l+m+mi}{1}\PY{o}{\PYZhy{}}\PY{l+m+mi}{2}\PY{o}{*}\PY{n}{dp}\PY{p}{[}\PY{n}{i}\PY{p}{]}\PY{o}{.}\PY{n}{subs}\PY{p}{(}\PY{n}{x}\PY{p}{,}\PY{n}{vecx}\PY{p}{[}\PY{n}{i}\PY{p}{]}\PY{p}{)}\PY{o}{*}\PY{p}{(}\PY{n}{t}\PY{o}{\PYZhy{}}\PY{n}{vecx}\PY{p}{[}\PY{n}{i}\PY{p}{]}\PY{p}{)}\PY{p}{)}\PY{o}{*}\PY{n}{lp}\PY{p}{[}\PY{n}{i}\PY{p}{]}\PY{p}{(}\PY{n}{t}\PY{p}{)}\PY{o}{*}\PY{o}{*}\PY{l+m+mi}{2}\PYZbs{}
            \PY{k}{for} \PY{n}{i} \PY{o+ow}{in} \PY{n+nb}{range}\PY{p}{(}\PY{n+nb}{len}\PY{p}{(}\PY{n}{vecx}\PY{p}{)}\PY{p}{)}\PY{p}{]}
    \PY{n}{hh} \PY{o}{=} \PY{p}{[}\PY{k}{lambda} \PY{n}{t}\PY{p}{,} \PY{n}{i}\PY{o}{=}\PY{n}{i}\PY{p}{:}\PYZbs{}
            \PY{p}{(}\PY{n}{t}\PY{o}{\PYZhy{}}\PY{n}{vecx}\PY{p}{[}\PY{n}{i}\PY{p}{]}\PY{p}{)}\PY{o}{*}\PY{p}{(}\PY{n}{lp}\PY{p}{[}\PY{n}{i}\PY{p}{]}\PY{p}{(}\PY{n}{t}\PY{p}{)}\PY{p}{)}\PY{o}{*}\PY{o}{*}\PY{l+m+mi}{2}\PYZbs{}
            \PY{k}{for} \PY{n}{i} \PY{o+ow}{in} \PY{n+nb}{range}\PY{p}{(}\PY{n+nb}{len}\PY{p}{(}\PY{n}{vecx}\PY{p}{)}\PY{p}{)}\PY{p}{]}
    \PY{k}{return} \PY{p}{[}\PY{n}{hf}\PY{p}{,}\PY{n}{hh}\PY{p}{]}

\PY{k}{def} \PY{n+nf}{lagherm\PYZus{}coords}\PY{p}{(}\PY{n}{f}\PY{p}{,}\PY{n}{vecx}\PY{p}{)}\PY{p}{:}
    \PY{n}{vecf} \PY{o}{=} \PY{p}{[}\PY{n}{f}\PY{p}{(}\PY{n}{vecx}\PY{p}{[}\PY{n}{i}\PY{p}{]}\PY{p}{)} \PY{k}{for} \PY{n}{i} \PY{o+ow}{in} \PY{n+nb}{range}\PY{p}{(}\PY{n+nb}{len}\PY{p}{(}\PY{n}{vecx}\PY{p}{)}\PY{p}{)}\PY{p}{]}
    \PY{n}{df} \PY{o}{=} \PY{k}{lambda} \PY{n}{t}\PY{p}{:} \PY{n}{f}\PY{p}{(}\PY{n}{sym}\PY{o}{.}\PY{n}{Symbol}\PY{p}{(}\PY{l+s+s1}{\PYZsq{}}\PY{l+s+s1}{x}\PY{l+s+s1}{\PYZsq{}}\PY{p}{)}\PY{p}{)}\PY{o}{.}\PY{n}{diff}\PY{p}{(}\PY{n}{x}\PY{p}{)}\PY{o}{.}\PY{n}{subs}\PY{p}{(}\PY{n}{x}\PY{p}{,}\PY{n}{t}\PY{p}{)}
    \PY{n}{vecdf} \PY{o}{=} \PY{p}{[}\PY{n}{df}\PY{p}{(}\PY{n}{vecx}\PY{p}{[}\PY{n}{i}\PY{p}{]}\PY{p}{)} \PY{k}{for} \PY{n}{i} \PY{o+ow}{in} \PY{n+nb}{range}\PY{p}{(}\PY{n+nb}{len}\PY{p}{(}\PY{n}{vecx}\PY{p}{)}\PY{p}{)}\PY{p}{]}
    \PY{k}{return} \PY{p}{[}\PY{n}{np}\PY{o}{.}\PY{n}{array}\PY{p}{(}\PY{n}{vecf}\PY{p}{)}\PY{p}{,} \PY{n}{np}\PY{o}{.}\PY{n}{array}\PY{p}{(}\PY{n}{vecdf}\PY{p}{)}\PY{p}{]}

\PY{c+c1}{\PYZsh{} TODO rewrite hermite\PYZus{}basis \PYZam{} lagherm\PYZus{}coords to}
\PY{c+c1}{\PYZsh{} avoid flattening their outputs}
\PY{n}{flatten} \PY{o}{=} \PY{k}{lambda} \PY{n}{l}\PY{p}{:} \PY{p}{[}\PY{n}{item} \PY{k}{for} \PY{n}{sublist} \PY{o+ow}{in} \PY{n}{l} \PY{k}{for} \PY{n}{item} \PY{o+ow}{in} \PY{n}{sublist}\PY{p}{]}

\PY{k}{def} \PY{n+nf}{lagherm\PYZus{}interpol}\PY{p}{(}\PY{n}{lagherm\PYZus{}coords}\PY{p}{,} \PY{n}{hermite\PYZus{}basis}\PY{p}{,} \PY{n}{t}\PY{p}{)}\PY{p}{:}
    \PY{n}{coords} \PY{o}{=} \PY{n}{flatten}\PY{p}{(}\PY{n}{lagherm\PYZus{}coords}\PY{p}{)}
    \PY{n}{polys} \PY{o}{=} \PY{n}{flatten}\PY{p}{(}\PY{n}{hermite\PYZus{}basis}\PY{p}{)}
    \PY{n}{scale} \PY{o}{=} \PY{p}{[}\PY{k}{lambda} \PY{n}{p}\PY{p}{,} \PY{n}{a}\PY{o}{=}\PY{n}{a}\PY{p}{:} \PY{n}{p} \PY{o}{*} \PY{n}{a} \PY{k}{for} \PY{n}{a} \PY{o+ow}{in} \PY{n}{coords}\PY{p}{]}
    \PY{k}{return} \PY{n+nb}{sum}\PY{p}{(}\PY{p}{[}\PY{n}{scale}\PY{p}{[}\PY{n}{i}\PY{p}{]}\PY{p}{(}\PY{n}{polys}\PY{p}{[}\PY{n}{i}\PY{p}{]}\PY{p}{(}\PY{n}{t}\PY{p}{)}\PY{p}{)} \PY{k}{for} \PY{n}{i} \PY{o+ow}{in} \PY{n+nb}{range}\PY{p}{(}\PY{n+nb}{len}\PY{p}{(}\PY{n}{polys}\PY{p}{)}\PY{p}{)}\PY{p}{]}\PY{p}{)}
\end{InVerbatim}

    Consider the specific case \(f(x) = x \ln x\), with sample points
\((x_0, x_1) = (1, 3)\).

    
\prompt{In}{incolor}{15}
\begin{InVerbatim}[fontsize=\small,commandchars=\\\{\}]
\PY{n}{x} \PY{o}{=} \PY{n}{sym}\PY{o}{.}\PY{n}{Symbol}\PY{p}{(}\PY{l+s+s1}{\PYZsq{}}\PY{l+s+s1}{x}\PY{l+s+s1}{\PYZsq{}}\PY{p}{)}
\PY{n}{f} \PY{o}{=} \PY{k}{lambda} \PY{n}{x}\PY{p}{:} \PY{n}{x}\PY{o}{*}\PY{n}{sym}\PY{o}{.}\PY{n}{log}\PY{p}{(}\PY{n}{x}\PY{p}{)}
\PY{n}{vecx} \PY{o}{=} \PY{n}{np}\PY{o}{.}\PY{n}{array}\PY{p}{(}\PY{p}{[}\PY{l+m+mi}{1}\PY{p}{,}\PY{l+m+mi}{3}\PY{p}{]}\PY{p}{)}

\PY{k}{def} \PY{n+nf}{prob8plots}\PY{p}{(}\PY{p}{)}\PY{p}{:}
    \PY{c+c1}{\PYZsh{} we define the osculatory cubic polynomial}
    \PY{n}{coordinates} \PY{o}{=} \PY{n}{lagherm\PYZus{}coords}\PY{p}{(}\PY{n}{f}\PY{p}{,}\PY{n}{vecx}\PY{p}{)}
    \PY{n}{basis} \PY{o}{=} \PY{n}{hermite\PYZus{}basis}\PY{p}{(}\PY{n}{vecx}\PY{p}{)}
    \PY{n}{H} \PY{o}{=} \PY{k}{lambda} \PY{n}{t}\PY{p}{:} \PY{n}{lagherm\PYZus{}interpol}\PY{p}{(}\PY{n}{coordinates}\PY{p}{,} \PY{n}{basis}\PY{p}{,} \PY{n}{t}\PY{p}{)}

    \PY{c+c1}{\PYZsh{} and generate stiples for plotting}
    \PY{n}{stiple} \PY{o}{=} \PY{n}{np}\PY{o}{.}\PY{n}{linspace}\PY{p}{(}\PY{l+m+mi}{1}\PY{p}{,}\PY{l+m+mi}{3}\PY{p}{)}
    \PY{n}{Hstiple} \PY{o}{=} \PY{p}{[}\PY{n}{H}\PY{p}{(}\PY{n}{t}\PY{p}{)} \PY{k}{for} \PY{n}{t} \PY{o+ow}{in} \PY{n}{stiple}\PY{p}{]}
    \PY{n}{plt}\PY{o}{.}\PY{n}{plot}\PY{p}{(}\PY{n}{stiple}\PY{p}{,}\PY{n}{Hstiple}\PY{p}{)}

    \PY{c+c1}{\PYZsh{} for comparison, here\PYZsq{}s the linear interpolant}
    \PY{n}{m} \PY{o}{=} \PY{p}{(}\PY{n}{f}\PY{p}{(}\PY{l+m+mi}{3}\PY{p}{)} \PY{o}{\PYZhy{}} \PY{n}{f}\PY{p}{(}\PY{l+m+mi}{1}\PY{p}{)}\PY{p}{)}\PY{o}{/}\PY{l+m+mi}{2}
    \PY{n}{line} \PY{o}{=} \PY{k}{lambda} \PY{n}{x}\PY{p}{:} \PY{n}{m}\PY{o}{*}\PY{p}{(}\PY{n}{x}\PY{o}{\PYZhy{}}\PY{l+m+mi}{1}\PY{p}{)} \PY{o}{+} \PY{n}{f}\PY{p}{(}\PY{l+m+mi}{1}\PY{p}{)}
    \PY{n}{lstiple} \PY{o}{=} \PY{n}{line}\PY{p}{(}\PY{n}{stiple}\PY{p}{)}
    \PY{n}{plt}\PY{o}{.}\PY{n}{plot}\PY{p}{(}\PY{n}{stiple}\PY{p}{,}\PY{n}{lstiple}\PY{p}{)}

    \PY{c+c1}{\PYZsh{} and the exact graph}
    \PY{n}{fstiple} \PY{o}{=} \PY{p}{[}\PY{n}{f}\PY{p}{(}\PY{n}{t}\PY{p}{)} \PY{k}{for} \PY{n}{t} \PY{o+ow}{in} \PY{n}{stiple}\PY{p}{]}
    \PY{n}{plt}\PY{o}{.}\PY{n}{plot}\PY{p}{(}\PY{n}{stiple}\PY{p}{,}\PY{n}{fstiple}\PY{p}{)}
    \PY{n}{plt}\PY{o}{.}\PY{n}{title}\PY{p}{(}\PY{l+s+s1}{\PYZsq{}}\PY{l+s+s1}{Interpolation of }\PY{l+s+s1}{\PYZsq{}}\PY{o}{+}\PY{n+nb}{str}\PY{p}{(}\PY{n}{f}\PY{p}{(}\PY{n}{sym}\PY{o}{.}\PY{n}{Symbol}\PY{p}{(}\PY{l+s+s1}{\PYZsq{}}\PY{l+s+s1}{x}\PY{l+s+s1}{\PYZsq{}}\PY{p}{)}\PY{p}{)}\PY{p}{)}\PY{p}{)}
    \PY{n}{plt}\PY{o}{.}\PY{n}{legend}\PY{p}{(}\PY{p}{(}\PY{l+s+s1}{\PYZsq{}}\PY{l+s+s1}{Hermite Cubic}\PY{l+s+s1}{\PYZsq{}}\PY{p}{,}\PY{l+s+s1}{\PYZsq{}}\PY{l+s+s1}{Linear}\PY{l+s+s1}{\PYZsq{}}\PY{p}{,} \PY{l+s+s1}{\PYZsq{}}\PY{l+s+s1}{Exact}\PY{l+s+s1}{\PYZsq{}}\PY{p}{)}\PY{p}{)}
    \PY{n}{plt}\PY{o}{.}\PY{n}{show}\PY{p}{(}\PY{p}{)}

\PY{n}{prob8plots}\PY{p}{(}\PY{p}{)}
\end{InVerbatim}

    \begin{center}
    \adjustimage{max size={0.6\linewidth}{0.6\paperheight}}{2018-07-19-interpolation_files/2018-07-19-interpolation_54_0.png}
    \end{center}
    { \hspace*{\fill} \\}
    
    The approximation is so close, we don't get a chance to build any
intuition around how stiff a cubic polynomial really is---stiff. We'll
interpolate two other analytic functions on \([1,3]\) to demonstrate
this effect. Notably \(f(x) = \tan(x/2)\) is singular about \(pi\), so
the osculatory polynomial will really have to \emph{strech} to match
\(f'(3)\).

    
\prompt{In}{incolor}{16}
\begin{InVerbatim}[fontsize=\small,commandchars=\\\{\}]
\PY{n}{f} \PY{o}{=} \PY{k}{lambda} \PY{n}{x}\PY{p}{:} \PY{n}{sym}\PY{o}{.}\PY{n}{sin}\PY{p}{(}\PY{n}{x}\PY{p}{)}
\PY{n}{prob8plots}\PY{p}{(}\PY{p}{)}
\end{InVerbatim}

    \begin{center}
    \adjustimage{max size={0.6\linewidth}{0.6\paperheight}}{2018-07-19-interpolation_files/2018-07-19-interpolation_56_0.png}
    \end{center}
    { \hspace*{\fill} \\}
    
    
\prompt{In}{incolor}{17}
\begin{InVerbatim}[fontsize=\small,commandchars=\\\{\}]
\PY{n}{f} \PY{o}{=} \PY{k}{lambda} \PY{n}{x}\PY{p}{:} \PY{n}{sym}\PY{o}{.}\PY{n}{tan}\PY{p}{(}\PY{n}{x}\PY{o}{/}\PY{l+m+mi}{2}\PY{p}{)}
\PY{n}{prob8plots}\PY{p}{(}\PY{p}{)}
\end{InVerbatim}

    \begin{center}
    \adjustimage{max size={0.6\linewidth}{0.6\paperheight}}{2018-07-19-interpolation_files/2018-07-19-interpolation_57_0.png}
    \end{center}
    { \hspace*{\fill} \\}
    
    Returning to \(f(x) = x \ln x\), as desired, we compare the exact value,
the Hermite interpolant, and the linear interpolant at \(x =1.5\).

    
\prompt{In}{incolor}{18}
\begin{InVerbatim}[fontsize=\small,commandchars=\\\{\}]
\PY{n}{f} \PY{o}{=} \PY{k}{lambda} \PY{n}{x}\PY{p}{:} \PY{n}{x}\PY{o}{*}\PY{n}{sym}\PY{o}{.}\PY{n}{log}\PY{p}{(}\PY{n}{x}\PY{p}{)}
\PY{c+c1}{\PYZsh{} reset mutated Hermite poly back to xlnx}
\PY{c+c1}{\PYZsh{} TODO get these assignments out of the global namespace}
\PY{n}{basis} \PY{o}{=} \PY{n}{hermite\PYZus{}basis}\PY{p}{(}\PY{n}{vecx}\PY{p}{)}
\PY{n}{coordinates} \PY{o}{=} \PY{n}{lagherm\PYZus{}coords}\PY{p}{(}\PY{n}{f}\PY{p}{,}\PY{n}{vecx}\PY{p}{)}
\PY{n}{H} \PY{o}{=} \PY{k}{lambda} \PY{n}{t}\PY{p}{:} \PY{n}{lagherm\PYZus{}interpol}\PY{p}{(}\PY{n}{coordinates}\PY{p}{,} \PY{n}{basis}\PY{p}{,} \PY{n}{t}\PY{p}{)}
\PY{n}{m} \PY{o}{=} \PY{p}{(}\PY{n}{f}\PY{p}{(}\PY{l+m+mi}{3}\PY{p}{)} \PY{o}{\PYZhy{}} \PY{n}{f}\PY{p}{(}\PY{l+m+mi}{1}\PY{p}{)}\PY{p}{)}\PY{o}{/}\PY{l+m+mi}{2}
\PY{n}{line} \PY{o}{=} \PY{k}{lambda} \PY{n}{x}\PY{p}{:} \PY{n}{m}\PY{o}{*}\PY{p}{(}\PY{n}{x}\PY{o}{\PYZhy{}}\PY{l+m+mi}{1}\PY{p}{)} \PY{o}{+} \PY{n}{f}\PY{p}{(}\PY{l+m+mi}{1}\PY{p}{)}

\PY{n+nb}{print}\PY{p}{(}\PY{n}{f}\PY{p}{(}\PY{l+m+mf}{1.5}\PY{p}{)}\PY{o}{.}\PY{n}{evalf}\PY{p}{(}\PY{p}{)}\PY{p}{)}
\PY{n+nb}{print}\PY{p}{(}\PY{n}{H}\PY{p}{(}\PY{l+m+mf}{1.5}\PY{p}{)}\PY{o}{.}\PY{n}{evalf}\PY{p}{(}\PY{p}{)}\PY{p}{)}
\PY{n+nb}{print}\PY{p}{(}\PY{n}{line}\PY{p}{(}\PY{l+m+mf}{1.5}\PY{p}{)}\PY{o}{.}\PY{n}{evalf}\PY{p}{(}\PY{p}{)}\PY{p}{)}
\end{InVerbatim}

    \begin{Verbatim}[fontsize=\small,commandchars=\\\{\}]
0.608197662162247
0.599479608250541
0.823959216501082

    \end{Verbatim}

    Now we'll compute the error bound.

    
\prompt{In}{incolor}{19}
\begin{InVerbatim}[fontsize=\small,commandchars=\\\{\}]
\PY{k}{def} \PY{n+nf}{hermite\PYZus{}error\PYZus{}bound}\PY{p}{(}\PY{n}{f}\PY{p}{,} \PY{n}{vecx}\PY{p}{)}\PY{p}{:}
    \PY{n}{w} \PY{o}{=} \PY{k}{lambda} \PY{n}{x}\PY{p}{:} \PY{n}{np}\PY{o}{.}\PY{n}{prod}\PY{p}{(}\PY{p}{[}\PY{p}{(}\PY{n}{x}\PY{o}{\PYZhy{}}\PY{n}{i}\PY{p}{)}\PY{o}{*}\PY{o}{*}\PY{l+m+mi}{2} \PY{k}{for} \PY{n}{i} \PY{o+ow}{in} \PY{n}{vecx}\PY{p}{]}\PY{p}{)}
    \PY{c+c1}{\PYZsh{} extreme values of cts diff\PYZsq{}ble func on compact set}
    \PY{n}{extrema} \PY{o}{=} \PY{n}{sym}\PY{o}{.}\PY{n}{solve}\PY{p}{(}\PY{n}{sym}\PY{o}{.}\PY{n}{diff}\PY{p}{(}\PY{n}{w}\PY{p}{(}\PY{n}{sym}\PY{o}{.}\PY{n}{Symbol}\PY{p}{(}\PY{l+s+s1}{\PYZsq{}}\PY{l+s+s1}{x}\PY{l+s+s1}{\PYZsq{}}\PY{p}{)}\PY{p}{)}\PY{p}{,}\PY{n}{x}\PY{p}{)}\PY{p}{,}\PY{n}{x}\PY{p}{)}
    \PY{n}{worst\PYZus{}wiggle} \PY{o}{=} \PY{n+nb}{max}\PY{p}{(}\PY{p}{[}\PY{n+nb}{abs}\PY{p}{(}\PY{n}{w}\PY{p}{(}\PY{n}{x}\PY{p}{)}\PY{p}{)} \PY{k}{for} \PY{n}{x} \PY{o+ow}{in} \PY{n}{extrema}\PY{p}{]}\PY{p}{)}
    \PY{n}{jerk} \PY{o}{=} \PY{n}{sym}\PY{o}{.}\PY{n}{diff}\PY{p}{(}\PY{n}{f}\PY{p}{(}\PY{n}{sym}\PY{o}{.}\PY{n}{Symbol}\PY{p}{(}\PY{l+s+s1}{\PYZsq{}}\PY{l+s+s1}{x}\PY{l+s+s1}{\PYZsq{}}\PY{p}{)}\PY{p}{)}\PY{p}{,}\PY{n}{x}\PY{p}{,}\PY{l+m+mi}{2}\PY{o}{*}\PY{n+nb}{len}\PY{p}{(}\PY{n}{vecx}\PY{p}{)}\PY{p}{)}
    \PY{c+c1}{\PYZsh{} assume \PYZdl{}f\PYZca{}\PYZob{}(2n+n)\PYZcb{}\PYZdl{} is monotone and find extrema at endpoints}
    \PY{n}{worst\PYZus{}jerk} \PY{o}{=} \PY{n+nb}{max}\PY{p}{(}\PY{p}{[}\PY{n}{jerk}\PY{o}{.}\PY{n}{subs}\PY{p}{(}\PY{n}{x}\PY{p}{,}\PY{n}{i}\PY{p}{)} \PY{k}{for} \PY{n}{i} \PY{o+ow}{in} \PY{n}{vecx}\PY{p}{]}\PY{p}{)}
    \PY{k}{return} \PY{p}{(}\PY{n+nb}{abs}\PY{p}{(}\PY{n}{worst\PYZus{}jerk}\PY{p}{)}\PY{o}{*}\PY{n}{worst\PYZus{}wiggle}\PY{p}{)}\PY{o}{/}\PY{n}{sym}\PY{o}{.}\PY{n}{factorial}\PY{p}{(}\PY{l+m+mi}{2}\PY{o}{*}\PY{n+nb}{len}\PY{p}{(}\PY{n}{vecx}\PY{p}{)}\PY{p}{)}

\PY{n}{hermite\PYZus{}error\PYZus{}bound}\PY{p}{(}\PY{n}{f}\PY{p}{,} \PY{n}{vecx}\PY{p}{)}\PY{o}{.}\PY{n}{evalf}\PY{p}{(}\PY{p}{)}
\end{InVerbatim}

            
\prompt{Out}{outcolor}{19}
\begin{OutVerbatim}[fontsize=\small,commandchars=\\\{\}]
0.0833333333333333
\end{OutVerbatim}
        
    To verify, consider the function given by distance at \(x\) between
\(f\) and its Hermite interpolant. To find its extrema, we'll solve
\((f - H)'(x) =0\).

    
\prompt{In}{incolor}{20}
\begin{InVerbatim}[fontsize=\small,commandchars=\\\{\}]
\PY{c+c1}{\PYZsh{} we define the desired fixed point iterator}
\PY{n}{err} \PY{o}{=} \PY{n}{f}\PY{p}{(}\PY{n}{sym}\PY{o}{.}\PY{n}{Symbol}\PY{p}{(}\PY{l+s+s1}{\PYZsq{}}\PY{l+s+s1}{x}\PY{l+s+s1}{\PYZsq{}}\PY{p}{)}\PY{p}{)} \PY{o}{\PYZhy{}} \PY{n}{H}\PY{p}{(}\PY{n}{sym}\PY{o}{.}\PY{n}{Symbol}\PY{p}{(}\PY{l+s+s1}{\PYZsq{}}\PY{l+s+s1}{x}\PY{l+s+s1}{\PYZsq{}}\PY{p}{)}\PY{p}{)}
\PY{n}{g} \PY{o}{=} \PY{k}{lambda} \PY{n}{t}\PY{p}{:} \PY{n}{t} \PY{o}{\PYZhy{}} \PY{p}{(}\PY{n}{sym}\PY{o}{.}\PY{n}{diff}\PY{p}{(}\PY{n}{err}\PY{p}{,}\PY{n}{x}\PY{p}{)}\PY{o}{/}\PY{n}{sym}\PY{o}{.}\PY{n}{diff}\PY{p}{(}\PY{n}{err}\PY{p}{,}\PY{n}{x}\PY{p}{,}\PY{l+m+mi}{2}\PY{p}{)}\PY{p}{)}\PY{o}{.}\PY{n}{subs}\PY{p}{(}\PY{n}{x}\PY{p}{,}\PY{n}{t}\PY{p}{)}

\PY{c+c1}{\PYZsh{} recall a fixed point iterating algorithm}
\PY{k}{def} \PY{n+nf}{fixedpt}\PY{p}{(}\PY{n}{g}\PY{p}{,} \PY{n}{x0}\PY{p}{,} \PY{n}{eps}\PY{p}{,} \PY{n}{Nmax}\PY{p}{)}\PY{p}{:}
    \PY{n}{i}\PY{o}{=}\PY{l+m+mi}{0}\PY{p}{;} \PY{n}{seq} \PY{o}{=} \PY{p}{[}\PY{n}{x0}\PY{p}{]}
    \PY{n}{x1} \PY{o}{=} \PY{n}{g}\PY{p}{(}\PY{n}{x0}\PY{p}{)}\PY{o}{.}\PY{n}{evalf}\PY{p}{(}\PY{p}{)}
    \PY{k}{while} \PY{n+nb}{abs}\PY{p}{(}\PY{n}{x1} \PY{o}{\PYZhy{}} \PY{n}{x0}\PY{p}{)} \PY{o}{\PYZgt{}}\PY{o}{=} \PY{n}{eps} \PY{o}{*} \PY{n+nb}{abs}\PY{p}{(}\PY{n}{x1}\PY{p}{)} \PY{o+ow}{and} \PY{n}{i} \PY{o}{\PYZlt{}} \PY{n}{Nmax}\PY{p}{:}
        \PY{n}{x0} \PY{o}{=} \PY{n}{x1}
        \PY{n}{x1} \PY{o}{=} \PY{n}{g}\PY{p}{(}\PY{n}{x1}\PY{p}{)}\PY{o}{.}\PY{n}{evalf}\PY{p}{(}\PY{p}{)}
        \PY{n}{i} \PY{o}{+}\PY{o}{=} \PY{l+m+mi}{1}
        \PY{n}{seq} \PY{o}{=} \PY{n}{seq} \PY{o}{+} \PY{p}{[}\PY{n}{x1}\PY{p}{]}
    \PY{k}{return} \PY{n}{seq}

\PY{c+c1}{\PYZsh{} and search for roots with parameters}
\PY{n}{eps} \PY{o}{=} \PY{l+m+mi}{10}\PY{o}{*}\PY{o}{*}\PY{p}{(}\PY{o}{\PYZhy{}}\PY{l+m+mi}{8}\PY{p}{)}\PY{p}{;} \PY{n}{Nmax} \PY{o}{=} \PY{l+m+mi}{10}

\PY{c+c1}{\PYZsh{} I start at 1.1, 2.0, and 2.9, just guessing the cubic polynomial is}
\PY{c+c1}{\PYZsh{} crossing the function 3 times...}

\PY{p}{[}\PY{n}{fixedpt}\PY{p}{(}\PY{n}{g}\PY{p}{,} \PY{n}{x0}\PY{p}{,} \PY{n}{eps}\PY{p}{,} \PY{n}{Nmax}\PY{p}{)}\PY{p}{[}\PY{o}{\PYZhy{}}\PY{l+m+mi}{1}\PY{p}{]} \PY{k}{for} \PY{n}{x0} \PY{o+ow}{in} \PY{p}{[}\PY{l+m+mf}{1.1}\PY{p}{,}\PY{l+m+mi}{2}\PY{p}{,}\PY{l+m+mf}{2.9}\PY{p}{]}\PY{p}{]}
\end{InVerbatim}

            
\prompt{Out}{outcolor}{20}
\begin{OutVerbatim}[fontsize=\small,commandchars=\\\{\}]
[0.999999999999996, 1.91984231854365, 3.00000000000000]
\end{OutVerbatim}
        
    Good, the endpoints should be minima for the absolute error (the
polynomial \emph{was} constructed to match the function at \(1\) and
\(3\)). A continuous function on a compact set has alternating extrema,
so we should have maximal absolute error around \texttt{1.92}. That
error is:

    
\prompt{In}{incolor}{21}
\begin{InVerbatim}[fontsize=\small,commandchars=\\\{\}]
\PY{n}{err}\PY{o}{.}\PY{n}{subs}\PY{p}{(}\PY{n}{x}\PY{p}{,} \PY{l+m+mf}{1.92}\PY{p}{)}\PY{o}{.}\PY{n}{evalf}\PY{p}{(}\PY{p}{)}
\end{InVerbatim}

            
\prompt{Out}{outcolor}{21}
\begin{OutVerbatim}[fontsize=\small,commandchars=\\\{\}]
0.0131939381401981
\end{OutVerbatim}
        
    Consider the same analysis performed on \(x \mapsto \sin(x)\) or
\(x \mapsto \tan(x/2)\)

    
\prompt{In}{incolor}{22}
\begin{InVerbatim}[fontsize=\small,commandchars=\\\{\}]
\PY{n}{f} \PY{o}{=} \PY{k}{lambda} \PY{n}{x}\PY{p}{:} \PY{n}{x}\PY{o}{*}\PY{n}{sym}\PY{o}{.}\PY{n}{sin}\PY{p}{(}\PY{n}{x}\PY{p}{)}
\PY{n}{coordinates} \PY{o}{=} \PY{n}{lagherm\PYZus{}coords}\PY{p}{(}\PY{n}{f}\PY{p}{,}\PY{n}{vecx}\PY{p}{)}
\PY{n}{H} \PY{o}{=} \PY{k}{lambda} \PY{n}{t}\PY{p}{:} \PY{n}{lagherm\PYZus{}interpol}\PY{p}{(}\PY{n}{coordinates}\PY{p}{,} \PY{n}{basis}\PY{p}{,} \PY{n}{t}\PY{p}{)}
\PY{n+nb}{print}\PY{p}{(}\PY{l+s+s1}{\PYZsq{}}\PY{l+s+s1}{bound: }\PY{l+s+s1}{\PYZsq{}} \PY{o}{+} \PY{n+nb}{str}\PY{p}{(}\PY{n}{hermite\PYZus{}error\PYZus{}bound}\PY{p}{(}\PY{n}{f}\PY{p}{,} \PY{n}{vecx}\PY{p}{)}\PY{o}{.}\PY{n}{evalf}\PY{p}{(}\PY{p}{)}\PY{p}{)}\PY{p}{)}

\PY{n}{err} \PY{o}{=} \PY{n}{f}\PY{p}{(}\PY{n}{sym}\PY{o}{.}\PY{n}{Symbol}\PY{p}{(}\PY{l+s+s1}{\PYZsq{}}\PY{l+s+s1}{x}\PY{l+s+s1}{\PYZsq{}}\PY{p}{)}\PY{p}{)} \PY{o}{\PYZhy{}} \PY{n}{H}\PY{p}{(}\PY{n}{sym}\PY{o}{.}\PY{n}{Symbol}\PY{p}{(}\PY{l+s+s1}{\PYZsq{}}\PY{l+s+s1}{x}\PY{l+s+s1}{\PYZsq{}}\PY{p}{)}\PY{p}{)}
\PY{n}{g} \PY{o}{=} \PY{k}{lambda} \PY{n}{t}\PY{p}{:} \PY{n}{t} \PY{o}{\PYZhy{}} \PY{p}{(}\PY{n}{sym}\PY{o}{.}\PY{n}{diff}\PY{p}{(}\PY{n}{err}\PY{p}{,}\PY{n}{x}\PY{p}{)}\PY{o}{/}\PY{n}{sym}\PY{o}{.}\PY{n}{diff}\PY{p}{(}\PY{n}{err}\PY{p}{,}\PY{n}{x}\PY{p}{,}\PY{l+m+mi}{2}\PY{p}{)}\PY{p}{)}\PY{o}{.}\PY{n}{subs}\PY{p}{(}\PY{n}{x}\PY{p}{,}\PY{n}{t}\PY{p}{)}
\PY{n}{c} \PY{o}{=} \PY{n}{fixedpt}\PY{p}{(}\PY{n}{g}\PY{p}{,} \PY{l+m+mi}{2}\PY{p}{,} \PY{n}{eps}\PY{p}{,} \PY{n}{Nmax}\PY{p}{)}\PY{p}{[}\PY{o}{\PYZhy{}}\PY{l+m+mi}{1}\PY{p}{]}
\PY{n+nb}{print}\PY{p}{(}\PY{l+s+s1}{\PYZsq{}}\PY{l+s+s1}{worst: }\PY{l+s+s1}{\PYZsq{}} \PY{o}{+} \PY{n+nb}{str}\PY{p}{(}\PY{n}{err}\PY{o}{.}\PY{n}{subs}\PY{p}{(}\PY{n}{x}\PY{p}{,}\PY{n}{c}\PY{p}{)}\PY{o}{.}\PY{n}{evalf}\PY{p}{(}\PY{p}{)}\PY{p}{)}\PY{p}{)}
\end{InVerbatim}

    \begin{Verbatim}[fontsize=\small,commandchars=\\\{\}]
bound: 0.182638750440891
worst: 0.134275292380452

    \end{Verbatim}

    
\prompt{In}{incolor}{23}
\begin{InVerbatim}[fontsize=\small,commandchars=\\\{\}]
\PY{n}{f} \PY{o}{=} \PY{k}{lambda} \PY{n}{x}\PY{p}{:} \PY{n}{x}\PY{o}{*}\PY{n}{sym}\PY{o}{.}\PY{n}{tan}\PY{p}{(}\PY{n}{x}\PY{o}{/}\PY{l+m+mi}{2}\PY{p}{)}
\PY{n}{coordinates} \PY{o}{=} \PY{n}{lagherm\PYZus{}coords}\PY{p}{(}\PY{n}{f}\PY{p}{,}\PY{n}{vecx}\PY{p}{)}
\PY{n}{H} \PY{o}{=} \PY{k}{lambda} \PY{n}{t}\PY{p}{:} \PY{n}{lagherm\PYZus{}interpol}\PY{p}{(}\PY{n}{coordinates}\PY{p}{,} \PY{n}{basis}\PY{p}{,} \PY{n}{t}\PY{p}{)}
\PY{n}{err} \PY{o}{=} \PY{n}{f}\PY{p}{(}\PY{n}{sym}\PY{o}{.}\PY{n}{Symbol}\PY{p}{(}\PY{l+s+s1}{\PYZsq{}}\PY{l+s+s1}{x}\PY{l+s+s1}{\PYZsq{}}\PY{p}{)}\PY{p}{)} \PY{o}{\PYZhy{}} \PY{n}{H}\PY{p}{(}\PY{n}{sym}\PY{o}{.}\PY{n}{Symbol}\PY{p}{(}\PY{l+s+s1}{\PYZsq{}}\PY{l+s+s1}{x}\PY{l+s+s1}{\PYZsq{}}\PY{p}{)}\PY{p}{)}
\PY{n}{g} \PY{o}{=} \PY{k}{lambda} \PY{n}{t}\PY{p}{:} \PY{n}{t} \PY{o}{\PYZhy{}} \PY{p}{(}\PY{n}{sym}\PY{o}{.}\PY{n}{diff}\PY{p}{(}\PY{n}{err}\PY{p}{,}\PY{n}{x}\PY{p}{)}\PY{o}{/}\PY{n}{sym}\PY{o}{.}\PY{n}{diff}\PY{p}{(}\PY{n}{err}\PY{p}{,}\PY{n}{x}\PY{p}{,}\PY{l+m+mi}{2}\PY{p}{)}\PY{p}{)}\PY{o}{.}\PY{n}{subs}\PY{p}{(}\PY{n}{x}\PY{p}{,}\PY{n}{t}\PY{p}{)}

\PY{n+nb}{print}\PY{p}{(}\PY{l+s+s1}{\PYZsq{}}\PY{l+s+s1}{bound: }\PY{l+s+s1}{\PYZsq{}} \PY{o}{+} \PY{n+nb}{str}\PY{p}{(}\PY{n}{hermite\PYZus{}error\PYZus{}bound}\PY{p}{(}\PY{n}{f}\PY{p}{,} \PY{n}{vecx}\PY{p}{)}\PY{o}{.}\PY{n}{evalf}\PY{p}{(}\PY{p}{)}\PY{p}{)}\PY{p}{)}
\end{InVerbatim}

    \begin{Verbatim}[fontsize=\small,commandchars=\\\{\}]
bound: 110401.830454949

    \end{Verbatim}

    
\prompt{In}{incolor}{24}
\begin{InVerbatim}[fontsize=\small,commandchars=\\\{\}]
\PY{c+c1}{\PYZsh{} Since the error changes dramatically when interpolating tangent, finding the}
\PY{c+c1}{\PYZsh{} extrema of the error function is a bit trickier.}
\PY{n}{regimes} \PY{o}{=} \PY{n}{np}\PY{o}{.}\PY{n}{empty}\PY{p}{(}\PY{p}{(}\PY{l+m+mi}{0}\PY{p}{,}\PY{l+m+mi}{3}\PY{p}{)}\PY{p}{,} \PY{n+nb}{float}\PY{p}{)}
\PY{k}{for} \PY{n}{x0} \PY{o+ow}{in} \PY{n}{np}\PY{o}{.}\PY{n}{linspace}\PY{p}{(}\PY{l+m+mi}{1}\PY{p}{,}\PY{l+m+mi}{3}\PY{p}{,}\PY{l+m+mi}{21}\PY{p}{)}\PY{p}{:}
    \PY{n}{c} \PY{o}{=} \PY{n}{fixedpt}\PY{p}{(}\PY{n}{g}\PY{p}{,} \PY{n}{x0}\PY{p}{,} \PY{n}{eps}\PY{p}{,} \PY{n}{Nmax}\PY{p}{)}\PY{p}{[}\PY{o}{\PYZhy{}}\PY{l+m+mi}{1}\PY{p}{]}
    \PY{n}{fc} \PY{o}{=} \PY{n}{err}\PY{o}{.}\PY{n}{subs}\PY{p}{(}\PY{n}{x}\PY{p}{,}\PY{n}{c}\PY{p}{)}\PY{o}{.}\PY{n}{evalf}\PY{p}{(}\PY{p}{)}
    \PY{n}{regimes} \PY{o}{=} \PY{n}{np}\PY{o}{.}\PY{n}{vstack}\PY{p}{(}\PY{p}{[}\PY{n}{regimes}\PY{p}{,}\PYZbs{}
              \PY{n}{np}\PY{o}{.}\PY{n}{array}\PY{p}{(}\PY{p}{[}\PY{n}{x0}\PY{p}{,} \PY{n}{c}\PY{p}{,} \PY{n}{fc}\PY{p}{]}\PY{p}{)}\PY{o}{.}\PY{n}{astype}\PY{p}{(}\PY{n+nb}{float}\PY{p}{)}\PY{p}{]}\PY{p}{)}
\end{InVerbatim}

    
\prompt{In}{incolor}{25}
\begin{InVerbatim}[fontsize=\small,commandchars=\\\{\}]
\PY{n}{plt}\PY{o}{.}\PY{n}{scatter}\PY{p}{(}\PY{n}{regimes}\PY{p}{[}\PY{l+m+mi}{0}\PY{p}{:}\PY{p}{,} \PY{l+m+mi}{0}\PY{p}{]}\PY{p}{,} \PY{n}{regimes}\PY{p}{[}\PY{l+m+mi}{0}\PY{p}{:}\PY{p}{,} \PY{l+m+mi}{1}\PY{p}{]}\PY{p}{)}
\PY{n}{plt}\PY{o}{.}\PY{n}{xlabel}\PY{p}{(}\PY{l+s+s1}{\PYZsq{}}\PY{l+s+s1}{initial \PYZdl{}x\PYZus{}0\PYZdl{} for iteration}\PY{l+s+s1}{\PYZsq{}}\PY{p}{)}
\PY{n}{plt}\PY{o}{.}\PY{n}{ylabel}\PY{p}{(}\PY{l+s+s1}{\PYZsq{}}\PY{l+s+s1}{corresponding fixed point}\PY{l+s+s1}{\PYZsq{}}\PY{p}{)}
\PY{n}{plt}\PY{o}{.}\PY{n}{show}\PY{p}{(}\PY{p}{)}
\end{InVerbatim}

    \begin{center}
    \adjustimage{max size={0.6\linewidth}{0.6\paperheight}}{2018-07-19-interpolation_files/2018-07-19-interpolation_70_0.png}
    \end{center}
    { \hspace*{\fill} \\}
    
    
\prompt{In}{incolor}{26}
\begin{InVerbatim}[fontsize=\small,commandchars=\\\{\}]
\PY{n}{plt}\PY{o}{.}\PY{n}{scatter}\PY{p}{(}\PY{n}{regimes}\PY{p}{[}\PY{l+m+mi}{0}\PY{p}{:}\PY{p}{,} \PY{l+m+mi}{1}\PY{p}{]}\PY{p}{,} \PY{n}{regimes}\PY{p}{[}\PY{l+m+mi}{0}\PY{p}{:}\PY{p}{,} \PY{l+m+mi}{2}\PY{p}{]}\PY{p}{)}
\PY{n}{plt}\PY{o}{.}\PY{n}{xlabel}\PY{p}{(}\PY{l+s+s1}{\PYZsq{}}\PY{l+s+s1}{\PYZdl{}x\PYZdl{}\PYZhy{}values of extrema of error function}\PY{l+s+s1}{\PYZsq{}}\PY{p}{)}
\PY{n}{plt}\PY{o}{.}\PY{n}{ylabel}\PY{p}{(}\PY{l+s+sa}{r}\PY{l+s+s1}{\PYZsq{}}\PY{l+s+s1}{\PYZdl{}}\PY{l+s+s1}{\PYZbs{}}\PY{l+s+s1}{tan(x/2) \PYZhy{} H(x)\PYZdl{}}\PY{l+s+s1}{\PYZsq{}}\PY{p}{)}
\PY{n}{plt}\PY{o}{.}\PY{n}{show}\PY{p}{(}\PY{p}{)}
\end{InVerbatim}

    \begin{center}
    \adjustimage{max size={0.6\linewidth}{0.6\paperheight}}{2018-07-19-interpolation_files/2018-07-19-interpolation_71_0.png}
    \end{center}
    { \hspace*{\fill} \\}
    
    Which is all to say that \(\tan(x/2) - H(x)\) on \([1,3]\) is at max
\(67.3952\), clearly bounded by \(110401.8305\).

\hypertarget{mixed-initial-conditions}{%
\subsection{Mixed initial conditions}\label{mixed-initial-conditions}}

We find the polynomial \(p\) of least degree (i.e., \(\deg p \leq 4\))
satisfying

\begin{align*}
p(1) &= -1\\
p(2) &= 1\\
p'(1) &= 2\\
p'(2) &= -2\\
p''(1) &=0\\
\end{align*}

With Newton's divided difference table, we compute the coeffecients
\(-1,2,0,0,-4\), and write \(p(x) = -1 + 2(x-1) + 4(x-1)^3(x-2)\).
Verifying,

    
\prompt{In}{incolor}{32}
\begin{InVerbatim}[fontsize=\small,commandchars=\\\{\}]
\PY{n}{p} \PY{o}{=} \PY{k}{lambda} \PY{n}{x}\PY{p}{:} \PY{o}{\PYZhy{}}\PY{l+m+mi}{1} \PY{o}{+} \PY{l+m+mi}{2}\PY{o}{*}\PY{p}{(}\PY{n}{x}\PY{o}{\PYZhy{}}\PY{l+m+mi}{1}\PY{p}{)} \PY{o}{\PYZhy{}} \PY{l+m+mi}{4}\PY{o}{*}\PY{p}{(}\PY{n}{x}\PY{o}{\PYZhy{}}\PY{l+m+mi}{1}\PY{p}{)}\PY{o}{*}\PY{o}{*}\PY{l+m+mi}{3} \PY{o}{*}\PY{p}{(}\PY{n}{x}\PY{o}{\PYZhy{}}\PY{l+m+mi}{2}\PY{p}{)}
\PY{n}{dp} \PY{o}{=} \PY{k}{lambda} \PY{n}{t}\PY{p}{:} \PY{n}{sym}\PY{o}{.}\PY{n}{diff}\PY{p}{(}\PY{n}{p}\PY{p}{(}\PY{n}{sym}\PY{o}{.}\PY{n}{Symbol}\PY{p}{(}\PY{l+s+s1}{\PYZsq{}}\PY{l+s+s1}{x}\PY{l+s+s1}{\PYZsq{}}\PY{p}{)}\PY{p}{)}\PY{p}{,}\PY{n}{x}\PY{p}{)}\PY{o}{.}\PY{n}{subs}\PY{p}{(}\PY{n}{x}\PY{p}{,}\PY{n}{t}\PY{p}{)}
\PY{n}{d2p} \PY{o}{=} \PY{k}{lambda} \PY{n}{t}\PY{p}{:} \PY{n}{sym}\PY{o}{.}\PY{n}{diff}\PY{p}{(}\PY{n}{p}\PY{p}{(}\PY{n}{sym}\PY{o}{.}\PY{n}{Symbol}\PY{p}{(}\PY{l+s+s1}{\PYZsq{}}\PY{l+s+s1}{x}\PY{l+s+s1}{\PYZsq{}}\PY{p}{)}\PY{p}{)}\PY{p}{,}\PY{n}{x}\PY{p}{,}\PY{l+m+mi}{2}\PY{p}{)}\PY{o}{.}\PY{n}{subs}\PY{p}{(}\PY{n}{x}\PY{p}{,}\PY{n}{t}\PY{p}{)}

\PY{n}{p}\PY{p}{(}\PY{l+m+mi}{1}\PY{p}{)}\PY{p}{,}\PY{n}{p}\PY{p}{(}\PY{l+m+mi}{2}\PY{p}{)}\PY{p}{,}\PY{n}{dp}\PY{p}{(}\PY{l+m+mi}{1}\PY{p}{)}\PY{p}{,}\PY{n}{dp}\PY{p}{(}\PY{l+m+mi}{2}\PY{p}{)}\PY{p}{,}\PY{n}{d2p}\PY{p}{(}\PY{l+m+mi}{1}\PY{p}{)}
\end{InVerbatim}

            
\prompt{Out}{outcolor}{32}
\begin{OutVerbatim}[fontsize=\small,commandchars=\\\{\}]
(-1, 1, 2, -2, 0)
\end{OutVerbatim}
        

    % Add a bibliography block to the postdoc
    
    
    
    \end{document}
